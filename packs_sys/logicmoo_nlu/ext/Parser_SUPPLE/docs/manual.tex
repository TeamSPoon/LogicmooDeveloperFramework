\documentclass[a4paper,titlepage,openany,twoside]{book}
\usepackage[left=2cm,right=2cm,top=2cm,bottom=2cm]{geometry}
\usepackage{times}
\usepackage{graphicx}
\usepackage{amstext}
\usepackage{url}

\title{SUPPLE:\\The Sheffield University Prolog Parser for Language Engineering}
\author{Horacio Saggion and Mark A. Greenwood\\
Department of Computer Science\\University of Sheffield}

\begin{document}
\maketitle


{\huge A royalty-free license is granted for the use of this software for
NON\_COMMERCIAL PURPOSES ONLY.\\

The program is provided ``as is'' without warranty of any kind, either
expressed or implied, including, but not limited to, the implied
warranties of merchantability and fitness for a particular purpose.
The entire risk as to the quality and performance of the program is
with you.  Should the program prove defective, you assume the cost of
all necessary servicing, repair or correction.}

\chapter{Introduction}


SUPPLE is a bottom-up  parser that constructs syntax trees and
logical forms for English sentences. The parser is complete in the
sense that every analysis licensed by the grammar is produced. In the
current version only the `best' parser is selected at the end of the
parsing process. The English grammar (see Chapter~\ref{sec:grammar}) is implemented as an
attribute-value 
context free grammar which consists of subgrammars for  noun phrases (NP), verb phrases (VP), prepositional 
phrases (PP), relative phrases (R) and sentences (S). The semantics
 associated with each grammar rule allow the parser to produce logical
forms composed of unary predicates to denote  entities and
events (e.g., {\em chase(e1)},  {\em run(e2)}) and binary predicates
for properties  (e.g. {\em lsubj(e1,e2)}). Constants (e.g., $e1$,
$e2$) are used to represent entity and event identifiers.
The Gate SUPPLE Wrapper stores syntactic infomation produced by the
parser
in the gate document in the form of: SyntaxTreeNodes which are used to
display the parsing tree when the sentence is `edited'; `parse'
annotations containing a bracketed representation of the parse; and
`semantics' annotations that contains the logical forms produced by
the parser.  




\chapter{Software requirements}


In order to compile and use the parser you will need the following 
software:

\begin{itemize}

\item SUPPLE parser and Gate wrapper\\
Available from \url{http://nlp.shef.ac.uk/research/supple/}

\item The GATE Framework\\
Available from \url{http://gate.ac.uk}
 
\item Java (v1.4 or above)
Available from \url{http://www.javasoft.com}

\item Ant (to construct the parser)\\
Available from \url{http://ant.apache.org}

\item A Prolog implementation. We currently support:
\begin{itemize}
\item SICStus prolog (v3.8.6 or above)\\
For details see \url{http://www.sics.se/sicstus/}

\item PrologCafe (v0.9 or above)\\
Available from \url{http://kaminari.scitec.kobe-u.ac.jp/PrologCafe/}

\item SWI Prolog (v5.4.2 or above)\\
Available from \url{http://www.swi-prolog.org/}
\end{itemize}
See Chapter \ref{chap:prolog} for details of how to use other Prolog implementations.

\end{itemize}

\chapter{Distribution}


The present distribution (`supple' is the root)  contains the
following project directories and files:

\begin{itemize}

\item build.xml: the  build for ANT necessary to construct the system

\item classes: the java classes

\item config: contains the configuration files the wrapper needs to
translate information from Gate into SUPPLE. Two files can be found here
mapping.config and feature\_table.config. See
Section~\ref{sec:config} for an explanation of how the mapping is specified. 


\item docs: documentation. Contains this document.

\item lib: libraries necessary to compile and run different prolog
implementations.

\item src: the java source files.

\item creole.xml the configuration of the Gate SUPPLE Wrapper

\end{itemize}

\chapter{Building the system}

You will have to edit the build.xml file and adapt the `user
modifiable options' to your particular settings.

\begin{itemize}

\item {\em ant plcafe} constructs the PrologCafe version of the parser

\item {\em ant sicstus} constructs the SICStus version of the parser

\item {\em ant swi} constructts the SWIProlog version of the parser

\end{itemize}

\chapter{Running the parser in a Gate Gui}

To run the parser from the Gate GUI you need to load the creole.xml
that comes with this distribution, you have to this in the usual way
in Gate. If you want  a standalone application, just look at the
testing code provided in the source. In order to parse a document you will
need to construct an application that has:

\begin{itemize}
\item document reset (usefull)

\item tokeniser (mandatory)

\item splitter (mandatory)

\item POS-tagger (mandatory)

\item Morphology (mandatory)

\item Bottom-Up Chart Parser (mandatory) with parameters
\begin{itemize}
\item mapping file (config/mapping.config)
\item feature table file (config/feature\_table,mapping)
\item parser file (supple.plcafe, supple.sicstus or supple.swi)
\item the classname of the Prolog wrapper (i.e. shef.nlp.supple.prolog.SICStusProlog)
\end{itemize}
You can take a look at build.xml to see examples of invocation for
the different implementations.

\end{itemize} 

{\bf IMPORTANT NOTE:} The wrapper writes temporary files
in your temp directory, these files are deleted once the
process is finished {\em unless} the debug option is set to {\em true}
in which case the temporary files will be kept in your file system.
Do not forget to  turn the debug option to false in order to avoid 
disk space problems.

\chapter{Running the parser standalone}

One test for each configuration is provided with the build.xml. 

\begin{itemize}

\item {\em ant test.plcafe} tests the PrologCafe implementation

\item {\em ant test.sicstus} tests the Sicstus implementation

\item {\em ant test.swi} tests the SWI Prolog implementation


\end{itemize}

The program run is shef.nlp.supple.TestSuite which you can use as a
base for developing your own application.

\chapter{Configuration files}
\label{sec:config}

Two files are used to pass information from Gate to the SUPPLE
parser: the {\em mapping} file and the {\em feature table} file.\\

\section{Mapping file}
The mapping file specifies how annotations produced using Gate 
are to be passed to the parser.   
The file is composed of a number of  pairs of lines, the first line in
a pair specifies a Gate annotation  we want to pass to the parser. It includes the AnnotationSet (or
default), the AnnotationType, and a number of features and values that
depend on the AnnotationType.
The second line of the pair specifies how to encode the Gate
annotation in a SUPPLE syntactic category, this line also
includes a number of features and values.
As an example consider the mapping:

\begin{verbatim}
Gate;AnnotationType=Token;category=DT;string=&S
SUPPLE;category=dt;m_root=&S;s_form=&S
\end{verbatim}


It specifies how a determinant (`DT') will be translated into
a category `dt' for the parser. The construct `\&S' is used to
represent a variable that will be instantiated to the appropriate
value during the mapping process. More specifically a token like `The'
recognised as a DT by the POS-tagging will be mapped into the
following category:

\begin{verbatim}
dt(s_form:`The',m_root:`The',m_affix:`_',text:`_').
\end{verbatim}


As another example consider the mapping:

\begin{verbatim}
Gate;AnnotationType=Lookup;majorType=person_first;minorType=female;string=&S
SUPPLE;category=list_np;s_form=&S;ne_tag=person;ne_type=person_first;gender=female
\end{verbatim}


It specified that an annotation of type `Lookup' in Gate is mapped
into
a category `list\_np' with specific features and values. More
specifically a token like `Mary' identified in Gate as a Lookup will
be
mapped into the following SUPPLE category: 

\begin{verbatim}
list_np(s_form:`Mary',m_root:`_',m_affix:`_',
text:`_',ne_tag:`person',ne_type:`person_first',gender:`female').
\end{verbatim}



\section{Feature table}
\label{sec:features}

The feature table file specifies SUPPLE `lexical' categories and its
features.  As an example an entry in this file is:

\begin{verbatim}
n;s_form;m_root;m_affix;text;person;number
\end{verbatim}


which specifies which features and in which order a noun category
should be writen. In this case:

\begin{verbatim}
n(s_form:...,m_root:...,m_affix:...,text:...,person:...,number:....).
\end{verbatim}

\chapter{Parser and Grammar}
\label{sec:grammar}

The parser which is a Bottom-up Chart Parser builds a semantic representation
compositionally, and a `best parse' algorithm is applied to each final
chart, providing a partial parse if no complete sentence span can be
constructed. The parser uses a feature valued grammar.  Each \texttt{Category} entry has the form:
\begin{verbatim}
Category(Feature1:Value1,...,FeatureN:ValueN)
\end{verbatim}
where the number and type of features is dependent on the category
type (see Section ~\ref{sec:features}).  All categories will have the features \texttt{s\_form} (surface
form) and \texttt{m\_root} (morphological root); nominal and verbal
categories will also have \texttt{person} and \texttt{number}
features; verbal categories will also have \texttt{tense} and
\texttt{vform} features; and adjectival categories will have a
\texttt{degree} feature.  The \texttt{list\_np} category has the same
features as other nominal categories plus \texttt{ne\_tag} and
\texttt{ne\_type}.\\

Syntactic rules are specifed in Prolog with the predicate
$rule(LHS,RHS)$ where $LHS$ is a syntactic category and 
$RHS$ is a list of syntactic categories. A rule such as $BNP\_HEAD
\Rightarrow N$ (``a basic noun phrase head is composed of a noun'') is
writen as follows:

\begin{verbatim}
rule(bnp_head(sem:E^[[R,E],[number,E,N]],number:N), 
[n(m_root:R,number:N)]).
\end{verbatim}


Where the feature `sem' is used to construct the semantics while the
parser
processes input, and E, R, and N are variables tobe instantiated
during parsing.\\

The full grammar of this distribution can be found in the
prolog/grammar directory, the file load.pl specifies which grammars
are used by the parser. The grammars are compiled when the system is
built and the compied version is used for parsing.\\

\subsection{Mapping Named Entities}


SUPPLE has a prolog grammar which deals with named entities, the only
information required is the Lookup annotations
produced by Gate, which are specified in the mapping file.
 However, you may want to  pass named entities identified with your
own Jape grammars in 
Gate. This can be done using a special syntactic category provided 
with this distribution. The category
sem\_cat is used as a bridge between GATE named entities and the
SUPPLE grammar.
An example of how to use it (provided in the mapping file) is:

\begin{verbatim}
Gate;AnnotationType=Date;string=&S
SUPPLE;category=sem_cat;type=Date;text=&S;kind=date;name=&S
\end{verbatim}

which maps a named entity `Date' into a syntactic  category
`sem\_cat'.  A grammar file called semantic\_rules.pl is provided 
to map sem\_cat into the appropriate  syntactic category expected by
the phrasal rules. The following rule for example:

\begin{verbatim}
rule(ne_np(s_form:F,sem:X^[[name,X,NAME],[KIND,X]]),[
sem_cat(s_form:F,text:TEXT,type:`Date',kind:KIND,name:NAME)]).
\end{verbatim}


is used to parse a `Date' into a named entity in SUPPLE which in turn
will be parsed into a noun phrase.


\chapter{Using Other Prolog Implementations}
\label{chap:prolog}
You have to write a new Prolog wrapper by extending shef.nlp.supple.prolog.Prolog
and you will probably have to write a Prolog file to compile the parser (look at mkparser-*.pl)
for examples of how to do this.

\end{document}


