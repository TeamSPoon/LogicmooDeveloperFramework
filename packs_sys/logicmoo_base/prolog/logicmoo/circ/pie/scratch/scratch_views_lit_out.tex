\documentclass[a4paper]{article}
\usepackage[utf8]{inputenc}
\usepackage[T1]{fontenc}
\usepackage{imakeidx}
\usepackage[hidelinks]{hyperref}
%% A screen friendly geometry:	    
\usepackage[paper=a5paper,scale=0.9]{geometry}
%% PPL Setup
	   
\newcommand{\assign}{\mathrel{\mathop:}=}
\newcommand{\concat}{\mathrel{+\!+}}
	    
\newcommand{\f}[1]{\mathsf{#1}}
\newcommand{\true}{\top}
\newcommand{\false}{\bot}
\newcommand{\imp}{\rightarrow}
\newcommand{\revimp}{\leftarrow}
\newcommand{\equi}{\leftrightarrow}
\newcommand{\entails}{\models}	    
\newcommand{\eqdef}{\; 
\raisebox{-0.1ex}[0mm]{$ \stackrel{\raisebox{-0.2ex}{\tiny 
\textnormal{def}}}{=} $}\; }
\newcommand{\iffdef}{\n{iff}_{\mbox{\scriptsize \textnormal{def}}}}

\newcommand{\pplmacro}[1]{\mathit{#1}}
\newcommand{\ppldefmacro}[1]{\mathit{#1}}
\newcommand{\pplparam}[1]{\mathit{#1}}
\newcommand{\pplparamidx}[2]{\mathit{#1}_{#2}}
\newcommand{\pplparamplain}[1]{#1}
\newcommand{\pplparamplainidx}[2]{#1_{#2}}
\newcommand{\pplparamsup}[2]{\mathit{#1}^{#2}}
\newcommand{\pplparamsupidx}[3]{\mathit{#1}^{#2}_{#3}}
\newcommand{\pplparamplainsup}[2]{#1^{#2}}
\newcommand{\pplparamplainsupidx}[3]{#1^{#2}_{#3}}
\newcommand{\pplparamnum}[1]{\mathit{X}_{#1}}

%%	    
%% We use @startsection just to obtain reduced vertical spacing above
%% macro headers which are immediately after other headers, e.g. of sections
%%	    
\makeatletter%
\newcounter{entry}%
\newcommand{\entrymark}[1]{}%
\newcommand\entryhead{%
\@startsection{entry}{10}{\z@}{12pt plus 2pt minus 2pt}{0pt}{}}%
\makeatother
	    
\newcommand{\pplkbBefore}
{\entryhead*{}%
\setlength{\arraycolsep}{0pt}%
\pagebreak[0]%
\begin{samepage}%
\noindent%
\rule[0.5pt]{\textwidth}{2pt}\\%
\noindent}

% \newcommand{\pplkbDefType}[1]{\hspace{\fill}{{[}#1{]}\\}}

\newcommand{\pplkbBetween}
{\setlength{\arraycolsep}{3pt}%
\\\rule[3pt]{\textwidth}{1pt}%
\par\nopagebreak\noindent Defined as\begin{center}}

\newcommand{\pplkbAfter}{\end{center}\end{samepage}\noindent}

\newcommand{\pplkbBodyBefore}{\par\noindent where\begin{center}}
\newcommand{\pplkbBodyAfter}{\end{center}}

\newcommand{\pplkbFreePredicates}[1]{\f{free\_predicates}(#1)}
% \newcommand{\pplkbRenameFreeOccurrences}[3]{\f{rename\_free\_occurrences}(#1,#2,#3)}

\newcommand{\pplIsValid}[1]{\noindent This formula is valid: $#1$\par}
\newcommand{\pplIsNotValid}[1]{\noindent This formula is not valid: $#1$\par}	    
\newcommand{\pplFailedToValidate}[1]{\noindent Failed to validate this formula: $#1$\par}

\newcounter{def}
	    
\makeindex

\begin{document}
%
% Doc at position 0
%
\title{Access Predicates -- Examples from the Literature}
\date{Revision: March 28, 2019; Rendered: \today}
\maketitle

\noindent Some examples of solutions to view-based query processing and query
optimization tasks from the literature. Makes use of scratch\_forgetting and
scratch\_definientia. Formalized with the
\href{http://cs.christophwernhard.com/pie/}{\textit{PIE}} system.

\tableofcontents
%
% Doc at position 934
%
\section{Examples from Benedikt, ten Cate and Tsamoura: Generating Low-Cost
Plans from Proofs}

These examples stem from \cite{benedikt:etal:2014:generating}.  The numbering
of examples refers to that paper.  The representation here is different from
the paper -- (it seem new and is called here ``SV-modeling''). Also the
methods and solutions are different.
%
% Doc at position 1302
%
\subsection{Example 1}

Note: $\f{ia}$ in the background formula represents that the name (``smith'' in
the paper) is given. However $\f{ia}$ is actually not used to compute the
interpolant.
%
% Statement at position 1501
%
\pplkbBefore
\index{exbct_1_a@$\ppldefmacro{exbct\_1_{a}}$}$\begin{array}{lllll}
\ppldefmacro{exbct\_1_{a}}
\end{array}
$\pplkbBetween
$\begin{array}{lllll}
\forall \mathit{n}\mathit{o}\mathit{e} \, (\mathsf{i}\mathit{e} \imp  (\mathsf{profinfo_{a}}(\mathit{n},\mathit{o},\mathit{e}) \equi  \mathsf{profinfo}(\mathit{n},\mathit{o},\mathit{e}))) &&&&\; \land \\
\forall \mathit{n}\mathit{o}\mathit{e} \, (\mathsf{profinfo}(\mathit{n},\mathit{o},\mathit{e}) \imp  \mathsf{i}\mathit{n} \land  \mathsf{i}\mathit{o} \land  \mathsf{i}\mathit{e}).
\end{array}
$\pplkbAfter
%
% Statement at position 1636
%
\pplkbBefore
\index{exbct_1_b@$\ppldefmacro{exbct\_1_{b}}$}$\begin{array}{lllll}
\ppldefmacro{exbct\_1_{b}}
\end{array}
$\pplkbBetween
$\begin{array}{lllll}
\forall \mathit{n}\mathit{e} \, (\mathsf{udirect}(\mathit{n},\mathit{e}) \imp  \mathsf{i}\mathit{n} \land  \mathsf{i}\mathit{e}) &&&&\; \land \\
\forall \mathit{n}\mathit{o}\mathit{e} \, (\mathsf{profinfo}(\mathit{n},\mathit{o},\mathit{e}) \imp  \mathsf{udirect}(\mathit{n},\mathit{e})).
\end{array}
$\pplkbAfter
%
% Statement at position 1746
%
\pplkbBefore
\index{exbtc_1_c@$\ppldefmacro{exbtc\_1_{c}}$}$\begin{array}{lllll}
\ppldefmacro{exbtc\_1_{c}}
\end{array}
$\pplkbBetween
$\begin{array}{lllll}
\pplmacro{definiens}(\mathsf{profinfo}(\mathsf{a},\mathsf{b},\mathsf{c}),\\
\hphantom{\pplmacro{definiens}(} \pplmacro{exbct\_1_{a}} \land  \pplmacro{exbct\_1_{b}} \land  \mathsf{i}\mathsf{a},\\
\hphantom{\pplmacro{definiens}(} {[}\mathsf{profinfo_{a}},\mathsf{udirect}{]}).
\end{array}
$\pplkbAfter

\noindent Input: $\pplmacro{exbtc\_1_{c}}.$\\
\noindent Result of interpolation:
\[\begin{array}{lllll}
\mathsf{udirect}(\mathsf{a},\mathsf{c}) \land  \mathsf{profinfo_{a}}(\mathsf{a},\mathsf{b},\mathsf{c}).
\end{array}
\]
%
% Doc at position 1916
%
\noindent In the following formula the query is expressed more accurately with
existential midle argument, as described in Example~3 of
\cite{benedikt:etal:2014:generating}.
%
% Statement at position 2098
%
\pplkbBefore
\index{exbtc_1_d@$\ppldefmacro{exbtc\_1_{d}}$}$\begin{array}{lllll}
\ppldefmacro{exbtc\_1_{d}}
\end{array}
$\pplkbBetween
$\begin{array}{lllll}
\pplmacro{definiens}(\exists \mathit{o} \, \mathsf{profinfo}(\mathsf{a},\mathit{o},\mathsf{c}),\\
\hphantom{\pplmacro{definiens}(} \pplmacro{exbct\_1_{a}} \land  \pplmacro{exbct\_1_{b}} \land  \mathsf{i}\mathsf{a},\\
\hphantom{\pplmacro{definiens}(} {[}\mathsf{profinfo_{a}},\mathsf{udirect}{]}).
\end{array}
$\pplkbAfter

\noindent Input: $\pplmacro{exbtc\_1_{d}}.$\\
\noindent Result of interpolation:
\[\begin{array}{lllll}
\exists \mathit{x} \, (\mathsf{udirect}(\mathsf{a},\mathsf{c}) \land  \mathsf{profinfo_{a}}(\mathsf{a},\mathit{x},\mathsf{c})).
\end{array}
\]
%
% Doc at position 2277
%
\subsection{Variant of Example 1}

This is the variant of Example~1 from \cite[p.~101 Left
Column]{benedikt:etal:2014:generating}.
It leads to different interpolants, obtained with the enum\_ips option.
Here the first 3 interpolants are shown.
%
% Statement at position 2529
%
\pplkbBefore
\index{exbtc_1_e@$\ppldefmacro{exbtc\_1_{e}}$}$\begin{array}{lllll}
\ppldefmacro{exbtc\_1_{e}}
\end{array}
$\pplkbBetween
$\begin{array}{lllll}
\pplmacro{definiens}(\mathsf{profinfo}(\mathsf{a},\mathsf{b},\mathsf{c}),\\
\hphantom{\pplmacro{definiens}(} \pplmacro{exbct\_1_{a}} &&&\; \land \\
\hphantom{\pplmacro{definiens}(} \forall \mathit{n}\mathit{e} \, (\mathsf{udirect_{1}}(\mathit{n},\mathit{e}) \imp  \mathsf{i}\mathit{n} \land  \mathsf{i}\mathit{e}) &&&\; \land \\
\hphantom{\pplmacro{definiens}(} \forall \mathit{n}\mathit{o}\mathit{e} \, (\mathsf{profinfo}(\mathit{n},\mathit{o},\mathit{e}) \imp  \mathsf{udirect_{1}}(\mathit{n},\mathit{e})) &&&\; \land \\
\hphantom{\pplmacro{definiens}(} \forall \mathit{n}\mathit{e} \, (\mathsf{udirect_{2}}(\mathit{n},\mathit{e}) \imp  \mathsf{i}\mathit{n} \land  \mathsf{i}\mathit{e}) &&&\; \land \\
\hphantom{\pplmacro{definiens}(} \forall \mathit{n}\mathit{o}\mathit{e} \, (\mathsf{profinfo}(\mathit{n},\mathit{o},\mathit{e}) \imp  \mathsf{udirect_{2}}(\mathit{n},\mathit{e})) &&&\; \land \\
\hphantom{\pplmacro{definiens}(} \mathsf{i}\mathsf{a},\\
\hphantom{\pplmacro{definiens}(} {[}\mathsf{profinfo_{a}},\mathsf{udirect_{1}},\mathsf{udirect_{2}}{]}).
\end{array}
$\pplkbAfter

\noindent Input: $\pplmacro{exbtc\_1_{e}}.$\\
\noindent Result of interpolation:
\[\begin{array}{lllll}
\mathsf{udirect_{1}}(\mathsf{a},\mathsf{c}) \land  \mathsf{profinfo_{a}}(\mathsf{a},\mathsf{b},\mathsf{c}).
\end{array}
\]

\noindent Input: $\pplmacro{exbtc\_1_{e}}.$\\
\noindent Result of interpolation:
\[\begin{array}{lllll}
\mathsf{udirect_{2}}(\mathsf{a},\mathsf{c}) \land  \mathsf{profinfo_{a}}(\mathsf{a},\mathsf{b},\mathsf{c}).
\end{array}
\]

\noindent Input: $\pplmacro{exbtc\_1_{e}}.$\\
\noindent Result of interpolation:
\[\begin{array}{lllll}
\mathsf{udirect_{1}}(\mathsf{a},\mathsf{c}) \land  \mathsf{udirect_{2}}(\mathsf{a},\mathsf{c}) \land  \mathsf{profinfo_{a}}(\mathsf{a},\mathsf{b},\mathsf{c}).
\end{array}
\]
%
% Doc at position 3171
%
\subsection{Example~4}

Note that we have the id as last argument of profinfo, following the natural
language description of Example~1 of the paper. In the formal version, of
Example~4 in the paper the id is the first argument.
The $i(a)$ has been dropped here.
%
% Statement at position 3441
%
\pplkbBefore
\index{exbtc_4_a@$\ppldefmacro{exbtc\_4_{a}}$}$\begin{array}{lllll}
\ppldefmacro{exbtc\_4_{a}}
\end{array}
$\pplkbBetween
$\begin{array}{lllll}
\pplmacro{definiens}(\exists \mathit{c} \, \mathsf{profinfo}(\mathsf{a},\mathsf{o},\mathit{c}),\\
\hphantom{\pplmacro{definiens}(} \pplmacro{exbct\_1_{a}} \land  \pplmacro{exbct\_1_{b}},\\
\hphantom{\pplmacro{definiens}(} {[}\mathsf{profinfo_{a}},\mathsf{udirect}{]}).
\end{array}
$\pplkbAfter

\noindent Input: $\pplmacro{exbtc\_4_{a}}.$\\
\noindent Result of interpolation:
\[\begin{array}{lllll}
\exists \mathit{x} \, (\mathsf{udirect}(\mathsf{a},\mathit{x}) \land  \mathsf{profinfo_{a}}(\mathsf{a},\mathsf{o},\mathit{x})).
\end{array}
\]
%
% Doc at position 3610
%
\subsection{Example~5}

Note that we have the id as last argument of profinfo, following the natural
language description of Example~1 of the paper. In the formal version, of
Examples~4 and 5 in the paper the id is the first argument.

Here we need $i(b)$ because, as described in the paper, the access to profinfo
requires all arguments bound, where only the first and last can be bound at
all by udirect. Perhaps this is a bug in the paper.

The point of the example seems that the method of the paper involves all three
udirect accesses (in some order), but it is hard to see why this is useful,
when the query could be answered by accessing just one of them.
%
% Statement at position 4281
%
\pplkbBefore
\index{exbtc_5_a@$\ppldefmacro{exbtc\_5_{a}}$}$\begin{array}{lllll}
\ppldefmacro{exbtc\_5_{a}}
\end{array}
$\pplkbBetween
$\begin{array}{lllll}
\pplmacro{definiens}(\exists \mathit{c} \, \mathsf{profinfo}(\mathsf{a},\mathsf{b},\mathit{c}),\\
\hphantom{\pplmacro{definiens}(} \forall \mathit{n}\mathit{o}\mathit{e} \, (\mathsf{i}\mathit{n} \land  \mathsf{i}\mathit{o} \land  \mathsf{i}\mathit{e} \imp  (\mathsf{profinfo_{a}}(\mathit{n},\mathit{o},\mathit{e}) \equi  \mathsf{profinfo}(\mathit{n},\mathit{o},\mathit{e}))) &&&\; \land \\
\hphantom{\pplmacro{definiens}(} \forall \mathit{n}\mathit{o}\mathit{e} \, (\mathsf{profinfo}(\mathit{n},\mathit{o},\mathit{e}) \imp  \mathsf{i}\mathit{n} \land  \mathsf{i}\mathit{o} \land  \mathsf{i}\mathit{e}) &&&\; \land \\
\hphantom{\pplmacro{definiens}(} \pplmacro{exbct\_1_{a}} &&&\; \land \\
\hphantom{\pplmacro{definiens}(} \forall \mathit{n}\mathit{e} \, (\mathsf{udirect_{1}}(\mathit{n},\mathit{e}) \imp  \mathsf{i}\mathit{n} \land  \mathsf{i}\mathit{e}) &&&\; \land \\
\hphantom{\pplmacro{definiens}(} \forall \mathit{n}\mathit{o}\mathit{e} \, (\mathsf{profinfo}(\mathit{n},\mathit{o},\mathit{e}) \imp  \mathsf{udirect_{1}}(\mathit{n},\mathit{e})) &&&\; \land \\
\hphantom{\pplmacro{definiens}(} \forall \mathit{n}\mathit{e} \, (\mathsf{udirect_{2}}(\mathit{n},\mathit{e}) \imp  \mathsf{i}\mathit{n} \land  \mathsf{i}\mathit{e}) &&&\; \land \\
\hphantom{\pplmacro{definiens}(} \forall \mathit{n}\mathit{o}\mathit{e} \, (\mathsf{profinfo}(\mathit{n},\mathit{o},\mathit{e}) \imp  \mathsf{udirect_{2}}(\mathit{n},\mathit{e})) &&&\; \land \\
\hphantom{\pplmacro{definiens}(} \forall \mathit{n}\mathit{e} \, (\mathsf{udirect_{3}}(\mathit{n},\mathit{e}) \imp  \mathsf{i}\mathit{n} \land  \mathsf{i}\mathit{e}) &&&\; \land \\
\hphantom{\pplmacro{definiens}(} \forall \mathit{n}\mathit{o}\mathit{e} \, (\mathsf{profinfo}(\mathit{n},\mathit{o},\mathit{e}) \imp  \mathsf{udirect_{3}}(\mathit{n},\mathit{e})) &&&\; \land \\
\hphantom{\pplmacro{definiens}(} \mathsf{i}\mathsf{b},\\
\hphantom{\pplmacro{definiens}(} {[}\mathsf{profinfo_{a}},\mathsf{udirect_{1}},\mathsf{udirect_{2}},\mathsf{udirect_{3}}{]}).
\end{array}
$\pplkbAfter

\noindent Input: $\pplmacro{exbtc\_5_{a}}.$\\
\noindent Result of interpolation:
\[\begin{array}{lllll}
\exists \mathit{x} \, (\mathsf{udirect_{1}}(\mathsf{a},\mathit{x}) \land  \mathsf{profinfo_{a}}(\mathsf{a},\mathsf{b},\mathit{x})).
\end{array}
\]

\noindent Input: $\pplmacro{exbtc\_5_{a}}.$\\
\noindent Result of interpolation:
\[\begin{array}{lllll}
\exists \mathit{x} \, (\mathsf{udirect_{2}}(\mathsf{a},\mathit{x}) \land  \mathsf{profinfo_{a}}(\mathsf{a},\mathsf{b},\mathit{x})).
\end{array}
\]
%
% Doc at position 5042
%
\subsection{Example~2}

There seems a bug in the paper: Actually a referential constraint from direct2
into direct1 w.r.t. $n,a$ is required here. The paper suggests the reverse direction.
%
% Statement at position 5239
%
\pplkbBefore
\index{exbtc_2_schema@$\ppldefmacro{exbtc\_2\_schema}$}$\begin{array}{lllll}
\ppldefmacro{exbtc\_2\_schema}
\end{array}
$\pplkbBetween
$\begin{array}{lllll}
\forall \mathit{n}\mathit{a}\mathit{u} \, (\mathsf{i}\mathit{n} \land  \mathsf{i}\mathit{u} \imp  (\mathsf{direct1_{a}}(\mathit{n},\mathit{a},\mathit{u}) \equi  \mathsf{direct_{1}}(\mathit{n},\mathit{a},\mathit{u}))) &&&&\; \land \\
\forall \mathit{n}\mathit{a}\mathit{u} \, (\mathsf{direct_{1}}(\mathit{n},\mathit{a},\mathit{u}) &&&\; \imp \\
\hphantom{\forall \mathit{n}\mathit{a}\mathit{u} \, (} \mathsf{i}\mathit{n} \land  \mathsf{i}\mathit{a} \land  \mathsf{i}\mathit{u}) &&&&\; \land \\
\forall \mathit{n}\mathit{a}\mathit{u} \, (\mathsf{direct_{1}}(\mathit{n},\mathit{a},\mathit{u}) \imp  \mathsf{ids}(\mathit{u})) &&&&\; \land \\
\forall \mathit{u} \, (\mathsf{ids}(\mathit{u}) \imp  \mathsf{i}\mathit{u}) &&&&\; \land \\
\forall \mathit{n}\mathit{a}\mathit{p} \, (\mathsf{i}\mathit{n} \land  \mathsf{i}\mathit{a} \imp  (\mathsf{direct2_{a}}(\mathit{n},\mathit{a},\mathit{p}) \equi  \mathsf{direct_{2}}(\mathit{n},\mathit{a},\mathit{p}))) &&&&\; \land \\
\forall \mathit{n}\mathit{a}\mathit{p} \, (\mathsf{direct_{2}}(\mathit{n},\mathit{a},\mathit{p}) &&&\; \imp \\
\hphantom{\forall \mathit{n}\mathit{a}\mathit{p} \, (} \mathsf{i}\mathit{n} \land  \mathsf{i}\mathit{a} \land  \mathsf{i}\mathit{p}) &&&&\; \land \\
\forall \mathit{n}\mathit{a}\mathit{p} \, (\mathsf{direct_{2}}(\mathit{n},\mathit{a},\mathit{p}) \imp  \mathsf{names}(\mathit{n})) &&&&\; \land \\
\forall \mathit{n} \, (\mathsf{names}(\mathit{n}) \imp  \mathsf{i}\mathit{n}) &&&&\; \land \\
\forall \mathit{n}\mathit{a}\mathit{p} \, (\mathsf{direct_{2}}(\mathit{n},\mathit{a},\mathit{p}) \imp  \exists \mathit{u} \, \mathsf{direct_{1}}(\mathit{n},\mathit{a},\mathit{u})).
\end{array}
$\pplkbAfter
%
% Statement at position 5716
%
\pplkbBefore
\index{exbtc_2_a@$\ppldefmacro{exbtc\_2_{a}}$}$\begin{array}{lllll}
\ppldefmacro{exbtc\_2_{a}}
\end{array}
$\pplkbBetween
$\begin{array}{lllll}
\pplmacro{definiens}(\exists \mathit{n}\mathit{a} \, \mathsf{direct_{2}}(\mathit{n},\mathit{a},\mathsf{p}),\\
\hphantom{\pplmacro{definiens}(} \pplmacro{exbtc\_2\_schema},\\
\hphantom{\pplmacro{definiens}(} {[}\mathsf{direct1_{a}},\mathsf{direct2_{a}},\mathsf{ids},\mathsf{names}{]}).
\end{array}
$\pplkbAfter

\noindent Input: $\pplmacro{exbtc\_2_{a}}.$\\
\noindent Result of interpolation:
\[\begin{array}{lllll}
\exists \mathit{x},\mathit{y},\mathit{z} \, (\mathsf{ids}(\mathit{z}) &&&&\; \land \\
\hphantom{\exists \mathit{x},\mathit{y},\mathit{z} \, (} \mathsf{names}(\mathit{x}) &&&&\; \land \\
\hphantom{\exists \mathit{x},\mathit{y},\mathit{z} \, (} \mathsf{direct1_{a}}(\mathit{x},\mathit{y},\mathit{z}) &&&&\; \land \\
\hphantom{\exists \mathit{x},\mathit{y},\mathit{z} \, (} \mathsf{direct2_{a}}(\mathit{x},\mathit{y},\mathsf{p})).
\end{array}
\]
%
% Doc at position 5967
%
\section{Examples from Toman and Wedell: Fundamentals of
Physical Design and Query Compilation}

These examples are from \cite[Chapters~3 and~5]{toman:wedell:book}.
%
% Doc at position 6141
%
\subsection{Example~5.14}
%
% Statement at position 6175
%
\pplkbBefore
\index{extw_514_a@$\ppldefmacro{extw\_514_{a}}$}$\begin{array}{lllll}
\ppldefmacro{extw\_514_{a}}
\end{array}
$\pplkbBetween
$\begin{array}{lllll}
\forall \mathit{x}\mathit{y} \, (\mathsf{v_{1}}\mathit{x}\mathit{y} \equi  \exists \mathit{u}\mathit{w} \, (\mathsf{r}\mathit{u}\mathit{x} \land  \mathsf{r}\mathit{u}\mathit{w} \land  \mathsf{r}\mathit{w}\mathit{y})) &&&&\; \land \\
\forall \mathit{x}\mathit{y} \, (\mathsf{v_{2}}\mathit{x}\mathit{y} \equi  \exists \mathit{u}\mathit{w} \, (\mathsf{r}\mathit{x}\mathit{u} \land  \mathsf{r}\mathit{u}\mathit{w} \land  \mathsf{r}\mathit{w}\mathit{y})) &&&&\; \land \\
\forall \mathit{x}\mathit{y} \, (\mathsf{v_{3}}\mathit{x}\mathit{y} \equi  \exists \mathit{u} \, (\mathsf{r}\mathit{x}\mathit{u} \land  \mathsf{r}\mathit{u}\mathit{y})).
\end{array}
$\pplkbAfter
%
% Statement at position 6370
%
\pplkbBefore
\index{extw_514_b@$\ppldefmacro{extw\_514_{b}}$}$\begin{array}{lllll}
\ppldefmacro{extw\_514_{b}}
\end{array}
$\pplkbBetween
$\begin{array}{lllll}
\pplmacro{definiens}(\exists \mathit{u}\mathit{v}\mathit{w} \, (\mathsf{r}\mathit{u}\mathsf{x} \land  \mathsf{r}\mathit{u}\mathit{w} \land  \mathsf{r}\mathit{w}\mathit{v} \land  \mathsf{r}\mathit{v}\mathsf{y}),\\
\hphantom{\pplmacro{definiens}(} \pplmacro{extw\_514_{a}},\\
\hphantom{\pplmacro{definiens}(} {[}\mathsf{v_{1}},\mathsf{v_{2}},\mathsf{v_{3}}{]}).
\end{array}
$\pplkbAfter

\noindent Input: $\pplmacro{extw\_514_{b}}.$\\
\noindent Result of interpolation:
\[\begin{array}{lllll}
\exists \mathit{z} \, \forall \mathit{u} \, (\mathsf{v_{1}}(\mathsf{x},\mathit{z}) \land  (\mathsf{v_{3}}(\mathit{u},\mathit{z}) \imp  \mathsf{v_{2}}(\mathit{u},\mathsf{y}))).
\end{array}
\]
%
% Doc at position 6579
%
Notes: The book shows this solution, but also another, longer formula which is
then used as basis for plan generation. The longer formula seems not easily to
obtain as alternative interpolant with CM prover.
%
% Statement at position 6795
%
\pplkbBefore
\index{extw_514_altsol@$\ppldefmacro{extw\_514\_altsol}$}$\begin{array}{lllll}
\ppldefmacro{extw\_514\_altsol}
\end{array}
$\pplkbBetween
$\begin{array}{lllll}
\exists \mathit{u}\mathit{v} \, (\mathsf{v_{1}}\mathsf{x}\mathit{u} \land  \mathsf{v_{3}}\mathit{v}\mathit{u} \land  \mathsf{v_{2}}\mathit{v}\mathsf{y} \land  \forall \mathit{w} \, (\lnot  \mathsf{v_{3}}\mathit{w}\mathit{u} \lor  \mathsf{v_{2}}\mathit{w}\mathsf{y})).
\end{array}
$\pplkbAfter
%
% Statement at position 6890
%
\pplkbBefore
\index{extw_514_query@$\ppldefmacro{extw\_514\_query}$}$\begin{array}{lllll}
\ppldefmacro{extw\_514\_query}
\end{array}
$\pplkbBetween
$\begin{array}{lllll}
\exists \mathit{u}\mathit{v}\mathit{w} \, (\mathsf{r}\mathit{u}\mathsf{x} \land  \mathsf{r}\mathit{u}\mathit{w} \land  \mathsf{r}\mathit{w}\mathit{v} \land  \mathsf{r}\mathit{v}\mathsf{y}).
\end{array}
$\pplkbAfter
%
% Statement at position 6958
%
\pplkbBefore
\index{extw_514_check_altsol_1@$\ppldefmacro{extw\_514\_check\_altsol_{1}}$}$\begin{array}{lllll}
\ppldefmacro{extw\_514\_check\_altsol_{1}}
\end{array}
$\pplkbBetween
$\begin{array}{lllll}
\pplmacro{extw\_514_{a}} \imp  (\pplmacro{extw\_514\_altsol} \revimp  \pplmacro{extw\_514\_query}).
\end{array}
$\pplkbAfter
%
% Statement at position 7042
%
\pplkbBefore
\index{extw_514_check_altsol_2@$\ppldefmacro{extw\_514\_check\_altsol_{2}}$}$\begin{array}{lllll}
\ppldefmacro{extw\_514\_check\_altsol_{2}}
\end{array}
$\pplkbBetween
$\begin{array}{lllll}
\pplmacro{extw\_514_{a}} \imp  (\pplmacro{extw\_514\_altsol} \imp  \pplmacro{extw\_514\_query}).
\end{array}
$\pplkbAfter
%
% Doc at position 7126
%
The next formula uses literal forgetting for $v3$.  This seems hard for the
CM prover (12 sec, 3 of them in the last depth 7). When literal forgetting is
used to restrict polarities for all three access paths, i.e.,
$[v1-p,v2-p,v3-n]$, the CM prover does not succeed in a few minutes.
%
% Statement at position 7419
%
\pplkbBefore
\index{extw_514_c@$\ppldefmacro{extw\_514_{c}}$}$\begin{array}{lllll}
\ppldefmacro{extw\_514_{c}}
\end{array}
$\pplkbBetween
$\begin{array}{lllll}
\pplmacro{definiens\_lit}(\exists \mathit{u}\mathit{v}\mathit{w} \, (\mathsf{r}\mathit{u}\mathsf{x} \land  \mathsf{r}\mathit{u}\mathit{w} \land  \mathsf{r}\mathit{w}\mathit{v} \land  \mathsf{r}\mathit{v}\mathsf{y}),\\
\hphantom{\pplmacro{definiens\_lit}(} \pplmacro{extw\_514_{a}},\\
\hphantom{\pplmacro{definiens\_lit}(} {[}\mathsf{v_{1}}\textrm{-}\mathsf{pn},\mathsf{v_{2}}\textrm{-}\mathsf{pn},\mathsf{v_{3}}\textrm{-}\mathsf{n}{]}).
\end{array}
$\pplkbAfter
%
% Statement at position 7536
%
\pplkbBefore
\index{extw_514_d@$\ppldefmacro{extw\_514_{d}}$}$\begin{array}{lllll}
\ppldefmacro{extw\_514_{d}}
\end{array}
$\pplkbBetween
$\begin{array}{lllll}
\pplmacro{definiens\_lit\_lemma}(\exists \mathit{u}\mathit{v}\mathit{w} \, (\mathsf{r}\mathit{u}\mathsf{x} \land  \mathsf{r}\mathit{u}\mathit{w} \land  \mathsf{r}\mathit{w}\mathit{v} \land  \mathsf{r}\mathit{v}\mathsf{y}),\\
\hphantom{\pplmacro{definiens\_lit\_lemma}(} \pplmacro{extw\_514_{a}},\\
\hphantom{\pplmacro{definiens\_lit\_lemma}(} {[}\mathsf{v_{1}}\textrm{-}\mathsf{pn},\mathsf{v_{2}}\textrm{-}\mathsf{pn},\mathsf{v_{3}}\textrm{-}\mathsf{n}{]}).
\end{array}
$\pplkbAfter
%
% Statement at position 7659
%
\pplkbBefore
\index{extw_514_e@$\ppldefmacro{extw\_514_{e}}$}$\begin{array}{lllll}
\ppldefmacro{extw\_514_{e}}
\end{array}
$\pplkbBetween
$\begin{array}{lllll}
\pplmacro{definiens\_lit\_lemma}(\exists \mathit{u}\mathit{v}\mathit{w} \, (\mathsf{r}\mathit{u}\mathsf{x} \land  \mathsf{r}\mathit{u}\mathit{w} \land  \mathsf{r}\mathit{w}\mathit{v} \land  \mathsf{r}\mathit{v}\mathsf{y}),\\
\hphantom{\pplmacro{definiens\_lit\_lemma}(} \pplmacro{extw\_514_{a}},\\
\hphantom{\pplmacro{definiens\_lit\_lemma}(} {[}\mathsf{v_{1}}\textrm{-}\mathsf{p},\mathsf{v_{2}}\textrm{-}\mathsf{p},\mathsf{v_{3}}\textrm{-}\mathsf{n}{]}).
\end{array}
$\pplkbAfter
%
% Doc at position 7837
%
\subsection{Examples~3.2, 3.4, 3.5}

These examples are also discussed in the book in Examples~5.3 and 5.4.

The access paths have arguments \textit{Salary}, \textit{Number},
\textit{Name}. The employee/3 relation has these arguments in the ordering
\textit{Number}, \textit{Name} \textit{Salary}.

We give the access paths here different number suffixes than in the book to
utilize that lexically smaller predicates are preferred by the interpolant
computation (the ordp option of the CM prover) and thus returned in earlier
solutions.

NOTE: Currently the ordp option prefers solution with lexically smaller
predicates, however for the price of possibly choosing a larger clause, which
might introduce redundancies. The ordp option is not required if access
patterns are disjoint.
%
% Statement at position 8628
%
\pplkbBefore
\index{extw_3_schema@$\ppldefmacro{extw\_3\_schema}$}$\begin{array}{lllll}
\ppldefmacro{extw\_3\_schema}
\end{array}
$\pplkbBetween
$\begin{array}{lllll}
\forall \mathit{x}\mathit{y}\mathit{z} \, (\mathsf{emp\_array_{3}}(\mathit{z},\mathit{x},\mathit{y}) \equi  \mathsf{employee}(\mathit{x},\mathit{y},\mathit{z})) &&&&\; \land \\
\forall \mathit{x}\mathit{y}\mathit{z} \, (\mathsf{i}\mathit{z} \imp  (\mathsf{emp\_array_{2}}(\mathit{z},\mathit{x},\mathit{y}) \equi  \mathsf{employee}(\mathit{x},\mathit{y},\mathit{z}))) &&&&\; \land \\
\forall \mathit{x}\mathit{y}\mathit{z} \, (\mathsf{i}\mathit{z} \land  \mathsf{i}\mathit{x} \imp  (\mathsf{emp\_array_{1}}(\mathit{z},\mathit{x},\mathit{y}) \equi  \mathsf{employee}(\mathit{x},\mathit{y},\mathit{z}))) &&&&\; \land \\
\forall \mathit{x}\mathit{y}\mathit{z} \, (\mathsf{employee}(\mathit{x},\mathit{y},\mathit{z}) \imp  \mathsf{i}\mathit{x} \land  \mathsf{i}\mathit{y} \land  \mathsf{i}\mathit{z}).
\end{array}
$\pplkbAfter
%
% Statement at position 8894
%
\pplkbBefore
\index{extw_32@$\ppldefmacro{extw_{32}}$}$\begin{array}{lllll}
\ppldefmacro{extw_{32}}
\end{array}
$\pplkbBetween
$\begin{array}{lllll}
\pplmacro{definiens}(\mathsf{employee}(\mathsf{x},\mathsf{y},\mathsf{z}),\\
\hphantom{\pplmacro{definiens}(} \pplmacro{extw\_3\_schema},\\
\hphantom{\pplmacro{definiens}(} {[}\mathsf{emp\_array_{1}},\mathsf{emp\_array_{2}},\mathsf{emp\_array_{3}}{]}).
\end{array}
$\pplkbAfter
%
% Statement at position 9005
%
\pplkbBefore
\index{extw_34@$\ppldefmacro{extw_{34}}$}$\begin{array}{lllll}
\ppldefmacro{extw_{34}}
\end{array}
$\pplkbBetween
$\begin{array}{lllll}
\pplmacro{definiens}(\mathsf{employee}(\mathsf{x},\mathsf{y},\mathsf{z}),\\
\hphantom{\pplmacro{definiens}(} \pplmacro{extw\_3\_schema} \land  \mathsf{i}\mathsf{z},\\
\hphantom{\pplmacro{definiens}(} {[}\mathsf{emp\_array_{1}},\mathsf{emp\_array_{2}},\mathsf{emp\_array_{3}}{]}).
\end{array}
$\pplkbAfter
%
% Statement at position 9126
%
\pplkbBefore
\index{extw_35@$\ppldefmacro{extw_{35}}$}$\begin{array}{lllll}
\ppldefmacro{extw_{35}}
\end{array}
$\pplkbBetween
$\begin{array}{lllll}
\pplmacro{definiens}(\mathsf{employee}(\mathsf{x},\mathsf{y},\mathsf{z}),\\
\hphantom{\pplmacro{definiens}(} \pplmacro{extw\_3\_schema} \land  \mathsf{i}\mathsf{x} \land  \mathsf{i}\mathsf{z},\\
\hphantom{\pplmacro{definiens}(} {[}\mathsf{emp\_array_{1}},\mathsf{emp\_array_{2}},\mathsf{emp\_array_{3}}{]}).
\end{array}
$\pplkbAfter

\noindent Input: $\pplmacro{extw_{32}}.$\\
\noindent Result of interpolation:
\[\begin{array}{lllll}
\mathsf{emp\_array_{3}}(\mathsf{z},\mathsf{x},\mathsf{y}).
\end{array}
\]

\noindent Input: $\pplmacro{extw_{34}}.$\\
\noindent Result of interpolation:
\[\begin{array}{lllll}
\mathsf{emp\_array_{2}}(\mathsf{z},\mathsf{x},\mathsf{y}).
\end{array}
\]

\noindent Input: $\pplmacro{extw_{35}}.$\\
\noindent Result of interpolation:
\[\begin{array}{lllll}
\mathsf{emp\_array_{1}}(\mathsf{z},\mathsf{x},\mathsf{y}).
\end{array}
\]
%
% Doc at position 9478
%
\subsection{Alternate Modeling, Similar to ``Option 3''}

This is just similar to ``Option~3'' in the book, but we use employee numbers
directly as IDs of employees. We map the schema to employee/3:
%
% Statement at position 9685
%
\pplkbBefore
\index{extw_3_o3_schema_addition@$\ppldefmacro{extw\_3\_o3\_schema\_addition}$}$\begin{array}{lllll}
\ppldefmacro{extw\_3\_o3\_schema\_addition}
\end{array}
$\pplkbBetween
$\begin{array}{lllll}
\forall \mathit{x} \, (\mathsf{emp}(\mathit{x}) \equi  \exists \mathit{y}\mathit{z} \, \mathsf{employee}(\mathit{x},\mathit{y},\mathit{z})) &&&&\; \land \\
\forall \mathit{x}\mathit{y} \, (\mathsf{name}(\mathit{x},\mathit{y}) \equi  \exists \mathit{z} \, \mathsf{employee}(\mathit{x},\mathit{y},\mathit{z})) &&&&\; \land \\
\forall \mathit{x}\mathit{y} \, (\mathsf{salary}(\mathit{x},\mathit{y}) \equi  \exists \mathit{z} \, \mathsf{employee}(\mathit{x},\mathit{z},\mathit{y})).
\end{array}
$\pplkbAfter
%
% Statement at position 10007
%
\pplkbBefore
\index{extw_o3_32@$\ppldefmacro{extw\_o3_{32}}$}$\begin{array}{lllll}
\ppldefmacro{extw\_o3_{32}}
\end{array}
$\pplkbBetween
$\begin{array}{lllll}
\pplmacro{definiens}(\mathsf{emp}(\mathsf{x}) \land  \mathsf{name}(\mathsf{x},\mathsf{y}) \land  \mathsf{salary}(\mathsf{x},\mathsf{z}),\\
\hphantom{\pplmacro{definiens}(} \pplmacro{extw\_3\_schema} \land  \pplmacro{extw\_3\_o3\_schema\_addition},\\
\hphantom{\pplmacro{definiens}(} {[}\mathsf{emp\_array_{1}},\mathsf{emp\_array_{2}},\mathsf{emp\_array_{3}}{]}).
\end{array}
$\pplkbAfter
%
% Statement at position 10167
%
\pplkbBefore
\index{extw_o3_34@$\ppldefmacro{extw\_o3_{34}}$}$\begin{array}{lllll}
\ppldefmacro{extw\_o3_{34}}
\end{array}
$\pplkbBetween
$\begin{array}{lllll}
\pplmacro{definiens}(\mathsf{emp}(\mathsf{x}) \land  \mathsf{name}(\mathsf{x},\mathsf{y}) \land  \mathsf{salary}(\mathsf{x},\mathsf{z}),\\
\hphantom{\pplmacro{definiens}(} \pplmacro{extw\_3\_schema} \land  \pplmacro{extw\_3\_o3\_schema\_addition} \land  \mathsf{i}\mathsf{z},\\
\hphantom{\pplmacro{definiens}(} {[}\mathsf{emp\_array_{1}},\mathsf{emp\_array_{2}},\mathsf{emp\_array_{3}}{]}).
\end{array}
$\pplkbAfter
%
% Statement at position 10333
%
\pplkbBefore
\index{extw_o3_35@$\ppldefmacro{extw\_o3_{35}}$}$\begin{array}{lllll}
\ppldefmacro{extw\_o3_{35}}
\end{array}
$\pplkbBetween
$\begin{array}{lllll}
\pplmacro{definiens}(\mathsf{emp}(\mathsf{x}) \land  \mathsf{name}(\mathsf{x},\mathsf{y}) \land  \mathsf{salary}(\mathsf{x},\mathsf{z}),\\
\hphantom{\pplmacro{definiens}(} \pplmacro{extw\_3\_schema} \land  \pplmacro{extw\_3\_o3\_schema\_addition} \land  \mathsf{i}\mathsf{x} \land  \mathsf{i}\mathsf{z},\\
\hphantom{\pplmacro{definiens}(} {[}\mathsf{emp\_array_{1}},\mathsf{emp\_array_{2}},\mathsf{emp\_array_{3}}{]}).
\end{array}
$\pplkbAfter

\noindent Input: $\pplmacro{extw\_o3_{32}}.$\\
\noindent Result of interpolation:
\[\begin{array}{lllll}
\exists \mathit{u},\mathit{v} \, (\mathsf{emp\_array_{3}}(\mathsf{z},\mathsf{x},\mathit{v}) \land  \mathsf{emp\_array_{3}}(\mathit{u},\mathsf{x},\mathsf{y})).
\end{array}
\]

\noindent Input: $\pplmacro{extw\_o3_{34}}.$\\
\noindent Result of interpolation:
\[\begin{array}{lllll}
\exists \mathit{u},\mathit{v} \, (\mathsf{emp\_array_{2}}(\mathsf{z},\mathsf{x},\mathit{v}) \land  \mathsf{emp\_array_{3}}(\mathit{u},\mathsf{x},\mathsf{y})).
\end{array}
\]

\noindent Input: $\pplmacro{extw\_o3_{35}}.$\\
\noindent Result of interpolation:
\[\begin{array}{lllll}
\exists \mathit{u},\mathit{v} \, (\mathsf{emp\_array_{1}}(\mathsf{z},\mathsf{x},\mathit{v}) \land  \mathsf{emp\_array_{3}}(\mathit{u},\mathsf{x},\mathsf{y})).
\end{array}
\]
%
% Doc at position 10817
%
\bibliographystyle{alpha}
\bibliography{bibscratch03}
\printindex
\end{document}
