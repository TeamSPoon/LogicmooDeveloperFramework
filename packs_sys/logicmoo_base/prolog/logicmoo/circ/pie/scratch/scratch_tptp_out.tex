\documentclass[a4paper]{article}
\usepackage[utf8]{inputenc}
\usepackage[T1]{fontenc}
\usepackage{imakeidx}
\usepackage[hidelinks]{hyperref}
%% A screen friendly geometry:	    
\usepackage[paper=a5paper,scale=0.9]{geometry}
%% PPL Setup
	   
\newcommand{\assign}{\mathrel{\mathop:}=}
\newcommand{\concat}{\mathrel{+\!+}}
	    
\newcommand{\f}[1]{\mathsf{#1}}
\newcommand{\true}{\top}
\newcommand{\false}{\bot}
\newcommand{\imp}{\rightarrow}
\newcommand{\revimp}{\leftarrow}
\newcommand{\equi}{\leftrightarrow}
\newcommand{\entails}{\models}	    
\newcommand{\eqdef}{\; 
\raisebox{-0.1ex}[0mm]{$ \stackrel{\raisebox{-0.2ex}{\tiny 
\textnormal{def}}}{=} $}\; }
\newcommand{\iffdef}{\n{iff}_{\mbox{\scriptsize \textnormal{def}}}}

\newcommand{\pplmacro}[1]{\mathit{#1}}
\newcommand{\ppldefmacro}[1]{\mathit{#1}}
\newcommand{\pplparam}[1]{\mathit{#1}}
\newcommand{\pplparamidx}[2]{\mathit{#1}_{#2}}
\newcommand{\pplparamplain}[1]{#1}
\newcommand{\pplparamplainidx}[2]{#1_{#2}}
\newcommand{\pplparamsup}[2]{\mathit{#1}^{#2}}
\newcommand{\pplparamsupidx}[3]{\mathit{#1}^{#2}_{#3}}
\newcommand{\pplparamplainsup}[2]{#1^{#2}}
\newcommand{\pplparamplainsupidx}[3]{#1^{#2}_{#3}}
\newcommand{\pplparamnum}[1]{\mathit{X}_{#1}}

%%	    
%% We use @startsection just to obtain reduced vertical spacing above
%% macro headers which are immediately after other headers, e.g. of sections
%%	    
\makeatletter%
\newcounter{entry}%
\newcommand{\entrymark}[1]{}%
\newcommand\entryhead{%
\@startsection{entry}{10}{\z@}{12pt plus 2pt minus 2pt}{0pt}{}}%
\makeatother
	    
\newcommand{\pplkbBefore}
{\entryhead*{}%
\setlength{\arraycolsep}{0pt}%
\pagebreak[0]%
\begin{samepage}%
\noindent%
\rule[0.5pt]{\textwidth}{2pt}\\%
\noindent}

% \newcommand{\pplkbDefType}[1]{\hspace{\fill}{{[}#1{]}\\}}

\newcommand{\pplkbBetween}
{\setlength{\arraycolsep}{3pt}%
\\\rule[3pt]{\textwidth}{1pt}%
\par\nopagebreak\noindent Defined as\begin{center}}

\newcommand{\pplkbAfter}{\end{center}\end{samepage}\noindent}

\newcommand{\pplkbBodyBefore}{\par\noindent where\begin{center}}
\newcommand{\pplkbBodyAfter}{\end{center}}

\newcommand{\pplkbFreePredicates}[1]{\f{free\_predicates}(#1)}
% \newcommand{\pplkbRenameFreeOccurrences}[3]{\f{rename\_free\_occurrences}(#1,#2,#3)}

\newcommand{\pplIsValid}[1]{\noindent This formula is valid: $#1$\par}
\newcommand{\pplIsNotValid}[1]{\noindent This formula is not valid: $#1$\par}	    
\newcommand{\pplFailedToValidate}[1]{\noindent Failed to validate this formula: $#1$\par}

\newcounter{def}
	    
\makeindex

\begin{document}
%
% Doc at position 0
%
\title{Problems and Axiom Sets from the TPTP}
\date{Revision: May 10, 2016; Rendered: \today}
\maketitle

\noindent Macros that make problems and axiom sets from the TPTP or in TPTP
FOF or CNF format conveniently available as subformulas.  Formalized with the
\href{http://cs.christophwernhard.com/pie/}{\textit{PIE}} system.
%
% Statement at position 408
%
\pplkbBefore
\index{tptp(ProblemSpec)@$\ppldefmacro{tptp}(\pplparam{ProblemSpec})$}$\begin{array}{lllll}
\ppldefmacro{tptp}(\pplparam{ProblemSpec})
\end{array}
$\pplkbBetween
$\begin{array}{lllll}
\pplparamplain{F},
\end{array}
$\pplkbAfter
\pplkbBodyBefore
$
\begin{array}{l}\mathrm{tptp\_problem(ProblemSpec,[validate=false],Format,T,A)},\\
\mathit{[\ldots unformattable\ Prolog\ code]}.

\end{array}$\pplkbBodyAfter
%
% Doc at position 677
%
$F$ represents the TPTP problem as reverse implication $\mathit{Theorem}
\leftarrow \mathit{Axioms}$. Here are some examples for using the
$\mathit{tptp}$ macro:

\begin{verbatim}
?- mac_expand(tptp('PUZ001+1')).
?- ppl_form(tptp(' PUZ001+1'), [expand=true, style=full]).
?- ppl_valid(tptp('PUZ001+1'), [prover=cm]).
?- ppl_ipol(tptp('PUZ001+1'), [style=full]).
?- ppl_ipol(tptp('PUZ001+1'), [style=full,debug=10]).
\end{verbatim}
%
% Statement at position 1116
%
\pplkbBefore
\index{tptp_axioms(AxiomsSpec)@$\ppldefmacro{tptp\_axioms}(\pplparam{AxiomsSpec})$}$\begin{array}{lllll}
\ppldefmacro{tptp\_axioms}(\pplparam{AxiomsSpec})
\end{array}
$\pplkbBetween
$\begin{array}{lllll}
\pplparamplain{F},
\end{array}
$\pplkbAfter
\pplkbBodyBefore
$
\begin{array}{l}\mathrm{tptp\_problem(axioms(AxiomsSpec),[validate=false],Format,T,A)},\\
\mathit{[\ldots unformattable\ Prolog\ code]}.

\end{array}$\pplkbBodyAfter
\printindex
\end{document}
