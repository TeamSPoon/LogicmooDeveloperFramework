\documentstyle[11pt,prospectra,openbib]{article}
%\documentstyle[11pt,prospectra,openbib,adbgerman]{article}

\textwidth 17cm
\addtolength{\baselineskip}{1.0ex}
\addtolength{\parsep}{0.5ex}
\oddsidemargin -0.4cm
\evensidemargin -0.4cm
\topmargin 0pt
\headheight 0pt
\headsep 0pt
\footheight 0pt
\footskip 40pt
\textheight 660pt

%----------------------------------------------------------------------
% Macros

% Theorems with roman font
% To declare a theorem-environment using roman font style just say
% \newtheorem{NAME}{TEXT}[COUNTER]
% \def\NAME{\Thm{NAME}{TEXT}}
% or
% \newtheorem{NAME}[OLDNAME]{TEXT}
% \def\NAME{\Thm{OLDNAME}{TEXT}}
\catcode`\@=11
\def\Thm#1#2#3{\refstepcounter
{#2}\@ifnextchar[{\@Ythm{#1}{#2}{#3}}{\@Xthm{#1}{#2}{#3}}}
\def\@Xthm#1#2#3{\@Begintheorem{#1}{#3}{\csname the#2\endcsname}\ignorespaces}
\def\@Ythm#1#2#3[#4]{\@Opargbegintheorem{#1}{#3}{\csname
       the#2\endcsname}{#4}\ignorespaces}
\def\@Begintheorem#1#2#3{\trivlist \item[\hskip \labelsep{\bf #2\ #3}]#1}
\def\@Opargbegintheorem#1#2#3#4{\trivlist
      \item[\hskip \labelsep{\bf #2\ #3\ (#4)}]#1}
\catcode`\@=12

%---------- layout ----------

\newcommand{\nl}{\hfill\\}

%\def\definitionname{Definition}
%\def\theoremname{Theorem}
%\def\propositionname{Proposition}
%\def\corollaryname{Corollary}
%\def\lemmaname{Lemma}
%\def\conjecturename{Conjecture}
%\def\remarkname{Remark}
%\def\examplename{Example}
%\def\proofname{Proof}

\def\definitionname{Definition}%
\def\theoremname{Satz}%
\def\propositionname{Proposition}%
\def\corollaryname{Korollar}%
\def\lemmaname{Lemma}%
\def\conjecturename{Vermutung}%
\def\remarkname{Bemerkung}%
\def\examplename{Beispiel}%
\def\proofname{Beweis}%

\newtheorem{definition}{}
  \def\definition{\Thm{\rm}{definition}{\definitionname}}
\newtheorem{theorem}[definition]{\theoremname}
  \def\theorem{\Thm{\it}{definition}{\theoremname}}
\newtheorem{proposition}[definition]{\propositionname}
  \def\proposition{\Thm{\it}{definition}{\propositionname}}
\newtheorem{corollary}[definition]{\corollaryname}
  \def\corollary{\Thm{\it}{definition}{\corollaryname}}
\newtheorem{lemma}[definition]{\lemmaname}
  \def\lemma{\Thm{\it}{definition}{\lemmaname}}
\newtheorem{conjecture}[definition]{\conjecturename}
  \def\conjecture{\Thm{\it}{definition}{\conjecturename}}
\newtheorem{remark}[definition]{\remarkname}
  \def\remark{\Thm{\rm}{definition}{\remarkname}}
\newtheorem{example}[definition]{\examplename}
  \def\example{\Thm{\rm}{definition}{\examplename}}
\newenvironment{proof}{\smallskip{\em \proofname:}}{\hfill\(\Box\)\\\smallskip}

%---------- internal ----------
\newcommand{\inMath}[1]{\relax\ifmmode{#1}\else${#1}$\fi}

%---------- math ----------

% decl : name * type -> declaration
\newcommand{\decl}[2]{\inMath{{#1}:{#2}}}

% fun : domain * codomain -> function type
\newcommand{\fun}[2]{\inMath{{#1}\rightarrow{#2}}}

% fundecl : function name * domain * codomain -> function declaration
\newcommand{\fundecl}[3]{\decl{#1}{\fun{#2}{#3}}}

% argpos : underscore to indicate argument positions for infix etc.
\newcommand{\argpos}{\underline{\ }}


\newcommand{\implies}{\;\Rightarrow\;}
\newcommand{\dfeq}{\;:=\;}
\newcommand{\dfiff}{\;\:\Longleftrightarrow\;}

% eset : elements -> enumerated set
\newcommand{\eset}[1]{\inMath{\{{#1}\}}}

% pset : variable * predicate -> set
\newcommand{\pset}[2]{\inMath{\{{#1}|{#2}\}}}

% family : element of family * variable in set -> family
\newcommand{\family}[2]{\inMath{({#1})_{#2}}}

% union of sets
\newcommand{\union}[2]{\inMath{{#1}\,\cup\,{#2}}}

% intersection of sets
\newcommand{\intersection}[2]{\inMath{{#1}\,\cap\,{#2}}}

% difference of sets
\newcommand{\difference}[2]{\inMath{{#1}\backslash{#2}}}

% disjointness of sets
\newcommand{\disjoint}[2]{\inMath{\intersection{#1}{#2}=\emptyset}}


% card : set -> its cardinality
\newcommand{\card}[1]{\inMath{|{#1}|}}

\newcommand{\tuple}[1]{\inMath{\langle{#1}\rangle}}
% pair : sth * sth -> pair
\newcommand{\pair}[2]{\tuple{{#1},{#2}}}
% triple : sth * sth * sth -> triple
\newcommand{\triple}[3]{\tuple{{#1},{#2},{#3}}}

% index * from * to -> from =< index =< to
\newcommand{\inrange}[3]{\inMath{{#1}={#2}\ldots{#3}}}

%---------- categories ----------

% Obj : collection of objects in a category
% Mor : collection of morphisms in a category
\newcommand{\Obj}{{\rm Obj}}
\newcommand{\Mor}{{\rm Mor}}

% ObjOf : category -> collection of objects in category
% MorOf : category -> collection of morphisms in category
\newcommand{\ObjOf}[1]{{\Obj\;{#1}}}
\newcommand{\MorOf}[1]{{\Mor\;{#1}}}

% dom : morphism -> its domain
% cod : morphism -> its codomain
% id : object -> identity morphism
% comp : morphism * morphism -> composition of morphisms
\newcommand{\id}[1]{\inMath{\dot{#1}}}
\newcommand{\dom}[1]{\inMath{{#1}^{-}}}
\newcommand{\cod}[1]{\inMath{{#1}^{+}}}
\newcommand{\comp}[2]{\inMath{{#1};{#2}}}

\newcommand{\Cat}{\mbox{\rm CAT}}
\newcommand{\Set}{\mbox{\rm SET}}
\newcommand{\dual}[1]{\inMath{{#1}^{\rm op}}}

%---------- signatures ----------

\newcommand{\emptysig}{\inMath{\emptyset}}

\newcommand{\opdecl}[3]{\fundecl{#1}{#2}{#3}}
\newcommand{\preddecl}[2]{\decl{#1}{#2}}
\newcommand{\vardecl}[2]{\decl{#1}{#2}}
\newcommand{\sigtermdecl}[3]{\inMath{{#2}:_{#1}{#3}}}
\newcommand{\termdecl}[2]{\sigtermdecl{}{#1}{#2}}

\newcommand{\Sign}{\inMath{\rm SIGN}}
\newcommand{\EqSign}{\inMath{\rm EQSIGN}}

\newcommand{\Spec}{\inMath{\rm SPEC}}

\newcommand{\sort}[1]{{\it {#1}}}
\newcommand{\pred}[1]{{\it {#1}}}

\newcommand{\varsOf}[1]{{\it var({#1})}}

%---------- structures ----------

\newcommand{\AlgOf}[1]{\inMath{\rm {#1}\mbox{-}ALG}}
\newcommand{\StructOf}[1]{\inMath{\rm {#1}\mbox{-}STRUCT}}

\newcommand{\reduct}[1]{\inMath{\bar{#1}}}

\newcommand{\T}[1]{\inMath{T_{#1}}}
\newcommand{\TSig}{\T{\Sigma}}
\newcommand{\TSigX}{\T{\Sigma(X)}}
\newcommand{\TSigY}{\T{\Sigma(Y)}}

\newcommand{\Atoms}[1]{\inMath{At_{#1}}}

\newcommand{\HerbStruct}[2]{\inMath{HS({#1},{#2})}}

\newcommand{\evalAss}[1]{\inMath{{#1}^{\#}}}
\newcommand{\evalGnd}[1]{\inMath{\#_{#1}}}

%---------- specifications ----------

\newcommand{\eq}{\approx}
\newcommand{\eqS}{\dot{\approx}}
\newcommand{\clause}[2]{\inMath{{#1}\rightarrow{#2}}}
\newcommand{\vclause}[3]{\inMath{\forall{#1}.{#2}\rightarrow{#3}}}
\newcommand{\vuclause}[2]{\inMath{\forall{#1}.{#2}}}

%---------- models ----------

\newcommand{\InitOf}[1]{\inMath{{\cal I}({#1})}}

%---------- theories ----------

\newcommand{\ThOf}[1]{\inMath{{\rm Th}({#1})}}
\newcommand{\IThOf}[1]{\inMath{{\rm ITh}({#1})}}

\newcommand{\entails}{\inMath{\;\models\;}}
\newcommand{\indEntails}{\inMath{\;\models_I\;}}

%---------- terms ----------

\newcommand{\subterm}[2]{\inMath{{#1}/{#2}}}

%---------- rewriting ----------

\newcommand{\rew}{\rightarrow}
\newcommand{\rewR}{\stackrel{\epsilon}{\rightarrow}}
\newcommand{\rewT}{\stackrel{+}{\rightarrow}}
\newcommand{\rewRT}{\stackrel{*}{\rightarrow}}
\newcommand{\rewI}{\leftarrow}
\newcommand{\rewIR}{\stackrel{\epsilon}{\leftarrow}}
\newcommand{\rewIT}{\stackrel{+}{\leftarrow}}
\newcommand{\rewIRT}{\stackrel{*}{\leftarrow}}
\newcommand{\rewS}{\leftrightarrow}
\newcommand{\rewSR}{\stackrel{\epsilon}{\leftrightarrow}}
\newcommand{\rewST}{\stackrel{+}{\leftrightarrow}}
\newcommand{\rewSRT}{\stackrel{*}{\leftrightarrow}}

\newcommand{\rewG}{\diamondsuit}
\newcommand{\rewGRT}{\stackrel{*}{\diamondsuit}}

\newcommand{\rewD}{\downarrow}

%---------- proof ----------

\newcommand{\ProofsOf}[1]{{\rm Proofs}({#1})}
\newcommand{\trans}{\inMath{\vdash}}

\newcommand{\Proofs}{\inMath{{\cal P}}}
\newcommand{\iProofs}{\inMath{\Proofs_i}}
\newcommand{\diProofs}{\inMath{\Proofs_d}}

\newcommand{\proofGT}{\succ^P}
\newcommand{\proofGE}{\succeq^P}
\newcommand{\proofLT}{\prec^P}
\newcommand{\proofLE}{\preceq^P}

\newcommand{\cplGT}{\succ^c}
\newcommand{\cplGE}{\succeq^c}
\newcommand{\cplLT}{\prec^c}
\newcommand{\cplLE}{\preceq^c}

%---------- type predicates ----------

\newcommand{\tp}[1]{\inMath{\mbox{\it is-}{#1}}}
\newcommand{\tc}[1]{\inMath{{\it tc}({#1})}}
\newcommand{\TC}[1]{\inMath{{\it TC}({#1})}}
\newcommand{\TS}{\inMath{\it TS}}
\newcommand{\TF}[1]{\inMath{{\TS_{#1}}}}

% end of macros
%----------------------------------------------------------------------


\begin{document}

\title{
{\bf CEC}\\
A System to Support Modular Order-Sorted Specifications
with Conditional Equations\\
User Manual (Version 1.5)}
\author{
Hubert Bertling\\
Harald Ganzinger\\
Renate Sch\"afers}

\prospectrano{M.1.3--R--18.0}
\prospectradist{public}

\maketitle

\begin{abstract}
{\bf CEC} is a rewrite rule laboratory for order-sorted specifications
with conditional equations. The major module of CEC is a powerful
completion procedure for conditional equations. This manual describes its 
use assuming the user to be familiar with basic notions in
conditional term rewriting.
\end{abstract}

\setcounter{page}{1}
\pagenumbering{roman}
\tableofcontents

\newpage

\setcounter{page}{1}
\pagenumbering{arabic}


\section{Introduction}

This manual describes the use of CEC,
a system for conditional equational completion. We assume familiarity
with well known notions in the term rewriting area e.g.
{\it
signature, term over a signature,
(conditional) term rewrite rule, applicability of a rewrite rule, 
\ldots}, cf. e.g. \cite{HO80}, \cite{Kap84}. The concepts are discussed only as far as
necessary to allow a meaningful interaction with the system. Theoretical
foundations of the concepts implemented in CEC are not discussed. 
Hints to the literature will be given whenever such concepts are introduced.

CEC can be used under Quintus-Prolog2.x.
An installation guide for CEC is to be found in the appendix of this manual.


There are some CEC-specific notions used throughout this manual. The input to 
CEC is called a 
{\it specification}. Also any later state of the initial
specification is called a specification. To distinguish the specification
which will be completed when invoking the completion procedure from any other
specification CEC presently knows about, this specification is called the 
{\it current specification}.

CEC includes a help-function. Type 
``\user{??}\nt{space}\user{.}'' to obtain a 
list of all available CEC-commands and type
``\user{?}\nt{FunctionName}\user{.}''
or ``\user{?(}\nt{FunctionName}\user{).}'' to 
get a short description for
CEC-command with the name {\it FunctionName}.
   % --.4.89
\section{An Example Session}
\label{exampleSession}

To provide a first impression of  the capabilities of the CEC-System, we
describe the development of a quicksort algorithm on lists of
natural numbers.

\subsection{The Specification of Quicksort}

Specifications can be written and completed in a {\em modular}
way. CEC can combine completed specifications without 
repeating previous work. The hierarchy of modules for the
specification of quicksort on lists of natural numbers is given
in the following diagram:

\begin{picture}(450,100)
\put(50,0){\makebox(400,100){}}
\put(205,85){\makebox(0,0){{\tt qsortnat}}}
\put(200,80){\vector(-2,-1){45}}
\put(200,80){\vector(2,-1){45}}

\put(150,50){\makebox(0,0){{\tt nat}}}

\put(257,50){\makebox(0,0){{\tt qsort}}}
\put(255,45){\vector(-3,-2){30}}
\put(255,45){\vector(3,-2){30}}

\put(215,20){\makebox(0,0){{\tt totalOrder}}}

\put(300,20){\makebox(0,0){{\tt lists}}}
\end{picture}

\noindent
The module \cec{qsort} --- saved in a file named \cec{qsort.eqn} ---
describes the quicksort algorithm:
\begin{spec}
module qsort using lists + totalOrder.

op sort  : (list -> list).
op split : (elem * list -> pair).
cons (',') : (list * list -> pair).

sort([]) = [].
split(x, l) = (l1 , l2) => sort([x|l]) = append(sort(l1), [x|sort(l2)]).

split(x, []) = ([] , []).
(y =< x) = true and split(x, l) = (l1,l2) => split(x, [y|l]) = ([y|l1],l2).
(y =< x) = false and split(x, l) = (l1,l2) => split(x, [y|l]) = (l1,[y|l2]).
\end{spec}

Complex specifications can be constructed from modules by the 
elementary operations {\em combination}, {\em enrichment} and {\em renaming}
(\refArrow chapter \ref{OperationsOnSpecifications}).
The combine operator (\cec{+}) forms the union of two specifications and
the rename operator (\cec{<-}) allows to renaming sorts and operators.
Any specification in CEC is the enrichment of a possibly empty base
specification (``{\tt using} $<$base$>$'') by new vocabulary and axioms.
The module \cec{qsort} is based on \cec{lists} 
and \cec{totalOrder}. Here \cec{lists} is the imported module
\begin{spec}
module lists.

cons []   : list .
cons '.'  : (elem * list -> list).
op append : (list * list -> list).

append([], l) = l.
append([e | l1], l2) = [e | append(l1, l2)].
\end{spec}

We use the constructor \cec{[]} to denote the empty list and 
the constructor `\cec{.}' for adding an element to a list
(This is exactly the way how lists are constructed in Prolog and we can
use the usual Prolog notation for lists in which \cec{[e|l]} is
a synonym for \cec{(e '.' l)}). 
Additionally we have defined an 
operator \cec{append} for the concatenation of two lists and described 
its behaviour through two equations.

\cec{totalOrder} plays the r$\hat{o}$le of a formal parameter
of \cec{qsort}:
\begin{spec}
module totalOrder.

op (=<) : (elem * elem -> bool) .

(x =< x) = true .
(x =< y) = false => (y =< x) = true .
(x =< y) = true and (y =< z) = true => (x =< z) = true .
(x =< y) = true and (y =< x) = true => x = y .
\end{spec}

It describes a usable approximation of the usual first-order axioms
for total orders in a Horn clause setting.
An actual parameter candidate for \cec{totalOrder} is
the following specification of natural numbers:
\begin{spec}
module nat using totalOrder(elem <- nat).

cons 0 : nat.
cons(s, 100, fy) :  (nat -> nat).

(0 =< n ) = true.
(s n =< 0) = false .
(s n =< s m) = (n =< m).
\end{spec}

Computing with the above specification one has to write the numbers
$n$ as \cec{s}$^n$\cec{0} which is a tedious work. In CEC it is possible to provide different external
representations for the terms of a specification by specifying
the translations from the external to the internal
and from the internal to the external representation 
(\refArrow chapter \ref{ParseAndPretty}).

The instantiation of the formal parameter of \cec{qsort} with
nat yields \cec{qsortnat}:
\begin{spec}
module qsortnat using qsort(elem <- nat) + nat.
\end{spec}

The modules \cec{qsort} and \cec{nat} are combined after
renaming the sort \cec{elem} of \cec{qsort} into \cec{nat}.
The completion of \cec{qsortnat} will check the consistency of the axioms 
for \cec{=<} in the actual parameter \cec{nat} and in the formal
parameter \cec{totalOrder}. The latter proves the correctness
of the actual parameter in cases where the actual parameter is
constructor-complete. \cec{nat} is constructor-complete.

\subsection{Order Specifications}

In CEC the following termination orderings are available:
\begin{itemize}
\item
Two  {\em precedence orderings} \kw{kns} and \kw{neqkns}, according to 
Kapur et. al \cite{KNS85} (\refArrow chapter \ref{kns}).
%They are based on the recursive comparison of paths occurring
%in the terms to be compared. This order is induced by a partial
%order on operators called the {\it precedence}. The precedence ordering
%\kw{neqkns} forbids that two different operators have the same precedence.
%All constructors are given a precedence less than any nonconstructor operator.
\item
The method of {\it polynomial interpretations}
where \[t_1 > t_2 : \iff I(t_1) > I(t_2). \]
if $I(t)$ is the polynomial or tuple of polynomials associated with
it.
The concrete version of this technique as it is used in CEC is due to 
\cite{CL86} (\refArrow chapter \ref{poly}).
\kw{poly}\nt{N} stands for polynomial interpretations with 
tuplelength $N$.
\end{itemize}
The precedence declarations or polynomial interpretations
for the operators of a specification are given interactively
during the completion process or in a {\em order specification}
associated with the specification
(\refArrow chapter \ref{OrderSpecification}).
%The order specification determines the termination ordering to be used for 
%\nt{specificationName}, the order names for its direct imports
%and gives precedence declarations for the operators of the
%specification (if the termination ordering is \kw{kns} or \kw{neqkns}) or
%polynomial interpretations (if the termination ordering is \kw{poly}\nt{N}).

For example, the order specification for the module \cec{lists}
may have the following form:

\begin{spec}
order poly3 for lists.

setInterpretation(['[]'        : [2, 2, 2],
                   '.'(x,y)    : [3 * x + y + 1, x + y, x + y],
                   append(x,y) : [x + y, x + y, 2 * x + y]]).
\end{spec}

In this case, polynomial interpretations with tuple length 3
have been chosen. If this information is stored in the file \cec{lists.pol.ord},
\cec{pol} is called the {\em order name} of this order specification.

\subsection{Reading and Displaying}

We want to demonstrate how CEC handles this specification. 
First, we read in the specification of \cec{lists} using
the \comRef{in}-command (\refArrow chapter \ref{InCommand})
with two arguments, the name of the module and the order name
(\refArrow chapter \ref{orderBase}):

\begin{screen}
| ?- in(lists,pol).
[collecting garbage...]
[evaluating base of lists.eqn with lists.pol.ord...]
 [thawing standard.poly3.q2.0 into user...]
 [storing to standard.poly3]
[reading body of lists.eqn and lists.pol.ord...]
[analyzing axioms...]
[collecting garbage...]

Time used: 3.634 sec.
\end{screen}

\noindent
We can use the \comRef{sig}-command to display the signature:

\begin{screen}
| ?- sig.

Signature :

cons true       : bool.
cons false      : bool.
cons []	: list.
cons .  : (elem * list -> list).
op append       : (list * list -> list).
\end{screen}

The boolean constants \cec{true} and \cec{false} are imported
from the \cec{standard}-module, which is automatically imported into 
every specification. The fact that \cec{true} and \cec{false}
are constructors implies that any equation, explicitly given in
a specification or derived through completion, equating these
constants is forbidden.
As expected, we also see our two constructors \cec{[]} and
`\cec{.}' and the operator \cec{append}.

The \comRef{show}-command can be used to display the 
set of equations and rules (\refArrow chapter \ref{Displaying}):

\begin{screen}
| ?- show.

Current equations

  1    append([],l) = l
  2    append([e|l1],l2) = [e|append(l1,l2)]

Current rules

Current nonoperational equations

All axioms reduced.
All superpositions computed.
The set of equations is not empty.
\end{screen}

\noindent
The information about the polynomial interpretations can be made
visible using the \comRef{interpretation}-command. If \cec{kns}
or \cec{neqkns} is used, this can be achieved by the
\comRef{operators}-command (\refArrow chapter \ref{InspecTermination}).

Every consulted specification is saved in a special {\em specification
variable}. The user can restore an older state of his system
using the \comRef{load}-command (\refArrow chapter \ref{LoadCommand})
or can update the content of the variable using the \comRef{store}-command
(\refArrow chapter \ref{StoreCommand}). 
Whenever a specification is to be imported, the CEC system uses the content
of the corresponding specification variable, if this specification variable 
exists. So it increases one's speed, if the completion process of a hierarchical
specification follows the tree structure from its leaves to its root, 
storing every participating specification after successful completion.
We assume that attention is paid to this advice for our example.
Information about the current specification variables
can be retrieved using the \comRef{specifications}-command.

It is also possible to save a system externally using the 
\comRef{freeze}-command (\refArrow chapter \ref{FreezeCommand}).
To restore a externally saved system use the \comRef{thaw}-command
(\refArrow chapter \ref{ThawCommand}).

\subsection{Completion}
We now apply the completion procedure (\refArrow chapter
\ref{Completion}) to our example.

\subsubsection{Completing {\tt lists}}

The completion process --- started with the \comRef{c}-command ---
is able to orient the two equations of \cec{lists}
without requesting any additional information:

\begin{screen}
| ?- c.

new rule   1    append([],l) = l .

new rule   2    append([e|l1],l2) = [e|append(l1,l2)] .

[4 superpositions yet to be considered.]

0 superpositions have been computed.
Time used: 0.883011 sec.
\end{screen}

\noindent
The system is complete, and has the following rules, displayed 
by the \comRef{show}-command:

\begin{screen}
| ?- show.

Current equations

Current rules

  1    append([],l) = l
  2    append([e|l1],l2) = [e|append(l1,l2)]

Current nonoperational equations

All axioms reduced.
All superpositions computed.
No more equations, the system is complete.
\end{screen}


\subsubsection{Completing {\tt totalOrder}}

An important feature is that CEC does not require to transform all
equations into rewrite rules. This can be demonstrated during the
completion of \cec{totalOrder}.

After orienting the first equation into a rewrite rule, 
the system discovers that it is unable to orient the second 
equation:

\begin{screen}
Checking reductivity constraints for rule
        x=<y = false => y=<x = true:
The current ordering fails to prove
[y=<x]  >  [x=<y,false].
At this point you may take any of the the following actions:
a. for assume to be proved
c. for checking quasi-reductivity of the equation
p. for postpone
n. for considering the equation as nonoperational
   Please answer with a. or c. or p. or n. (Type A. to abort) > \user{n.}
\end{screen}

\noindent
We want to consider this equation as nonoperational
(\refArrow chapter \ref{Completion}), so we answer with \user{n.}.

Nonoperational equations become useless for the equational theory and, hence,
in fact nonoperational, if they are superposed on at least one of their
conditions by all rewrite rules. This yields new conditional equations. 
What concerns the equational theory, the new equations together have 
the same power as the original equation. 
However, they may have better operational properties than the
equation they have been generated from.
Equations that cannot be oriented into a reductive rule must
either be eliminated eventually or considered nonoperational.

Checking the convergence of conditional equations requires to compare different
applications of equations and rewrite rules. The comparison of equation application is performed
by comparing the literals of the equation instance. The {\em status} of an equation 
determines the order in which the literals of the equation should be
inspected. The status $ms$ means that the literals are compared as a multiset.
Instead of $ms$ the user can choose an arbitrary sequence of the literals
by entering a permutation of [0 .. n] where n is the number of conditions
(\refArrow \cite{Gan88a}). 

We want to use $ms$ here:

\begin{screen}
In which order should the literals of the equation be
inspected when comparing proofs that use this equation?
Please enter ms (for multiset ordering) or a permutation of [0 .. 1]
(0 stands for the consequent, i>0 for the ith condition). > \user{ms.}
\end{screen}

\noindent
The other two nonreductive equations are handled in the same way and we get the 
following complete system:

\begin{screen}
| ?- \user{show.}

Current equations

Current rules

  1    x=<x = true

Current nonoperational equations

  1    x=<y = false => y=<x = true
  2    x=<y = true and y=<x = true => x = y
  3    x=<y = true and y=<z = true => x=<z = true

All axioms reduced.
All superpositions computed.
No more equations, the system is complete.
\end{screen}

In this case, superposing the only rule \cec{x =< x = true} on the first
condition of the nonoperational equations does not generate any nontrivial
(nonconvergent) consequences.

\subsubsection{Completing {\tt qsort}}

The next step is the completion of the \cec{qsort} specification.
The completion procedure is able to orient the first and third 
equation of our specification, but fails to orient the second one.

\begin{screen}
| ?- c.

new rule   9    sort([]) = [] .

new rule  10    split(x,[]) = [],[] .
The equation
        split(x,l) = l1,l2 => sort([x|l]) = append(sort(l1),[x|sort(l2)])
is not reductive.
\end{screen}

\noindent
This is due to the presence of the {\em extra variables} \cec{l1} and \cec{l2}
in the condition of the equation.

Equations with extra variables in the condition or in the right side of
the consequence are usually not admitted as rewrite rules. 
Fortunately some of these equations belong to the class of what we call
{\em quasi-reductive rules} (\refArrow chapter \ref{Completion}). 
CEC is able to prove the remaining equations 2, 4
and 5 of the module \cec{qsort} to be quasi-reductive.
Quasi-reductive rules are a generalization
of reductive conditional rewrite rules and the associated rewrite
process is similarly efficient.
It specifies e.g. for the second equation the replacement
of \cec{sort([x|l])} by the term \cec{append(sort(l1), [x|sort(l2)])} 
if the normalform of \cec{split(x, l)} matches \cec{(l1,l2)}.
After this match the extra variables \cec{l1} and \cec{l2} are instantiated to
terms which include only variables of the left hand side.
In quasi-reductive rules, conditions are oriented, too. To solve an instance
of a condition equation means to rewrite its left side into an instance of the
right side.

\begin{screen}
Do you want a check for quasi-reductivity?
c. for check
n. for considering the equation as nonoperational
p. for postpone
   Please answer with c. or n. or p. (Type A. to abort) >\user{c.}
\end{screen}

\noindent
To orient an equation into a quasi-reductive rule, we must first indicate the 
desired orientation of the equations in the condition and the conclusion. 
In this example, the equation in the condition
and the conclusion should be oriented from left to right (literal annotation
``\user{l}'').

\begin{screen}
Enter annotations of literals in
	split(x,l) = l1,l2 => sort([x|l]) = append(sort(l1),[x|sort(l2)])
	: \user{[l,l].}
\end{screen}

\noindent
Now CEC attemps to prove the quasi-reductivity of the equation according to the
definition in chapter \ref{Completion}. This involves giving  polynomial
interpretations to some auxiliary operators.
Here we must give an appropriate interpretation for \cec{$h5_0}:

\begin{screen}
Checking reductivity constraint:
Consider the terms
        $h5_0((l1,l2),x)
and
        append(sort(l1),[x|sort(l2)])

There is no interpretation of operator '$h5_0' with arity 2
The default interpretation is :
   [ 2 * x * y ,
     2 * x * y ,
     2 * x * y ]
Do you want to change it ? (if so type 'y') \user{y}
Do you want to change it component for component ? (if so type 'y') \user{n}
Type in the new interpretation tuple
[ $h5_0 ] (x,y) = \user{[2*x+3*y+1,2*x+2*y,2*x+2*y].}
Resulting interpretation for Operator '$h5_0' with arity 2 :
   [ 2 * x + 3 * y + 1 ,
     2 * x + 2 * y ,
     2 * x + 2 * y ]
Do you accept it ? (if not, type 'n') \user{y}
\end{screen}

\noindent
Now the proof of the quasi-reductivity of the equation is completed:

\begin{screen}
new rule  11    split(x,l) = l1,l2 => sort([x|l]) = 
                                      append(sort(l1),[x|sort(l2)]) .
\end{screen}

\noindent
In the same way the equations

\cec{y=<x = true and split(x,l) = (l1,l2) => split(x,[y|l]) = ([y|l1],l2)}

\noindent
and 

\cec{y=<x = false and split(x,l) = (l1,l2) => split(x,[y|l]) = (l1,[y|l2])}

\noindent
can be oriented into quasi-reductive rules.

For the quicksort specification three nontrivial superposition instances are
computed. For any nontrivial equation with at least one condition that is
generated during completion, CEC will ask the user what to do with it. In the
example we decide to declare these consequences as ``nonoperational''.

\begin{screen}
instance   6    l1,l2 = l4,l3 and split(x1,l5) = l1,l2 => 
                                     append(sort(l1),[x1|sort(l2)]) = 
                                     append(sort(l4),[x1|sort(l3)])
of        6    split(x,l) = l1,l2 => sort([x|l]) = 
                                     append(sort(l1),[x|sort(l2)])
by superposing
          6    split(x,l) = l1,l2 => sort([x|l]) = 
                                     append(sort(l1),[x|sort(l2)]) 
on the left side.

Consider the equation
        l1,l2 = l4,l3 and split(x1,l5) = l1,l2 => 
                                     append(sort(l1),[x1|sort(l2)]) = 
                                     append(sort(l4),[x1|sort(l3)]).
The following actions may be taken:
o. for attempting to orient into a (quasi-)reductive rule
p. for postpone
n. for considering equation as nonoperational
   Please answer with o. or p. or n. (Type A. to abort) >\user{n.}
\end{screen}

As mentioned before it is sufficient to superpose all rewrite rules
on just one condition of a nonoperational conditional equation. So
CEC asks the user on which condition superposition should be
applied. 
We will choose the first equation of the condition, since we know it
will generate no nontrivial superpositions:

\begin{screen}
Which of the condition equations in
        l1,l2 = l4,l3 and split(x1,l5) = l1,l2 
should be selected for superposition?
Please enter index from 1 to 2. > \user{1.}

The equation l1,l2 = l4,l3 and split(x1,l5) = l1,l2 => 
                                          append(sort(l1),[x1|sort(l2)]) = 
                                          append(sort(l4),[x1|sort(l3)]) 
will be considered as nonoperational.
\end{screen}

\subsubsection{Completing {\tt nat}}

The completion of the \cec{nat} specification is straightforward using
the following order specification:

\begin{spec}
order poly1 for nat. 

setInterpretation([0 : 2,
                   s(x) : 8 * x]).
\end{spec}

\subsubsection{Combining {\tt qsort} and {\tt nat}}

We now want to combine the \cec{nat} specification with the 
\cec{qsort} specification.

\begin{screen}
| ?- store.
yes
| ?- load(qsort,poly3).
\end{screen}

\noindent
The \comRef{store}-command saves the completed {\tt nat} specification
into its {\em specification variable}.
The \comRef{load}-command loads the completed {\tt qsort} specification
from its specification variable.

\begin{screen}
| ?- renameSpec(elem <- nat).
yes
| ?- sig.

Signature :

cons [] : list.
cons .  : (nat * list -> list).
op append       : (list * list -> list).
cons true       : bool.
cons false      : bool.
op =<   : (nat * nat -> bool).
op sort	: (list -> list).
op split        : (nat * list -> pair).
cons ,    : (list * list -> pair).
yes
| ?- store(qsortnatSpec).
yes
\end{screen}

\noindent
Now we combine the two specifications and make the result 
to the current specification:

\begin{screen}
| ?- combineSpecs(qsortnatSpec,'nat.poly1_qsort',user).
yes
\end{screen}

\noindent
Because \cec{totalOrder} is a formal parameter for \cec{qsort},
completion now checks the consistency of the axioms for
\cec{=<} in the actual parameter \cec{nat} and in the formal
parameter \cec{totalOrder}.
The completion process will not do unnecessary work again. 
For the above example the completion process
will only compute overlaps between axioms
of the renamed module \cec{qsort} and axioms of the module \cec{nat}
but it will not recompute overlaps between axioms of one of these
modules.

If two constructor terms are shown to be equal then the specification
is inconsistent. In our case the system is completed without any user 
interaction, no inconsistency showing up.

\subsection{Computing in Completed Specifications}

Computation in specifications is realized by term reduction with the
rules of the completed specification. The result of such computations are unique
normal forms (\refArrow chapter \ref{NormCommand}):

\begin{screen}
| ?- norm(sort([5,3,6,1])).
The normalform of sort([5,3,6,1]) is [1,3,5,6] .
\end{screen}

\noindent
If confluence can be achieved and if all rules are reductive equational theorems 
become decidable, e.g. it is decidable if two terms are equivalent with respect to the
equations in the specification (\refArrow chapter \ref{ProveCommand}):

\begin{screen}
| ?- prove(sort([5,3]) = [3,5]).
Normal forms are: [3,5] and [3,5]
yes
\end{screen}

\noindent 
Conditional narrowing (\refArrow chapter \ref{NarrowCommand}) can then be used to solve 
equations.

\begin{screen}
| ?- solve(sort([1,x]) = [x,1],U).

Time used: 7.43298 sec.

U = {x-nat/0} ;

Time used: 7.91595 sec.

U = {x-nat/1} ;

no
\end{screen}

\noindent
Here we proved that the only substitutions for \cec{x} such that
\cec{sort([1,x])} is equal to \cec{[x,1]} are \cec{0} and \cec{1}.

\subsection{Saving the CEC System}

The whole state of the CEC system can be saved using the
\comRef{saveCEC}-command. 

\begin{screen}
| ?- saveCEC('qsortCEC').
[ Prolog state saved into /home/helga/cec/cec/qsortCEC ]
\end{screen}

Using \cec{qsortCEC} instead of CEC offers the possibility to have
the complete hierarchy of our \cec{qsort} example present, without
wasting time to reconsult the necessary frozen states of all the
specification used for this example.

\section{Syntax}
\label{syn}
\texttt{cplint} permits the definition of discrete probability distributions and continuous probaility
densities.
\subsection{Discrete Probability Distributions}
\label{discrete}
LPAD and CP-logic programs consist of a set of annotated disjunctive clauses.
Disjunction in the head is represented with a semicolon and atoms in the head are separated from probabilities by a colon. For the rest, the usual syntax of Prolog is used.
A general CP-logic clause has the form
\begin{verbatim}
h1:p1 ; ... ; hn:pn :- Body.
\end{verbatim}
where \verb|Body| is a conjunction of goals as in Prolog.
 No parentheses are necessary. The \texttt{pi} are numeric expressions. It is up to the user to ensure that the numeric expressions are legal, i.e. that they sum up to less than one.

If the clause has an empty body, it can be represented like this
\begin{verbatim}
h1:p1 ; ... ; hn:pn.
\end{verbatim}
If the clause has a single head with probability 1, the annotation can be omitted and the clause takes the form of a normal prolog clause, i.e. 
\begin{verbatim}
h1 :- Body.
\end{verbatim}
stands for 
\begin{verbatim}
h1:1 :- Body.
\end{verbatim}
The coin example of  \cite{VenVer04-ICLP04-IC} is represented as (file \href{http://cplint.eu/example/inference/coin.pl}{\texttt{coin.pl}})
\begin{verbatim}
heads(Coin):1/2 ; tails(Coin):1/2 :- 
  toss(Coin),\+biased(Coin).

heads(Coin):0.6 ; tails(Coin):0.4 :- 
  toss(Coin),biased(Coin).

fair(Coin):0.9 ; biased(Coin):0.1.

toss(coin).
\end{verbatim}
The first clause states that if we toss a coin that is not biased it has equal probability of landing heads and tails. The second states that if the coin is biased it has a slightly higher probability of landing heads. The third states that the coin is fair with probability 0.9 and biased with probability 0.1 and the last clause states that we toss a coin with certainty.

Moreover, the bodies of rules may contain built-in predicates, predicates
from the libraries \verb|lists|, \verb|apply| and \verb|clpr/nf_r|
plus the predicate
\begin{verbatim}
average/2
\end{verbatim}
that, given a list of numbers, computes its arithmetic mean.

The body of rules may also contain the predicate \verb|prob/2| that computes the
probability of an atom, thus allowing nested probability computations.
For example (\href{http://cplint.eu/example/inference/meta.pl}{\texttt{meta.pl}})
\begin{verbatim}
a:0.2:-
  prob(b,P),
  P>0.2.
\end{verbatim}
is a valid rule.

Moreover, the probabilistic annotations can be variables, as in 
(\href{http://cplint.eu/example/inference/flexprob.pl}{\texttt{flexprob.pl}}))
\begin{verbatim}
red(Prob):Prob.

draw_red(R, G):-
  Prob is R/(R + G),
  red(Prob).
\end{verbatim}
Variables in probabilistic annotations must be ground when resolution reaches the end of the body, 
otherwise an exception is raised.

Alternative ways of specifying probability distribution include
\begin{verbatim}
A:discrete(Var,D):-Body.
\end{verbatim}
or
\begin{verbatim}
A:finite(Var,D):-Body.
\end{verbatim}
where \verb|A| is an atom containg variable \verb|Var| and \verb|D|
is a list of couples \verb|Value:Prob| assigning probability \verb|Prob|
to \verb|Value|. 
Moreover, you can use
\begin{verbatim}
A:uniform(Var,D):-Body.
\end{verbatim}
where \verb|A| is an atom containg variable \verb|Var| and \verb|D|
is a list of values each taking the same probability (1 over the length
of \verb|D|).
\subsubsection{ProbLog Syntax}
You can also use ProbLog \cite{DBLP:conf/ijcai/RaedtKT07} syntax, so a general clause takes the form
\begin{verbatim}
p1::h1 ; ... ; pn::hn :- Body
\end{verbatim}
where the \texttt{pi} are numeric expressions. 

\subsubsection{PRISM Syntax}
You can also use PRISM \cite{DBLP:conf/ijcai/SatoK97} syntax, so a program is composed of
a set of regular Prolog rules whose body may contain calls to the \verb|msw/2| predicate (multi-ary 
switch). A call \verb|msw(term,value)| means that a random variable associated to \verb|term|
assumes value \verb|value|. The admissible values  for a discrete random variable are 
specified using facts for the \verb|values/2| predicate of the form
\begin{verbatim}
values(T,L).
\end{verbatim}
where \verb|T| is a term (possibly containing variables) and \verb|L| is a list of values.
The distribution over values is specified using directives for \verb|set_sw/2| of the form
\begin{verbatim}
:- set_sw(T,LP).
\end{verbatim}
where \verb|T| is a term (possibly containing variables) and \verb|LP| is a list of
probability values.
Remember that in PRISM each call to \verb|msw/2| refers to a different random
variable, i.e., no memoing is performed, differently from the case of LPAD/CP-Logic/ProbLog.

For example, the coin example above in PRISM syntax becomes
(\href{http://cplint.eu/example/inference/coinmsw.pl}{\texttt{coinmsw.pl}})
\begin{verbatim}
values(throw(_),[heads,tails]).
:- set_sw(throw(fair),[0.5,0.5]).
:- set_sw(throw(biased),[0.6,0.4]).
values(fairness,[fair,biased]).
:- set_sw(fairness,[0.9,0.1]).
res(Coin,R):- toss(Coin),fairness(Coin,Fairness),msw(throw(Fairness),R).
fairness(_Coin,Fairness) :- msw(fairness,Fairness).
toss(coin).
\end{verbatim}
\subsection{Continuous Probability Densities}
\label{cont}

\verb|cplint| handles continuous random variables as well with its
sampling inference module.
To specify a probability density on an argument \verb|Var| of an atom
\verb|A| you can used rules of the form
\begin{verbatim}
A:Density:- Body
\end{verbatim}
where \verb|Density| is a special atom identifying a probability  density on variable \verb|Var| and \verb|Body| (optional) is a regular clause body.
Allowed \verb|Density| atoms are
\begin{itemize}
\item \verb|uniform(Var,L,U)|: \verb|Var| is uniformly distributed in $[L,U]$
\item \verb|gaussian(Var,Mean,Variance)|: \verb|Var| follows a Gaussian distribution with mean \verb|Mean| and variance \verb|Variance|. The distribution can be multivariate if \verb|Mean| 
is a list and \verb|Variance| a list of lists representing the mean vector and the covariance matrix. In this case the values of \verb|Var| are lists of real values with the same length as
that of \verb|Mean|
\item \verb|dirichlet(Var,Par)|: \verb|Var| is a list of real
numbers following a Dirichlet distribution with $\alpha$ parameters specified
by the list \verb|Par|
\item \verb|gamma(Var,Shape,Scale)|  \verb|Var| follows a gamma distribution 
with shape parameter \verb|Shape| and scale parameter \verb|Scale|.
\item \verb|beta(Var,Alpha,Beta)|  \verb|Var| follows a beta distribution 
with parameters \verb|Alpha| and \verb|Beta|.
\item \verb|poisson(Var,Lambda)|  \verb|Var| follows a Poisson distribution 
with parameter \verb|Lambda|.
\item \verb|binomial(Var,N,P)|  \verb|Var| follows a binomial distribution 
with parameters \verb|N| and \verb|P|.
\item \verb|geometric(Var,P)|  \verb|Var| follows a geometric distribution 
with parameter \verb|P|.
\end{itemize}
For example
\begin{verbatim}
g(X): gaussian(X,0, 1).
\end{verbatim}
states that argument \verb|X| of \verb|g(X)| follows a Gaussian 
distribution with mean 0 and variance 1, while
\begin{verbatim}
g(X): gaussian(X,[0,0], [[1,0],[0,1]]).
\end{verbatim}
states that argument \verb|X| of \verb|g(X)| follows a Gaussian 
multivariate distribution with mean vector $[0,0]$ and covariance matrix
$$\left[\begin{array}{rr}
1&0\\
0&1
\end{array}\right]$$.



For example, \href{http://cplint.eu/example/inference/gaussian_mixture.pl}{\texttt{gaussian\_mixture.pl}} defines a mixture of two Gaussians:
\begin{verbatim}
heads:0.6;tails:0.4.
g(X): gaussian(X,0, 1).
h(X): gaussian(X,5, 2).
mix(X) :- heads, g(X).
mix(X) :- tails, h(X).
\end{verbatim}
The argument \verb|X| of
\verb|mix(X)| follows a distribution that is a mixture of two Gaussian,
one with mean 0 and variance 1 with probability 0.6 and one with 
mean 5 and variance 2 with probability 0.4.

The parameters of the distribution atoms can be taken from the probabilistic
atom, the example \href{http://cplint.eu/example/inference/gauss_mean_est.pl}{\texttt{gauss\_mean\_est.pl}}
\begin{verbatim}
val(I,X) :-
  mean(M),
  val(I,M,X).
mean(M): gaussian(M,1.0, 5.0).
val(_,M,X): gaussian(X,M, 2.0).
\end{verbatim}
states that for an index \verb|I| the continuous variable \verb|X| is 
sampled from a Gaussian whose variance is 2 and whose mean is sampled from a Gaussian with mean 1 and
variance 5.

Any operation is allowed on continuous random variables. The example below
(\href{http://cplint.eu/example/inference/kalman_filter.pl}{\texttt{kalman\_filter.pl}}) encodes a Kalman filter:
\begin{verbatim}
kf(N,O, T) :-
  init(S),
  kf_part(0, N, S,O,T).
kf_part(I, N, S,[V|RO], T) :-
  I < N,
  NextI is I+1,
  trans(S,I,NextS),
  emit(NextS,I,V),
  kf_part(NextI, N, NextS,RO, T).
kf_part(N, N, S, [],S).
trans(S,I,NextS) :-
  {NextS =:= E + S},
  trans_err(I,E).
emit(NextS,I,V) :-
  {NextS =:= V+X},
  obs_err(I,X).
init(S):gaussian(S,0,1).
trans_err(_,E):gaussian(E,0,2).
obs_err(_,E):gaussian(E,0,1).
\end{verbatim}
Continuous random variables are involved
in arithmetic expressions (in \verb|trans/3| and \verb|emit/3|). It
is often convenient, as in this case, to use CLP(R) constraints (by
including the directive \verb|:- use_module(library(clpr)).|) as 
in this way the expressions can be used in multiple directions and 
the same clauses can be used both to sample and to evaluate the weight the sample on the basis
of evidence,
otherwise different clauses have to be written.
In case random variables are not sufficiently instantiated to 
exploit expressions for inferring the values of other variables, 
inference will return an error.

\subsubsection{Distributional Clauses Syntax}
\label{dc}
You can also use the syntax of Distributional Clauses (DC) \cite{Nitti2016}.
Continuous random variables are represented in this case by term whose distribution can be specified with density atoms as in
\begin{verbatim}
T~Density' := Body.
\end{verbatim}
Here \verb|:=| replaces the implication symbol, \verb|T| is a term and \verb|Density'| is one of the density atoms above without the \verb|Var| argument, because \verb|T|
itself represents a random variables. In the body of clauses you can use the infix operator \verb|~=| to equate a term representing a random variable with a logical variable or
a constant as in \verb|T ~= X|. Internally \verb|cplint| transforms the terms representing random variables into atoms with an extra argument for holding the variable.
DC can be used to represent also discrete distributions using
\begin{verbatim}
T~uniform(L) := Body.
T~finite(D) := Body.
\end{verbatim} 
where \verb|L| is a list of values and \verb|D| is a list of couples \verb|P:V| with \verb|P| a probability and \verb|V| a value.
If \verb|Body| is empty, as in regular Prolog, the implication symbol \verb|:=| can be omitted.

The Indian GPA problem from \url{http://www.robots.ox.ac.uk/~fwood/anglican/examples/viewer/?worksheet=indian-gpa}in distributional clauses syntax  (\url{https://github.com/davidenitti/DC/blob/master/examples/indian-gpa.pl})
takes the form (\href{http://cplint.eu/example/inference/indian_gpadc.pl}{\texttt{indian\_gpadc.pl}}):
\begin{verbatim}
is_density_A:0.95;is_discrete_A:0.05.
% the probability distribution of GPA scores for American students is
% continuous with probability 0.95 and discrete with probability 0.05

agpa(A): beta(A,8,2) :- is_density_A.
% the GPA of American students follows a beta distribution if the
% distribution is continuous

american_gpa(G) : finite(G,[4.0:0.85,0.0:0.15]) :- is_discrete_A.
% the GPA of American students is 4.0 with probability 0.85 and 0.0
% with 
% probability 0.15 if the
% distribution is discrete
american_gpa(A):- agpa(A0), A is A0*4.0.
% the GPA of American students is obtained by rescaling the value of
% agpa
% to the (0.0,4.0) interval
is_density_I : 0.99; is_discrete_I:0.01.
% the probability distribution of GPA scores for Indian students is
% continuous with probability 0.99 and discrete with probability 
% 0.01
igpa(I): beta(I,5,5) :- is_density_I.
% the GPA of Indian students follows a beta distribution if the
% distribution is continuous
indian_gpa(I): finite(I,[0.0:0.1,10.0:0.9]):-  is_discrete_I.
% the GPA of Indian students is 10.0 with probability 0.9 and 0.0
% with
% probability 0.1 if the
% distribution is discrete
indian_gpa(I) :- igpa(I0), I is I0*10.0.
% the GPA of Indian students is obtained by rescaling the value 
% of igpa
% to the (0.0,4.0) interval
nation(N) : finite(N,[a:0.25,i:0.75]).
% the nation is America with probability 0.25 and India with 
% probability 0.75
student_gpa(G):- nation(a),american_gpa(G).
% the GPA of the student is given by american_gpa if the nation is 
% America
student_gpa(G) :- nation(i),indian_gpa(G).
% the GPA of the student is given by indian_gpa if the nation 
%is India
\end{verbatim}
See 
   % --.4.89

\section{Input and Output of Specifications}
\label{InputOutput}

CEC has a variety of different I/O-commands. There are two commands for reading in
original specifications from a file (\comRef{in} and \comRef{enrich})
two commands for saving and restoring partially completed specifications 
to/from files (\comRef{freeze} and \comRef{thaw})
two commands for saving and restoring specifications to/from 
specification variables (\comRef{store} and \comRef{load}) and
two commands for storing and loading {\em log-files}
(\comRef{storeLog} and \comRef{loadLog}).
Also, the commands for saving and 
restoring the whole CEC state, \comRef{saveCEC} and \comRef{restoreCEC}, can be used. But
notice that saving CEC states (prolog states) usually generates very
large files. 
%Conventions concerning the allowed file and variable names are 
%those of the underlying prolog system. 
Reading specifications from standard input
(keyboard, terminal) is possible by using the predefined file name \cec{user}.


\subsection{Reading Specifications from Files}
\label{InCommand}

\begin{command}[\com{in}{\comArg{ModuleName}\ad\comArg{OrderName}}]
reads in a specification from the file \comArg{ModuleName}\suffix{.eqn}
and the associated order specification from the file
\comArg{ModuleName}\suffix{.}\comArg{OrderName}\suffix{.ord}.
As log-files are ``enriched'' order specifications, any log-file can
be used as an order file.
The  order  base of the order specification determines  the order names for the direct
imports of a specification. The system will first look for the
specification variable 

\hbox to \hsize{\hfill
\nt{specificationName}\kw{.}\nt{orderName}\hfill.}
If such a variable exists its content will be used. If such a variable does not
exist and some \nt{orderName} $\neq$ \kw{noorder} is associated with  
\nt{specificationName} the system will look for the file

\hbox to \hsize{\hfill
\nt{specificationName}\kw{.}\nt{orderName}\suffix{.q2.0}\hfill}
\noindent in the current directory containing a frozen state of the
referenced specification.
\noindent
If this file does'nt exist the system looks for a file

\hbox to \hsize{\hfill
\nt{specificationName}\suffix{.eqn}\hfill}
\noindent
which contains the specification according to the syntax of 
\nt{specification} and for a file 

\hbox to \hsize{\hfill
\nt{specificationName}\kw{.}\nt{orderName}\suffix{.ord}\hfill}
\noindent
which contains the order specification for the specification.
If the \nt{orderName} associated with \nt{specificationName} is 
\kw{noorder} the system looks for the specification in file

\hbox to \hsize{\hfill
\nt{specificationName}\suffix{.eqn}\hfill}
\noindent
in the current directory.

If not stated otherwise these files are assumed to be in the current 
directory. Before the specification is read in, CEC will be re-initialized, 
e.g. the current specification will be deleted. Specifications saved in 
variables will not be affected. \comArg{ModuleName} and \comArg{OrderName}
must be Prolog atoms. \comArg{OrderName} becomes the
current order name for the specification.\\
\com{in}{\comArg{ModuleName}\ad\cec{noorder}}
has the effect that no order specification is consulted.
The termination ordering for the new
specification will be initialized to a default value 
(\kw{neqkns} or \kw{poly1}, depending on the presence of AC-operators).
Using \comRef{in} only with the parameter \comArg{ModuleName} yields the same
effect. \comArg{ModuleName} = \kw{user} expects input from terminal.
(For Quintus-Prolog2.x under EMACS: {\it in} without
parameter reads from \cec{Scratch.pl}).
\end{command}

If you want to access files from other directories you will have to specify these
directories relative to the current directory at the time of invoking Prolog, e.g.
\cec{in('examples/math/int')}. A more convenient way will be to specify the
necessary prefix to all files once and for all by\bigskip

\begin{command}[\com{cd}{\comArg{Path}}]
Changes, as the cd command in UNIX, the directory for all following
file-related CEC-commands.
The path is given in form of a Prolog atom, hence don't forget the
quotes, if the path contains `\kw{/}', `\kw{.}', or `\kw{..}' or other special
characters.

\comRef{cd} is predeclared as prefix operator. So after execution 
of \bigskip

\cec{cd 'examples/math'.} \bigskip

\noindent
the command \cec{in(int)} will read in \cec{examples/math/int.eqn}.

\comRef{cd} without an argument resets the current directory to the one in
which the CEC-system was initially invoked.
\end{command}

\begin{command}[\comName{pwd}]
prints out the current path for I/O-related commands.
\end{command}

\begin{command}[\com{enrich}{\comArg{ModuleName}\ad\comArg{OrderName}}]
reads in additional parts of a specification from the files 
\comArg{ModuleName}\suffix{.eqn} and \comArg{ModuleName}\suffix{.}\comArg{OrderName}\suffix{.eqn}
after saving the current state for later checks for consistency of the enrichment. 
These additional parts must form an enrichment (cf. chapter~\ref{enrichment}).
\comArg{ModuleName} and \comArg{OrderName} can be arbitrary Prolog atoms.
The \comArg{OrderName} or even both arguments 
can be omitted, with similar effects as for \comRef{in}.
%Specify \comArg{ModuleName} = \kw{user} if input from terminal is wanted.
\end{command}

\subsection{Freezing and Thawing Partially Completed Specifications to/from Files}
\label{FreezeCommand}
\label{ThawCommand}

For saving and restoring states of specifications to/from files there exist the 
commands \comRef{freeze} and \comRef{thaw}.\bigskip

\begin{command}[\com{freeze}{\comArg{ModuleName}\ad\comArg{OrderName}}]
writes the state of the current specification to the file
\comArg{ModuleName}.\comArg{OrderName}\suffix{.q2.0}.
If freeze is called without \comArg{OrderName} the state is written
to \comArg{ModuleName}\suffix{.q2.0}.
%the current order name is taken for \comArg{OrderName}. 
If freeze is called without any argument at all the module name of the current 
specification is used for \comArg{ModuleName} and the current order name
is used for \comArg{OrderName}.
In this case, also a log-file is produced.
The specification may later be reused by thawing it from this file, cf. the 
\comRef{thaw}-command.
The state of CEC remains unchanged by this operation.
\end{command}

\begin{command}[\com{thaw}{\comArg{ModuleName}\ad\comArg{OrderName}\ad\comArg{SpecificationVariable}}]
This command is the inverse operation of \comRef{freeze} and restores the specification
previously frozen in \comArg{ModuleName}\suffix{.}\comArg{OrderName}\suffix{.q2.0}
to the specification variable \comArg{SpecificationVariable}. The current specification and other variables 
will not be affected by this operation.
If \comRef{thaw} is called without \comArg{SpecificationVariable} 
the current specification is overwritten by the thawed specification. 
Specifications stored in variables will not be affected by this operation.
If \comRef{thaw} is used only with argument \comArg{ModuleName} the frozen 
specification will be taken from the file \comArg{ModuleName}\suffix{.q2.0}.
\end{command}

\subsection{Storing and Loading of Log-files}
\label{StorelogCommand}
\label{LoadlogCommand}

Log-files are used to save the information given from the user during the last
completion process and to save the definition of the current termination
ordering. The log-file can be used to replay a completion fully (or partially)
automatically on ``closely related'' specifications. \bigskip

\begin{command}[\com{storeLog}{\comArg{ModuleName}\ad\comArg{OrderName}}]
creates the log-file.
The name of the log-file is 
\comArg{ModuleName}\suffix{.}\comArg{OrderName}\suffix{.@.ord}.
It has the format of an order specification file which can be used
with the \comRef{in}-command or the \comRef{loadLog}-command. 
If \comRef{storeLog} is used only with argument \comArg{ModuleName}
the log-file is named \comArg{ModuleName}\suffix{.@.ord},
if \comRef{storeLog} is used without any argument, the file is
created as \nt{moduleName}\suffix{.}\nt{orderName}\suffix{.@.ord},
with names as they are currently associated with the specification.
\end{command}

\begin{command}[\com{loadLog}{\comArg{ModuleName}\ad\comArg{OrderName}}]
reads in the file \comArg{ModuleName}\suffix{.}\comArg{OrderName}\suffix{.@.ord}. 
%This file contains all the answer given during
%the completion process and the final termination ordering saved using
%the \comRef{saveLog}-command. 
If the completion process is started again, % all
questions whose answers are already contained in 
\comArg{ModuleName}\suffix{.}\comArg{OrderName}\suffix{.@.ord} 
will be suppressed. 
If \comRef{loadLog} is used without the argument \comArg{OrderName} the 
information will be taken from the file \comArg{ModuleName}\suffix{.@.ord}.
If \comRef{loadLog} is used without any argument
the name of the current specification together with the current order name
will be used.
%The current order name is the
%order name of the order specification for the current specification
%or the current termination ordering if no order specification was used.
\end{command}

\subsection{Assigning and Retrieving Specifications to/from Specification Variables }
\label{StoreCommand}
\label{LoadCommand}

Some operations on specifications like \comRef{combine}, cf. the
\comRef{combineSpecs}-command, need a way to reference different specifications. In CEC
this is done by storing specifications to named variables and referencing them
afterwards by these variable names. Variable names are arbitrary Prolog atoms.\bigskip

\begin{command}[\com{store}{\comArg{ModuleName}\ad\comArg{OrderName}}]
saves the current specification in a specification variable named
\comArg{ModuleName}\kw{.}\comArg{OrderName}. If \comRef{store} is used only with
argument \comArg{ModuleName} the specification is saved in a variable with this name,
if \comRef{store} is used without any argument, the name is created as 
\nt{moduleName}\kw{.}\nt{orderName}, with names as they are currently associated 
with the specification. For later restoring 
use the command \comArg{load}. The system remains unchanged except for this variable 
containing afterwards the current specification.
\end{command}

\begin{command}[\com{load}{\comArg{ModuleName}\ad\comArg{OrderName}}]
loads the system which is currently the value of the variable
\comArg{ModuleName}\kw{.}\comArg{OrderName}, cf. the \comRef{store}-command. 
If \comRef{load} is used only with argument \comArg{ModuleName}, this actual parameter 
completely specifies the name of the variable.
Specification variables remain unchanged.
\comArg{StateName} = \cec{'$initial'} re-initializes the system.
\end{command}

   % 2.5.89
\section{Displaying and Changing the State of a Specification}

\subsection{Displaying the State of a Specification}
\label{Displaying}

\begin{command}[\comName{moduleName}]
displays the name of the current specification.
\end{command}

\begin{command}[\comName{orderName}]
displays the name of the order specification associated with
the current specification.
\end{command}

The state of a specification consists of the current signature, the set of equations
and rules and of the current definitions for the termination ordering. There are
several commands to print out this information.\bigskip

\begin{command}[\comName{sig}]
prints out the signature of the current specification:
\end{command}

If you are writing an order-sorted specifications there exist
additional, automatically generated operators which result from the 
translation into a many-sorted specification:
For every subsort relation s $<$ s' there will be an injective operator
with domain s und codomain s', called \verb|$inj-|s'\verb|-|s. If the domain can be
determined from the context, \verb|$inj-|s'\verb|-|s is abbreviated by s'.
These additional operators are also shown by the \comRef{sig}-command.\bigskip

\begin{command}[\comName{show}]
shows the sets of equations, rules and nonoperational equations of the 
current specification in order-sorted form.
(Usually there exists more than one many-sorted representation of
an order-sorted axiom.)

Rules \condRule{C}{\rewRule{l}{r}} which are marked by an asterix 
\kw{*} have an associated auxiliary rule of form
\condRule{C}{\rewRule{l+X}{r+X}}
where $+$ is the AC-operator on top of $l$ and $X$ is a new
variable of appropriate sort.
Auxiliary rules are automatically generated when needed during
completion modulo AC.
\end{command}

\begin{command}[\comName{showms}]
shows the sets of equations, rules and nonoperational equations of the 
current specification in many-sorted form.
\end{command}

\subsection{Inspecting the Termination Ordering}
\label{InspecTermination}

\begin{command}[\comName{order}]
indicates the current termination ordering and asks the user whether he wants to 
change it.
\end{command}

\begin{command}[\comName{operators}]
displays all precedences and stati in \kw{kns} or \kw{neqkns} or
all polynomial interpretations in \kw{poly}\nt{N} respectively.
\end{command}

\begin{command}[\comName{interpretation}]
displays all operator interpretations, provided \kw{poly}\nt{N} is the chosen termination
ordering and asks the user if he wants to change any. {\em If so all rules will be turned 
back into equations.}
\end{command}


\subsection{Checking Preregularity and Regularity}

To ensure that the set of order-sorted equations of the specification and
the set of rewrite rules produced by CEC describe the same equational
theory, the signature must be {\em preregular} (\refArrow \cite{SNGM87}).
If in addition the signature is {\em regular} the term algebra is an
initial algebra in the semantics of \cite{GM87} (and \cite{SNGM87}).

\noindent
There are commands which check preregularity and regularity:\medskip

\begin{command}[\comName{preregular}]
succeeds if the current (order-sorted) signature is preregular,
and fails otherwise.\\
The preregularity condition is the regularity of 
\cite{SNGM87}:
A signature is preregular, iff for every function symbol
$f$ and every string $w$ of sorts the set
$$   \{ t \mid \mbox{\ there is a declaration\ } f : w' \rightarrow t 
\mbox{\ such that\ } w \leq w' \} 
$$
is either empty or has a minimal element.
\end{command}

\begin{command}[\comName{regular}]
succeeds if the current (order-sorted) signature is regular,
and fails otherwise.
The regularity condition is the one of \cite{GM87}:
A signature is regular, iff for every function symbol
$f$ and every string $w$ of sorts the set
$$
   \{ (w',t) \mid \mbox{\ there is a declaration\ } f : w' \rightarrow t 
      \mbox{\ such that\ } w \leq w' \}
$$
is either empty or has a minimal element.
\end{command}

\subsection{The UNDO Mechanism}

\begin{command}[\comName{undo}]
can be entered at the system's top level to reset the system to the state
before the last command that has caused a state change, if there was
any. \comRef{undo} can be used repeatedly to undo several steps of state changes.
It also undoes \comRef{undoUndo}-calls. At the moment, there is no way to backtrack
from single decisions that have been taken while running the completion
process.
\end{command}

\begin{command}[\comName{undoUndo}]
allows to undo the last \comRef{undo}-command at the system's top level. Repeated use of
this command undoes sequences of \comRef{undo}-commands. Chains of undo-calls begin at 
the last user interaction different from an \comRef{undo} or \comRef{undoUndo}.
\end{command}


  % 16.3.89
\section{Operations on Specifications}
\label{OperationsOnSpecifications}

The specification building operators \kw{+} and \kw{rename} in 
module expressions can also be invoked interactively.
%As CEC is planned to be an experimental system for specification (program) development
%it offers several operations for combining specifications. The operations include the
%renaming of operator and sort names as well as the union of two specifications.

With these primitive operations more complex ones, such as passing actual parameters to
specification modules can be derived, cf. chapter \ref{exampleSession}.
%The parameter passing is
%demonstrated by the example of sorted lists over an arbitrary ordered set, that is
%replaced by the set of natural numbers \\
%(\refArrow file \file{cec/KBMANUAL/combining\_example} in this distribution).

\subsection{Renaming of Operator and Sort Names}

\begin{command}[\com{renameSpec}{\kw{[}\comArg{OldSort1} \kw{<-} \comArg{NewSort1}, \ldots , 
\comArg{OldSortN} \kw{<-} \comArg{NewSortN},\\ \hspace*{6.2em}
\comArg{OldOperator1} \kw{<-} \comArg{NewOperator1}, \ldots , 
\comArg{OldOperatorM} \kw{<-} \comArg{NewOperatorM}\/\kw{]}}]
renames the current specification according to the given lists of sort associations
and operator associations. Only injective renamings of operators are allowed.
Sorts may be renamed arbitrarily. Sorts and operators which are not mentioned remain 
unchanged.
\end{command}

%The system tries to carry over the  previous termination proof to the renamed
%specification. This is not always possible, as the collapsing of previously distinct
%operator symbols might not be compatible with precedences or abstract interpretations.
%In such a case, all rules will be turned back into equations.


\subsection{Combining of Specifications}

The combine operator forms the union of two specifications if they can be combined, 
that is if their termination orderings can proved to be compatible. If so, the two orderings
are combined. \cec{kns} and \cec{poly}\nt{N} are assumed to be incompatible.\bigskip

\begin{command}[\com{combineSpecs}{\comArg{StateName1}\ad\comArg{StateName2}\ad\comArg{CombinedSpec}}]
The specifications  stored (\refArrow \comRef{store}) 
in specification variables \comArg{StateName1} and \comArg{StateName2}
will  be combined, if possible, by forming
the union of the signature, axioms and pragmas.
If \comArg{CombinedSpec} $\neq$ \cec{user}, the result will be stored in
\comArg{CombinedSpec}, and the current specification will not be affected.
Otherwise, the combined specification becomes the new current
specification.
\comRef{combineSpecs} requires the compatibility of the termination
orderings of the involved specifications:
\begin{itemize}
\item
Termination orderings of the same type may be combinable, also \kw{kns} together with
\kw{neqkns}, yields \kw{kns} for the combined specification.
In the case of \kw{kns} and \kw{neqkns} 
the combination fails if the operator precedences are contradictory,
e.g.\ $f > g$ in the first specification but $g \geq f$ in the second 
specifcation. Also, different stati for operators are not compatible.
\item
\kw{poly}\nt{N} and \kw{poly}\nt{M} can be combined 
into \kw{poly}\nt{N} if $N \geq M$. In addition, operators 
which occur in both operand specifications must have
the same interpretation polynomials in the first M components.
\end{itemize}
\end{command}
   % 2.5.89
\section{Completion of Specifications}

\subsection{The Objectives of Completion}

CEC is designed to support the methodology of software specification 
using modular order-sorted specifications with conditional equations.
Being considerably more concise and abstract than many-sorted specifications,
order-sorted specifications even more call for a system that provides
proof techniques needed for
\begin{itemize}
\item checking the consistency of a specification,
\item proving the correctness of actual specification parameters to
parametric modules, and
\item checking, or achieving by transformation, confluence and termination
as a prerequisite
for correct operational execution of specifications by (conditional)
term rewriting.
\end{itemize}
The completion procedure in CEC can be seen as both a compiler and a refutation
proof procedure.
For the latter, CEC distinguishes between two kinds of operators.
Operators can be declared
as {\em constructors} or as regular operators.
It is assumed that any two constructor terms
are different in a consistent equational theory.
If the completion procedure in CEC infers an equation between two
different constructor terms it will stop and report the inconsistency.
One application of this is checking the consistency of parameter passing.
Actual and formal parameter specifications are combined, possibly after
some renaming of sorts or operators, and then completed.
If the actual parameter is {\em constructor-complete} and the completion
process 
discovers no inconsistency then the actual parameter is correct, i.e.
the initial algebra of the actual parameter specification is a model of the
formal one.
This is exactly the {\em ``proof-by-consistency''} method of inductive theorem
proving, and in \cite{HH80} it is shown how to incorporate it into
a completion procedure for the particular case of constructor-complete
theories. In CEC this method is extended to the conditional case.


Unfailing conditional completion
transforms a finite initial set of conditional 
equations $E$ into a possibly infinite
set $E'$ of final conditional equations such that
\footnotetext[1]{
\rew{E} denotes conditional rewriting using the equations in $E$ from left to right,
by \rewRT{E} and \rewSRT{E} we denote the reflexive-transitive and
the reflexiv-transitive-symmetric closure of \rew{E} respectively.}
$$\eqrel{*}{E} \ =\  \rewRT{R(E')} \irewRT{R(E')} \ =:\ \nfrel{R(E')},\footnotemark[1]$$
where $R(E') = \{ \sigma(e)| e \in E' \cup E'^{-1},
\sigma(e) {\rm\ reductive} \}$\footnote[2]{
$E'^{-1} = \{ \condEq{C}{\eq{t}{s}} \ |\  \condEq{C}{\eq{s}{t}} \in E'\}$.}.
This means that, for any two terms
$t_1$ and $t_2$, $t_1$
and $t_2$ are equal in the equational theory, iff 
$t_1$ and $t_2$ have the same normalform with
respect to $R(E')$. $R(E')$ is the set of all reductive
instances of the
equations in $E'$. With the definition of reductivity
as given below, unfailing completion of unconditional equations
is just a particular case of the conditional one.
Rewriting in $R(E')$ means that a redex is always replaced
by a smaller one. In the case of a conditional equation
it also means that the condition instance must be smaller
than the redex, too.

The notion of a reductive conditional equation has originated from
\cite{Kap84},\cite{Kap85} and \cite{JW86}. 
A {\em reduction ordering} is a partial ordering on terms that is
stable under substitution, monotonic and well-founded.
Given a reduction ordering $>$
on the term algebra, an equation
$$ \condEq{\cond{\eq{u_1}{v_1}\condAnd\eq{u_2}{v_2}}
                {\eq{u_n}{v_n}}}
          {\eq{s}{t}} $$
is called {\em reductive}, if $s > t$, $s > u_i$,  and $s > v_i$,
i.e. if the term on the right side and each term that occurs in the condition
are smaller than the left side of the equation. 
In the confluent case, verifying an instance $\eq{u}{v}$ of a condition
equation then means to check whether or not
both terms $u$ and $v$ have the same
normal form. As these terms are smaller than the redex, the applicability
of a reductive equation is always decidable.
 
The main problem is that $R(E')$ is infinite even for finite $E'$.
A particularly fortunate case is when all equations in the final
$E'$ are reductive. As reduction orderings are stable under substitutions,
$\eq{\rew{R(E')}}{\rew{E'}}$, in this case.
Unfortunately in almost all cases $E'$ will not be uniformly reductive.
This is in particular so when the initial $E$ already contains
nontrivial nonreductive equations.
Any equation with extra variables is nonreductive.
Even if the initial equations are reductive, nonreductive ones
can be generated from these during completion.

\subsection{CEC Completion}
\label{Completion}

The completion procedure in CEC is based on three
observations:
\begin{enumerate}
\item
Reductivity can be weakened to what we call {\em quasi-reductivity}.
\item
Depending on the completion strategy, $E'$ can be redundant.
It can contain {\em nonoperational equations}.
$e \in E'$ is called nonoperational, if $\nfrel{R(E')} = \nfrel{R(E'-\{e\})}$,
i.e. if $e$ need not be used for computing normal forms.
\item
The completion strategy can be tailored so
that {\em arbitrarily selected equations} with at least one condition become
nonoperational in the final system.
\end{enumerate}
Hence, whenever the reductivity of a conditional equation is
violated because of the ``size'' of a particular condition, two cases may occur.
If the equation is quasi-reductive, then it can be handled
almost as if it was reductive. Otherwise, the equation
can be classified as ``should become nonoperational''. Then, completion
computes superpositions on its condition such that
in the final system the equation is nonoperational indeed. 

Let's now explain the notion of quasi-reductivity more precisely.
Let
$$ \condEq{\cond{\eq{u_1}{v_1}\condAnd\eq{u_2}{v_2}}
                {\eq{u_n}{v_n}}}
          {\eq{s}{t}} $$
be a conditional equation, $n\ge 1$. It is called {\em quasi-reductive}
if
there exists a sequence $h_i(\xi)$ of context terms,
such that $s>h_1(u_1)$, $h_i(v_i)\ge h_{i+1}(u_{i+1})$,
$1\le i < n$, and $h_n(v_n)\ge t$.

Quasi-reductivity is a proper generalization of reductivity.
If the equation
$$\condEq{\cond{\eq{u_1}{u_{n+1}}}
               {\eq{u_n}{u_{2n}}}}
         {\eq{s}{t}}$$ 
is reductive, then the equation
$$
\condEq{\cond{\eq{u_1}{x_1}\condAnd\eq{u_{n+1}}{x_1}}
             {\eq{u_n}{x_n}\condAnd\eq{u_{2n}}{x_n}}}
       {\eq{s}{t}},
$$
is quasi-reductive,
if the $x_i$ are new, pairwise distinct variables.
Hence, where appropriate we assume that reductive equations are
identified with quasi-reductive ones in the way just explained.

Quasi-reductive conditional rewriting $\rewqr{E}$
with quasi-reductive equations $E$
is defined as classical reductive rewriting with
conditional equations.
The only difference is that goal solving for the conditions is restricted to
{\em oriented goal solving}.
Hereby a substitution $\sigma$ is a solution to a
condition equation $\eq{u}{v}$,
if $u\sigma\rewqrnf{E}v\sigma$, i.e. the right side
$v$ of the condition matches
the normal form of the $\sigma$-instance $u\sigma$ of the left side.
The reductivity requirements for quasi-reductive equations
now imply that $\rewqr{E}\subset\rew{E}$, i.e. any step
of quasi-reductive rewriting is at the same time a step of reductive
rewriting. Moreover, for a quasi-reductive equation and
any matching substitution $\sigma$ for the left side $s$
there is at most one extension of $\sigma$ to a directed solution of
the condition. In addition, $\sigma$ can be constructed deterministically if
$\rewqr{E}$ is confluent. Altogether we see that $\rewqr{E}$
is as efficient as rewriting with reductive equations.

To sum up, completion in CEC produces a possibly infinite $E'$ from an initially given $E$
such that
$$ E' = R_\infty + N_\infty, $$
where
$R_\infty = R_\infty^{red} + R_\infty^{qred}$ is a set of {\em reductive} equations and
{\em quasi-reductive} equations 
and $N_\infty$ is a set of {\em nonoperational} equations,
such that $\rew{R(E')}$ is confluent,
$$\rewSRT{E} = \nfrel{R(E')}$$
{\em and}
$$\rew{R(E')} \subset
\rewqrRT{R_\infty} \irewqrRT{R_\infty}
= \rewSRT{E'}.$$
Hence $\rewqr{R_\infty}$ is confluent too,
efficiently computable and
produces the same normal forms as $\rew{R(E')}$.
%Note that since
%$\rew{E'} \subset \rew{R(E')}$ but $\rew{E'} \neq \rew{R(E')}$
%in general, to achieve
%$$\rew{R(E')} \subset \rewqrRT{R_\infty} \irewqrRT{R_\infty}$$
%completion must perform sufficiently many paramodulation superpositions
%on the right sides of the conditions of the equations in $R_\infty^{qred}$.

We will formalize the CEC-completion procedure within the framework of
an inference system. Since we distinguish between equations, rewrite rules and
nonoperational equations, the objects of this inference system are tupels 
$\tripel{E}{R}{N}$, where $E$ is a set of conditional equations,
$R$ is a set of (quasi-) reductive conditional rewrite rules and $N$ is a
set of nonoperational equations.

The CEC\--com\-pletion procedure computes (possibly infinite) se\-quen\-ces 
$$\tripel{E_0}{R_0}{N_0},\ \tripel{E_1}{R_1}{N_1}, \ldots$$ 
of derivations
using the in\-fer\-ence rules presented below. The limit of a derivation is the
tuple $\tripel{E_\infty}{R_\infty}{N_\infty}$. 
In the following, $P_\tripel{E}{R}{N}\ :\ \eq{s}{t}$ means that $P$ is a proof of
\eq{s}{t} consisting of applications of equations and rules in \tripel{E}{R}{N}. 

The basic idea behind the completion is that in any new state
$\tripel{E_i}{R_i}{N_i}$ proofs become simpler w.r.t. 
a well-founded {\em proof ordering} \PO .
The simplest proofs of equations are rewrite proofs, cf. \cite{Gan87b}.
The goal
 is to construct for each proof 
$P_\tripel{E_0}{R_0}{N_0}\ :\ \eq{s}{t}$ a rewrite proof 
$Q_\tripel{E_i}{R_i}{N_i}\ :\ \eq{s}{t}$ at some step $i$,
yielding 
$$\rewSRT{E} = \nfrel{R_\infty}$$
in the limit.
With this idea in mind we say
that a conditional equation \condEq{C}{\eq{s}{t}} is {\em trivial} if for all
proofs $P : \eq{m}{n}$, which apply \condEq{C}{\eq{s}{t}} there exists a proof
$Q : \eq{m}{n}$ such that $P \PO Q$.

\noindent
If for some $k$ 
\begin{itemize}
\item all critical pairs between rules in $R_k$ and 
\item all superpositions on some condition of any  equation in $N_k$
by the rules in $R_k$ 
\end{itemize}
have been computed and 
\begin{itemize}
\item all equations in $E_k$ and 
\item all unconditional equations in $N_k$ (except for the AC-axioms)
\end{itemize}
have been shown to be trivial then $\tripel{E_k}{R_k}{N_k}$ is complete.
% for each proof \stateEl{P_\tripel{E_0}{R_0}{N _0}}{\eq{s}{t}} there 
% exists a rewrite-proof \stateEl{Q_\tripel{E_k}{R_k}{N_k}}{\eq{s}{t}}. 
Therefore
the completion procedure either tries to show that an equation
in $E$ is trivial or it
transforms the equation into a rewrite rule
or into a nonoperational equation. 

Unfortunately the property of a conditional equations being trivial is not decidable.
CEC applies several advanced techniques for detecting the convergence of
conditional equations cf. \cite{Gan88a}. The CEC-technique of using nonoperational
equations for elimination will discussed together with
the inference rule {\em Deleting a trivial equation} below. \bigskip

\begin{CRule}[(cp), Adding a contextual critical pair]
\deducRule{ E , R \cup \{ \condRule{C}{\rewRule{s}{t}},  
                       \condRule{D}{\rewRule{l}{r}} \} , N}{
 E \cup \set{
\condEq{\applysubst{\sigma}{C} \condAnd \applysubst{\sigma}{D}}{
\eq{\applysubst{\sigma}{\replace{s}{o}{r}}}{\applysubst{\sigma}{t}}}}, 
R \cup \set{ \condRule{C}{\rewRule{s}{t}}, \condRule{D}{\rewRule{l}{r}}} , N} 
\end{CRule}

\noindent
where $o$ is a nonvariable occurrence in $s$ so that \subterm{s}{o} and $l$ 
can be unified with the mgu $\sigma$.\bigskip

\begin{CRule}[(superpose (condition)), Adding a condition superposition instance]
\deducRule{ E , R \cup \set{ \condRule{D}{\rewRule{l}{r}}} , 
          N  \cup \set{ \condEq{C \condAnd \eq{u}{v}}{\eq{s}{t}}} }
{E \cup \set{ 
\condEq{\applysubst{\sigma}{D} \condAnd
\applysubst{\sigma}{C} \condAnd \applysubst{\sigma}{\replace{(\eq{u}{v})}{o}{r}}}{
\eq{\applysubst{\sigma}{s}}{\applysubst{\sigma}{t}}}} , 
 R \cup \set{ \condRule{D}{\rewRule{l}{r}}} , 
 N \cup \set{ \condEq{C \condAnd \eq{u}{v}}{\eq{s}{t}}} }
\end{CRule}

where $o$ is a nonvariable occurrence in \eq{u}{v} such 
that \subterm{(\eq{u}{v})}{o} and $l$ can be unified
with the mgu $\sigma$. The rewrite rule \condRule{D}{\rewRule{l}{r}} is 
used for superposition on the condition equation \eq{u}{v} 
in the nonoperational equation \condEq{C \condAnd \eq{u}{v}}{\eq{s}{t}}. \bigskip

\begin{CRule}[(superpose (reflexivity)), Adding a condition superposition instance]
\deducRule{ E , R , N \cup \set{\condEq{C \condAnd \eq{u}{v}}{\eq{s}{t}}} }
        { E \cup \set{\condEq{\applysubst{\sigma}{C}}{
\eq{\applysubst{\sigma}{s}}{\applysubst{\sigma}{t}}}} , R ,
          N \cup \set{\condEq{C \condAnd \eq{u}{v}}{\eq{s}{t}}} }
\end{CRule}

\noindent
where $u$ and $v$ can be unified with a mgu $\sigma$.\bigskip

\begin{CRule}[(superpose (conclusion)), Adding a superposition instance]
\deducRule{ E , R \cup \set{ \condRule{D}{\rewRule{l}{r}}} , 
          N  \cup \set{ \eq{s}{t}}}
{E \cup \set{ 
\condEq{\applysubst{\sigma}{D}}{
\eq{\applysubst{\sigma}{\replace{s}{o}{r}}}{\applysubst{\sigma}{t}}}} , 
 R \cup \set{ \condRule{D}{\rewRule{l}{r}}} , 
 N \cup \set{\eq{s}{t}}} 
where $o$ is a nonvariable occurrence in $s$ so that $\subterm{s}{o}$
and $l$ can be unified with the mgu $\sigma$.
\end{CRule}\bigskip


\begin{CRule}[(orderEq), Orienting an equation into a reductive rule]
\deducRule{ E \cup 
 \set{\condEq{\eq{u_1}{v_1} \condAnd \eq{u_2}{v_2} \condAnd \ldots
 \condAnd \eq{u_n}{v_n}}{\eq{s}{t}}},
          R , N }{ E , R \cup 
 \set{\condRule{\eq{u_1}{v_1} \condAnd \eq{u_2}{v_2} \condAnd \ldots
 \condAnd \eq{u_n}{v_n}}{\rewRule{s}{t}}} , N }

if $\set{s} \TO> \set{u_1, v_1, u_2, v_2, \ldots, u_n , v_n, t} , n \geq 0 $.

\deducRule{ E \cup \set{\condEq{\eq{u_1}{v_1} \condAnd \eq{u_2}{v_2}
   \condAnd \ldots \condAnd \eq{u_n}{v_n}}{\eq{s}{t}}},
  R, N }{ E , R \cup \set{
\condRule{\eq{u_1}{v_1} \condAnd \eq{u_2}{v_2} \condAnd \ldots
\condAnd \eq{u_n}{v_n}}{\rewRule{t}{s}}} , N }

if $\set{t} \TO> \set{u_1, v_1, u_2, v_2, \ldots, u_n, v_n, s}, n \geq 0$ .
\end{CRule} \bigskip

\begin{CRule}[(orderEq), Orienting an equation into a quasi-reductive rule]
\deducRule{ E \cup 
 \set{\condEq{\eq{u_1}{v_1} \condAnd \eq{u_2}{v_2} \condAnd \ldots
 \condAnd \eq{u_n}{v_n}}{\eq{s}{t}}},
          R , N }{ E , R \cup 
 \set{\condRule{\eq{u_1}{v_1} \condAnd \eq{u_2}{v_2} \condAnd \ldots
 \condAnd \eq{u_n}{v_n}}{\rewRule{s}{t}}} , N }

if $\exists h_1(x),\ldots,h_n(x) \in T_\Sigma(X), x \in X, 
 s \TO> h_1(u_1), h_i(v_i) \geq h_{i+1}(u_{i+1}), 1 \leq i < n,
 h_n(v_n) \geq t$.

\deducRule{ E \cup \set{\condEq{\eq{u_1}{v_1} \condAnd \eq{u_2}{v_2}
   \condAnd \ldots \condAnd \eq{u_n}{v_n}}{\eq{s}{t}}},
  R, N }{ E , R \cup \set{
\condRule{\eq{u_1}{v_1} \condAnd \eq{u_2}{v_2} \condAnd \ldots
\condAnd \eq{u_n}{v_n}}{\rewRule{t}{s}}} , N }

if $\exists h_1(x),\ldots,h_n(x) \in T_\Sigma(X), x \in X,  
 t \TO> h_1(u_1), h_i(v_i) \geq h_{i+1}(u_{i+1}), 1 \leq i < n,
 h_n(v_n) \geq s$.
\end{CRule}
\vspace{1.5ex}

\noindent
The {\em reductivity} condition is essential to avoid an infinite recursive evaluation of the
condition when checking the applicability of a conditional rewrite rule.\bigskip
%As shown in \cite{Gan87b} a much more powerful simplification of
%equations and rules is possible in the presence of orientable condition equations. 
%Therefore the system tries to orient condition 
%equations. However, it is not required that each condition is orientable.\bigskip

\begin{CRule}[(nopEq), Declaring an equation as nonoperational]
\deducRule{ E \cup \set{\condEq{C}{\eq{s}{t}}}, R, N }
{ E, R, N \cup \set{\condEq{C}{\eq{s}{t}}} } 
\end{CRule}

\noindent
For any nonreductive initial equation the system asks if it should be
considered as nonoperational. In addition for every conditional equation generated during
completion by \comRef{cp} or \comRef{superpose} the user is asked to decide as to 
wether \comRef{orderEq} or \comRef{nopEq} should be applied. It may be the case 
that a generated conditional equation can be oriented into a reductive rewrite rule, 
nevertheless is operationally useless, and may in fact cause nontermination 
of the completion procedure.
Since it is sufficient to superpose all rewrite rules on just one condition of any
nonoperational conditional equation, CEC asks the user on which condition superposition should be 
applied. At this point it is often sensible to select a condition which is a 
maximal one w.r.t.\ the given reduction ordering. An exception might be the case
where such conditions contain AC-operators since superposition is then very 
costly. 
For unconditional equations only nonreductive equations should be
declared as nonoperational. In that case all rewrite rules must be superposed
on both sides of the nonoperational unconditional equation. With this feature
CEC provides a kind of (semi-)unfailing completion\footnote[1]{If such an equation cannot be eliminated eventually,
completion fails as CEC does not support rewriting with 
instances of unorientable equations}. \bigskip

\begin{CRule}[(redEq), Deleting a trivial condition of a conditional equation]
\deducRule{ E \cup \set{\condEq{C \condAnd \eq{u}{u}}{\eq{s}{t}}}, R, N}
{ E \cup \set{\condEq{C}{\eq{s}{t}}}, R, N }
\end{CRule}

\begin{CRule}[(redNopEq), Deleting a trivial condition of a nonoperational conditional equation]
\deducRule{ E, R, N \cup \set{\condEq{C \condAnd \eq{u}{u}}{\eq{s}{t}}} }
{ E, R, N \cup \set{\condEq{C}{\eq{s}{t}}} } 
\end{CRule}

\noindent Orientable condition equations in $C$ may 
be used as additional rewrite rules for simplification. Technically, simplification
with oriented condition equations requires the skolemization (replacement of
variables by new constants) of these 
equations. 
$\tilde{s}, \tilde{u}$ denote the skolemized versions of $ s, u $ and
$\tilde{C} $ the skolemized and oriented subset of $C$.\bigskip

\begin{CRule}[(redEq), Simplifying a condition of a conditional equation]
\ifdeducRule{ E \cup \set{\condEq{C \condAnd \eq{u}{v}}{\eq{s}{t}}}, R, N }
{ E \cup \set{\condEq{C \condAnd \eq{w}{v}}{\eq{s}{t}}}, R, N }if{
\tilde{u} \rewqr{R \cup \tilde{C}} \tilde{w}}
\end{CRule}

\noindent
A condition equation may be simplified under the assumption that the remaining condition 
equations hold true. Again, orientable condition equations may be used as additional
rewrite rules.\bigskip

\begin{CRule}[(redNopEq), Simplifying a condition of a nonoperational equation]
\ifdeducRule{ E, R, N \cup \set{\condEq{C \condAnd \eq{u}{v}}{\eq{s}{t}}} }
{ E \cup \set{\condEq{C \condAnd \eq{w}{v}}{\eq{s}{t}}}, R, N }if{
\tilde{u} \rewqr{R \cup \tilde{C} } \tilde{w}}
\end{CRule}

\noindent
Simplified nonoperational equations are turned back into conventional equations. \bigskip

\begin{CRule}[(redRule), Simplifying the condition of a conditional rewrite rule]
\ifdeducRule{ E, R \cup \set{\condRule{C \condAnd \eq{u}{v}}{\rewRule{s}{t}}}, N  }
{ E, R \cup \set{\condRule{C \condAnd \eq{w}{v}}{\rewRule{s}{t}}}, N }if{
\tilde{u} \rewqr{R \cup \tilde{C}} \tilde{w}}
\end{CRule}

\begin{CRule}[(redEq), Simplifying the conclusion of a conditional equation]
\ifdeducRule{ E \cup \set{\condEq{C}{\eq{s}{t}}}, R, N }
{ E \cup \set{\condEq{C}{\eq{u}{t}}}, R, N }if{\tilde{s} \rewqr{R \cup \tilde{C} } \tilde{u}}
\end{CRule}

\begin{CRule}[(redNOpEq), Simplifying the conclusion of a nonoperational conditional equation]
\ifdeducRule{ E, R, N \cup \set{\condEq{C}{\eq{s}{t}}} }
{ E, R, N \cup \set{\condEq{C}{\eq{u}{t}}} }if{
\tilde{s} \rewqr{R \cup \tilde{C}} \tilde{u}}
\end{CRule}


\begin{CRule}[(redRule), Simplifying the right hand side of a rule]
\ifdeducRule{ E, R \cup \set{\condRule{C}{\rewRule{s}{t}}}, N }
{ E, R \cup \set{\condRule{C}{\rewRule{s}{u}}}, N }if{
\tilde{t} \rewqr{R \cup \tilde{C} } \tilde{u}}
\end{CRule}

\begin{CRule}[(redRule), Simplifying the left hand side of a rule]
\ifdeducRule{ E, R \cup \set{\condRule{C}{\rewRule{s}{t}}} , N }
{ E \cup \set{\condEq{C}{\eq{u}{t}}}, R, N }if{
\tilde{s} \rewqr{R \cup \tilde{C} } \tilde{u}}
\end{CRule}

\noindent
Rewrite rules with simplified left hand side must be turned back into equations.
This rule has a further restriction for its application which we do not want to
mention here.\bigskip

\begin{CRule}[(redEq), Deleting a trivial equation]
\ifdeducRule{ E \cup \set{\condEq{C}{\eq{s}{t}}}, R, N }
{ E, R, N }if{\condEq{C}{\eq{s}{t}} \mbox{{\rm \  is\ trivial}}}
\end{CRule}

\noindent
Convergent equations may be eliminated. Particularly simple cases of
($redEq$) are $s \equiv t$ or $ s=t \in C $. Apart from usual
simplification and subsumption methods, CEC also applies nonoperational
equations from $N$ for constructing simpler proofs for equations from $E$.
The principal schema can be sketched as follows. 
The equation $\condEq{C}{\eq{s}{t}} \in E$ is trivial if the following
cases apply:
\begin{enumerate}
\item
$s \equiv t$ or $\eq{s}{t} \in C$.
\item
\condEq{C}{\eq{s}{t}} is subsumed by another equation from $E$ or $N$
\item
\condEq{C}{\eq{s}{t}} can be simplified to a trivial equation.
\item
$C$ contains an equation \eq{u}{v} that is not satisfiable for consistent
specifications. 
An equation \eq{u}{v} is unsatisfiable if $u$ and $v$ are not identical and
both consist of constructor operators only\footnote[1]{Note that CEC does 
not allow rules with constructors at the top of the left sides. Any such
situation leeds to the failure of the completion}.
\item
if \condEq{C \condAnd (\eq{\applysubst{\sigma}{m}}{\applysubst{\sigma}{n}})}{\eq{s}{t}}
is trivial, where $\condEq{D}{\eq{m}{n}} \in N$, such that $C$ 
implies \applysubst{\sigma}{D} 
and additional constraints
concerning the \PO - relation on the involved proofs are fulfilled
cf. \cite{Gan87b}.
We say that \condEq{C}{\eq{s}{t}} is {\it forward chained} using the nonoperational 
equation \condEq{D}{\eq{m}{n}}. CEC reports every forward chaining of equations and
the result of the corresponding comparisons on proofs.
\item
Similar to 5.\ with backward chaining (resolution) instead of forward chains.
\end{enumerate}

\subsection{Incremental Termination Orderings}
\label{TerminationOrderings}

As described above the definition of a appropriate {\em reduction ordering} is 
fundamental for the completion procedure.
An important improvement of completion procedures today over the original one
by Knuth and Bendix 
\cite{KB69} is the use of {\it incremental termination orderings}. 
In the original procedure the reduction ordering had to be given
{\it a priori}. Incremental orderings are orderings where the partial order on terms
is induced from another partial order on operators or on interpretations of operators
occurring in terms. If two incomparable terms $t_1$ and $t_2$ shall be ordered,
one looks for a consistent extension of the current ordering so that
$t_1  > t_2 $ or $t_2  > t_1 $. Since there are
usually several different possible extensions of the underlying ordering which all
induce the desired extension of the term ordering, implementations of these orderings
ask the user to select one.

Together with the possibility to write an order specification, the user is now
able to supply some information about the termination ordering right at the
beginning, to extend this information interactively during the completion process
and to save the final termination ordering in an order specification again.
So he avoids to answer the same questions concerning the termination ordering 
when he tries to complete the specification again.

The default termination ordering used by CEC is \kw{neqkns}, except if the specification
contains associative and commutative operators. In this case \kw{poly1} is the
default ordering since termination proofs with \kw{kns} or \kw{neqkns} 
are invalid in the presence of AC-operators. 

The user can change the termination ordering using the 
\comRef{oder}-command (\refArrow\ chapter \ref{InspecTermination}).

%\begin{command}[\comName{order}]
%indicates the current termination ordering and asks the user whether he wants to 
%change it.
%When invoked it displays the following: 
%
%\begin{screen}
%The current termination ordering is "neqkns".
%The following alternative orderings may be selected:
%
%recursive path ordering without equality (after Kapur, Narendran,
%Sivakumar)
%
%recursive path ordering (after Kapur, Narendran, Sivakumar)
%
%polynomial abstraction with tuplelength N
%
%manual ordering
%
%"file." for reading from a file
%
%"no." for no change
%   Please answer with neqkns. or kns. or poly<N>. or manual. or
%file. or no. (Type A. to abort) >
%\end{screen}
%
%If a new ordering is selected and if the previous termination
%ordering is incompatible with the new ordering, all rules are turned back into
%equations and the completion must be repeated from the beginning.
%\end{command}

\subsubsection{KNS and NEQKNS: Recursive Path Orderings}
\label{kns}

\kw{kns} and \kw{neqkns} are {\it precedence orderings}. That means the partial order on terms is
induced by a partial order on operators called the {\it precedence}. 
The precedence ordering \kw{neqkns} forbids that two different operators have the same
precedence. Initially the
precedence is empty. Order Specifications may include precedence declarations, cf. 
chapter~\ref{OrderPragmas}. 
All constructors are given a precedence less than any nonconstructor operator.
The precedence ordering is extended during 
the completion process or using the \comRef{equal} and \comRef{greater}
commands. The current state of the precedence definitions can be
displayed using the \comRef{operators} command.

\subsubsection{POLY$<$N$>$: Polynomial Orderings.}
\label{poly}

The second group of termination orderings available in CEC is \kw{poly}\nt{N}. Here the idea is to
give {\it polynomial interpretations} to terms in a way that
\[t_1 > t_2 : \iff I(t_1) > I(t_2). \]
An m-tuple of integer polynomials $F_i(x_1, \ldots, x_n)$ is associated with
each $n$-ary operator $f$. The choice of coefficents must ensure {\it monotonicity},
e.g.
\[
I(t_1 ) > I(t_2 )\ \  {\rm implies}\ \  I(f(\ldots, t_1, \ldots)) >
I(f(\ldots, t_2, \ldots)) \]
and that terms are mapped into nonnegative integers only; this is the case if all
coefficients are positive.

The concrete version of this technique as it is implemented CEC is due to \cite{CL86}. Again
we only need to order the equations to infer the termination property for all
reductions on terms. 
Changing to a smaller tuple length {\rm N} can
make old termination proofs invalid. In that case CEC turns back all rules
into equations. The default termination ordering in the presence of AC-operators
is \kw{poly1}.

The difficulty of these termination orderings from a users point of view is how to
guess the appropriate polynomials such that all equations will be ordered in the
desired way. It will take some time to become familiar with this technique.\bigskip

Order Specifications may include polynomial interpretations for operators, cf. 
chapter~\ref{OrderPragmas}. Interpretations are added during the 
completion process or using the \comRef{setInterpretation} command.
The current polynomial interpretations can be displayed using the
\comRef{operators} command or the \comRef{interpretation}-command
(\refArrow\ chapter \ref{InspecTermination}).

There are two operators which allow to compute the interpretation of a term
and to compare two interpretations:\bigskip

\begin{command}[\com{polGreater}{\comArg{Interpretation1}\ad\comArg{Interpretation2}}]
Only useful with ordering \kw{poly}\nt{N}.
If it succeeds, \comArg{Interpretation1} $>$ \comArg{Interpretation2} holds true
(if $>$ is the ordering on tuples of polynomials).
Interpretations (i.e. tupels of polynomials) of terms can be generated
via \comRef{polynomial}.
\end{command}

\begin{command}[\com{polynomial}{\comArg{Term}\ad\comArg{Interpretation}}]
yields the polynomial interpretation of \comArg{Term}. It
fails, if the ordering is not \kw{poly}\nt{N}. 
If there are operators in \comArg{Term},
for which no polynomial interpretation is known, the user is asked for
such an interpretation (and the given interpretation is stored).
\end{command}

\subsection{Specifications with Associative and Commutative Operators}

As a commutativity axiom immediately destroys the termination property of a term
rewriting system, this property cannot be expressed by a term rewrite rule. A well
known solution to this problem is unification and rewriting modulo associativity and
commutativity, cf. 
\cite{Sti81} and \cite{Fag83}. CEC must know the AC-operators of the
signature. CEC extracts this information from the given set of axioms and treats
these equations differently, i.e. they will not be oriented into rewrite rules.
The AC-equations will be added to the set of nonoperational equations, but in contrast
to other nonoperational equations they are not superposed with rewrite rules to compute 
coherence pairs. In CEC the extended rule technique is implemented instead,
cf.\ \cite{PS81} and \cite{JK86b}.

Note that in the presence of AC-operators termination proofs using path orderings
like \kw{kns} are invalid. But termination proofs with polynomial orderings may
be possible. Therefore CEC sets the default termination ordering
to \kw{poly1} if there are AC-operators in a specification. 


\subsection{Running the Completion Procedure}

\begin{command}[\comName{c}]
calls the Knuth-Bendix completion procedure. This executes a fixed strategy of
applications of the ``completion inference'' predicates \comRef{orderEq}, 
\comRef{cp}, \comRef{superpose}, \comRef{redRule}, \comRef{redEq} and
\comRef{redNopEq}.
\end{command}

\begin{command}[\comName{cResume}]
restarts the completion procedure after the completion process was
aborted by answering ``\cec{A.}'' to some query of the system.
\end{command}

\noindent
A manual guidance of the process is possible by explicitly calling
completion inference rules. \bigskip

\begin{command}[\com{orderEq}{\comArg{EquationIndex}}]
orients equation with index \comArg{EquationIndex}. The predicate fails if equation 
\comArg{EquationIndex} does
not exist or if the equation cannot be oriented or turned into a nonoperational
equation or if the equation is eliminated during reduction.
\end{command}

\begin{command}[\com{nopEq}{\comArg{EquationIndex}}]
declares equation with index \comArg{EquationIndex} as nonoperational. 
The predicate fails if equation \comArg{EquationIndex} does not exist 
or if the equation is trivial.
\end{command}

\begin{command}[\com{cp}{\comArg{RuleIndex1}\ad\comArg{RuleIndex2}}]
computes all critical pairs of rule \comArg{RuleIndex1} on rule 
\comArg{RuleIndex2}. 
The predicate fails, if no nontrivial critical pair can be found.
\end{command}

\begin{command}[\com{superpose}{\comArg{RuleIndex}\ad\comArg{NopEqIndex}\ad\comArg{Literal}\ad\comArg{LiteralSide}}]
superposes the left-side of the rule \comArg{RuleIndex} on the 
\comArg{LiteralSide}-side of the
literal \comArg{Literal} of the nonoperational equation with index 
\comArg{NopEqIndex}.
It fails if no nontrivial superpositions can be found, if
any of the two axioms can be reduced, or if
superpositions of the specified type need not be computed to achieve 
fairness.\\
To denote the \comArg{LiteralSide} \kw{left} and \kw{right} are used. 
\comArg{NopEqIndex} must be the index of a nonoperational equation.
Considering superposition with
\condEq{L_1\condAnd\ldots\condAnd L_n}{L}, 
we use \kw{conclusion} to denote $L$, and \com{condition}{\comArg{$i$}}
to denote $L_i$ in \comArg{Literal}.
\end{command}

\begin{command}[\com{superpose}{\kw{reflexivity}\ad\comArg{NopEqIndex}\ad
\comArg{Literal}\ad\_}]
superposes \rewRule{x = x}{true} on the literal \comArg{Literal} of the 
nonoperational equation with index \comArg{NopEqIndex}.
It fails if no nontrivial superpositions can be found, if
any of the two axioms can be reduced, or if
superpositions of the specified type need not be computed to achieve 
fairness.
\end{command}

\begin{command}[\com{redRule}{\comArg{RuleIndex}}]
reduces rule with index \comArg{RuleIndex}. The predicate fails if rule 
\comArg{RuleIndex} does not
exist or if the rule cannot be reduced or if the rule is eliminated during 
reduction.
\end{command}

\begin{command}[\com{redEq}{\comArg{EquationIndex}}]
reduces equation with index \comArg{EquationIndex}. 
The predicate fails if equation \comArg{EquationIndex} does
not exist or if the equation cannot be reduced or if the equation is eliminated
during reduction.
\end{command}
 
\begin{command}[\com{redNopEq}{\comArg{NopEqIndex}}]
reduces nonoperational equation with index \comArg{NopEqIndex}. 
The predicate fails if equation
\comArg{NopEqIndex} does not exist or if the equation cannot be reduced or if the 
equation is eliminated during reduction.
\end{command}

\noindent
The \comRef{repeat} predicate can be used to execute a predicate repeatedly 
until no more instances of it can be applied.\bigskip

\begin{command}[\com{repeat}{\comArg{Predicate}}]
causes repeated backtracking of \comArg{Predicate} until \comArg{Predicate} fails.
\end{command}

\noindent
An arbitrary interleaving of manual and automatic completion is supported. Also
completion --- manual or automatic --- can always safely be restarted after any
abortion caused by answering ``\cec{A.}'' to some query of the system.





  % 16.3.89
\section{Computing with Completed Specifications}

One may regard the completed specification as a first step towards an operational
implementation. 
Terms can be efficiently normalized, identities efficiently proved or disproved.
Moreover the solutions of equations can be found through narrowing.

\subsection{Normalization of Terms}
\label{NormCommand}

Computation in specifications is realized by term reduction with the
rules of the completed specification. The result of such computations are unique
normal forms. \bigskip

\begin{command}[\com{norm}{\comArg{Expression}}]
normalizes the input expression \comArg{Expression}. Three kinds of normalization are
provided:
\begin{itemize}
\item
if \comArg{Expression} is a signature term, the current rules are used to compute the
normalform of \comArg{Expression}.
\item
if \comArg{Expression} has the form \condRule{\comArg{conjunction}}{\comArg{term}} 
where \comArg{conjunction} is
a conjunction of equations, CEC will try to orient these equations into
rules before normalizing \comArg{Term} with the current rules and these oriented
condition equations.
\item
if \comArg{Expression} is a {\em let-expression}, the definitions (which
are equations between constructor terms and let-expressions) 
are evaluated first,
binding variables to the evaluated terms, before normalizing the body
of the let-expression using the current set of rules. This feature
should be used in the order-sorted case to avoid typing problems.
\end{itemize}
Let-expressions are given according to the following syntax:

\begin{syntax}
\nt{let-expression} \IS \kw{let} \nt{definitions} \kw{in} \nt{let-expression}\END
\\
\nt{definitions}    \IS \nt{definition} \{ \kw{and} \nt{definition} \} \END
\nt{definition}     \IS \nt{pattern} \kw{=} \nt{let-expression}
		    \OR \nt{pattern} \kw{=} \nt{signatureTerm} \END
\\
\nt{pattern}        \IS $<$ \rm well typed term with variables build up from
                    \GETON \hspace{2ex} constructors of the user-specified signature$>$
\end{syntax}

\noindent 
It is possible to trace the application of all rules or some selected rules:\medskip

\kw{trace:=on}

\noindent enables the trace mechanism.\medskip

\kw{rulesToTrace:=[}\comArg{RuleIndex1}\ad\ldots\ad\comArg{RuleIndexN}\kw{]}

\noindent restricts the trace mechanism to applications of the
rules \comArg{RuleIndex1},\ldots,\comArg{RuleIndexN}.
\end{command}

\begin{command}[\com{applyRule}{\comArg{Term}\ad\comArg{RuleIndex}\ad\comArg{ReducedTerm}}]
attempts to apply the rule with the given index once to the given term.
Different redexes are tried upon backtracking. If successful, the reduced
term is computed.
\end{command}

%\noindent
%Given the following confluent and Noetherian specification: \bigskip
%
%\begin{screen}
%| ?- {\bf show.}
%
%Current equations
%
%
%
%Current rules:
%
%  1  append([],l) -> l
%  2  append([e|l1], l2) -> [e|append(l1,l2)]
%  3  rev([]) -> []
%  4  rev([e|l]) -> append(rev(l),[e])
%
%Current nonoperational equations
%
%
%All axioms reduced.
%All critical pairs computed.
%All superpositions on nonoperational equations computed.
%No more equations, the system is complete.
%yes
%| ?-
%\end{screen}
%
%\noindent
%Then we can compute the reverse of a given list by computing it's normal form.
%
%\begin{screen}
%| ?- {\bf norm(rev([a, b, c, d]), NormalForm).}
%
%NormalForm = [ d, c, b, a]
%
%| ?-
%\end{screen}
%
%\noindent
%With applyRule we can apply one specific rule to reduce a term:
%
%\begin{screen}
%| ?- {\bf applyRule(rev([a, b, c, d]), 4, ReducedForm).}
%
%ReducedForm = append(rev([b, c, d]), [a])
%
%| ?- {\bf applyRule(rev([a, b, c, d]), 3, ReducedForm).}
%
%no
%
%| ?-
%\end{screen}

\noindent
To improve the computation of the normal form of a term, one can {\it compile}
the current set of rewrite rules into compiled Prolog and
use this compiled set of rewrite rules to normalize given terms.\bigskip

\begin{command}[\comName{compile}]
compiles the current set of rewrite rules into \comRef{compiled} Prolog.
The compiled rules are used when calling \comRef{eval}. 
Later changes to the set of rewrite rules
have no effect on \comRef{eval} unless a new call to \comRef{compile} is
performed. The predicate \comRef{norm} always uses the current set of
rewrite rules.
\end{command}

\begin{command}[\com{eval}{\comArg{Expression}}]
computes the normalform of a expression using the most recently compiled set of rewrite
rules. \comRef{eval} fails, if \comRef{compile} has not been called yet, cf. \comRef{compile}.
The trace mechanism applies also to \comRef{eval}.
\end{command}

\subsection{Proving Theorems in the Equational Theory}
\label{ProveCommand}

If general confluence can be achieved equational theorems become
decidable, e.g. it is decidable if two terms are equivalent with respect to the
equations in the specification: simply reduce the two terms to their unique normal
form and check if they are identical. This can be done using the
\comRef{prove}-command:\bigskip

\begin{command}[\com{prove}{\comArg{ConditionalEquation}}]
proves or disproves the conditional equation \comArg{ConditionalEquation} by 
rewriting the conclusion to normalforms, using the equations in the 
condition as additional rewrite rules. The method is incomplete for
nonempty conditions and/or noncanonical systems.
\end{command}

%In the following example a theorem of the equational theory is proved.
%
%\begin{screen}
%| ?- {\bf prove((append(rev([x, y]), [c, d]) = [y, x, c, d])).}
%
%Normal forms are : [y,x,c,d] and [y,x,c,d]
%yes
%| ?-
%\end{screen}


\subsection{Solving Equations in the Equational Theory}
\label{NarrowCommand}

Narrowing can be used to solve equations in an equational theory if a
canonical set of rewrite rules equivalent to the set of equations  exists.\bigskip

\begin{command}[\com{solve}{\comArg{Goal}\ad\comArg{Solution}}]
tries to solve \comArg{Goal} and to return an answer-substitution 
if successful.
The set of all answer-substitutions can be obtained by backtracking
(enter '\kw{;}'\nt{return} at the user level). Enter \nt{return} if
no more solutions are wanted.

\noindent
Goals are given using the following syntax:

\begin{syntax}
\nt{Goal} \IS \nt{condition} \nt{equation} 
          \OR \nt{equation} \{ \kw{and} \nt{equation} \} \END
\\
\nt{condition} \IS \nt{equation} \{ \kw{and} \nt{equation} \} \kw{=>} \END
\nt{equation}  \IS \nt{signatureTerm} \kw{=} \nt{signatureTerm} \END
\end{syntax}
\end{command}
   % 2.5.89

\section{Listing of all CEC commands}
\label{commands}

Remember that commands must be followed by a full stop `\user{.}'.\bigskip

\begin{command}[\comName{??}\nt{space}]
lists the available CEC commands and refers to the \comRef{?}-command for further
informations.
\end{command}

\begin{command}[\com{??}{\comArg{Keyword}}]
Lists only topics which contain \comArg{Keyword} as a substring.
\comArg{Keyword} must be a Prolog atom, e.g.\ if it contains any special
characters it must be enclosed in single quotes.
\end{command}

\begin{command}[\com{?}{\comArg{Function}}]
prints a short description for the CEC command \comArg{Function}. 
\comRef{?} is specified as prefix operator.
\end{command}

\begin{command}[\com{applyRule}{\comArg{Term}\ad\comArg{RuleIndex}\ad\comArg{ReducedTerm}}]
attempts to apply the rule with the given index once to the given term.
Different redexes are tried upon backtracking. If successful, the reduced
term is computed.
\end{command}

\begin{command}[\comName{c}]
calls the Knuth-Bendix completion procedure. This executes a fixed strategy
of applications of the ``completion inference'' predicates \comRef{orderEq}, 
\comRef{cp}, \comRef{superpose}, \comRef{redRule}, \comRef{redEq} 
and \comRef{redNOpEq}.
A manual guidance of the process is possible by explicitly calling these
predicates. The \comRef{repeat} predicate can be used to execute a predicate 
repeatedly until no more instances of it can be applied. An arbitrary
interleaving of manual and automatic completion is supported. Also,
completion --- manual or automatic --- can always safely be restarted after 
any abortion caused by answering ``\user{A.}'' to some query of the system.
The resulting system is not necessarily a reduced system. In the conditional
case it is anyway not clear what a reduced equation is. On the other hand,
a user may always call any of the \comRef{red}-predicates after completion
to force reduction of the axioms. This may, however, lead to an incomplete
system.
\end{command}

\begin{command}[cResume]
restarts the completion procedure after the completion process was 
aborted by answering ``\user{A.}'' to some query of the system.
\end{command}

\begin{command}[\com{cd}{\comArg{Path}}]
Changes, as the cd command in UNIX, the directory for all following
file-related CEC-commands. It is declared as prefix operator.\\
The path is given in form of a Prolog-atom, hence don't forget the
quotes, if the path contains `\kw{/}', `\kw{.}', and `\kw{..}' and 
other special characters.

\comRef{cd} does not implement file name generation using patterns.
Hence, `\kw{*}', `\kw{?}' and '\kw{[}' do not receive any special treatment.\\
Without an argument \comRef{cd} resets the current directory to the one in
which the CEC-system was invoked initially.
\end{command}


\begin{command}[\com{combineSpecs}{\comArg{StateName1}\ad\comArg{StateName2}\ad\comArg{CombinedSpec}}]
The specifications  stored (\refArrow \comRef{store}) 
in specification variables \comArg{StateName1} and \comArg{StateName2}
will  be combined, if possible, by forming
the union of the signature, axioms and pragmas.
If \comArg{CombinedSpec} $\neq$ \cec{user}, the result will be stored in
\comArg{CombinedSpec}, and the current specification will not be affected.
Otherwise, the combined specification becomes the new current
specification.
\end{command}

\begin{command}[\comName{compile}]
compiles the current set of rewrite rules into \comRef{compiled} Prolog.
The compiled rules are used when calling \comRef{eval}. 
Later changes to the set of rewrite rules
have no effect on \comRef{eval} unless a new call to \comRef{compile} is
performed. The predicate \comRef{norm} always uses the current set of
rewrite rules.
\end{command}
 
\begin{command}[\com{compileRules}{\comArg{File}}]
compiles current set of rewrite rules to Prolog and writes the Prolog clauses into
the file \comArg{File}\suffix{.rules}. \comArg{File}\suffix{.rules} may later be consulted or compiled 
(cf. \comRef{loadRules}) to produce a new definition of \comRef{eval}, cf. \comRef{compile}.
\end{command}
 
\begin{command}[\com{cp}{\comArg{RuleIndex1}\ad\comArg{RuleIndex2}}]
computes all critical pairs of rule \comArg{RuleIndex1} on rule \comArg{RuleIndex2}. 
The predicate fails, if no nontrivial critical pair can be found.
\end{command}

\begin{command}[\comName{constructor}]
asks the user to enter an operator and declares this operator to be
a constructor, if this is consistent with the current specification.
\end{command}

% \newpage

\begin{command}[\com{delete}{\comArg{ModuleName}\ad\comArg{OrderName}}]
deletes the specification variable named
\comArg{ModuleName}\kw{.}\comArg{OrderName}. If \comRef{delete} is used only
with argument \comRef{ModuleName}, the specification stored under this name is
deleted.
\end{command}

\begin{command}[\com{enrich}{\comArg{ModuleName}\ad\comArg{OrderName}}]
reads in additional parts of a specification from the files 
\comArg{ModuleName}\suffix{.eqn} and \comArg{ModuleName}\suffix{.}\comArg{OrderName}\suffix{.eqn}
after saving the current state for later checks for consistency of the enrichment. 
These additional parts must form an enrichment (cf. chapter~\ref{enrichment}).
\comArg{ModuleName} and \comArg{OrderName} can be arbitrary Prolog atoms.
Leaving out \comArg{OrderName} or even both arguments yields the same
effect as for \comRef{in}. 
%Specify \comArg{ModuleName} = \kw{user} if input from terminal is wanted.
\end{command}
 

\begin{command}[\comName{equal}]
asks the user to enter a list of the following form: 
\begin{center}
\kw{[}\kw{[}\comArg{a}\kw{,}\comArg{b}\kw{,}\comArg{c}\kw{,} \ldots \kw{],}
\kw{[}\comArg{g}\kw{,}\comArg{h}\kw{,}\comArg{i}\kw{,} \ldots \kw{],} \ldots \kw{]}
\end{center}
and declares operators to have equivalent precedences 
--- only allowed for \kw{kns}.\\
Meaning: \comArg{a} = \comArg{b} = \comArg{c} = \ldots and \comArg{g} = \comArg{h} = \comArg{i} = \ldots.
\end{command}


\begin{command}[\com{eval}{\comArg{Expression}}]
computes the normalform of a expression using the most recently compiled set of rewrite
rules. \comRef{eval} fails, if \comRef{compile} has not been called yet, cf. \comRef{compile}.
The trace mechanism applies also to \comRef{eval}.
\end{command}

\begin{command}[\comName{forget}]
forgets the complete undo history.
\end{command}


\begin{command}[\com{freeze}{\comArg{ModuleName}\ad\comArg{OrderName}}]
writes the state of the current specification to the file
\comArg{ModuleName}.\comArg{OrderName}\suffix{.q2.0} and
updates the content of the specification variable
\comArg{ModuleName}.\comArg{OrderName}.
If freeze is used without \comArg{OrderName} the state is just written
to \comArg{ModuleName}\suffix{.q2.0}. 
If freeze is used without any argument the module name of the current 
specification is used for \comArg{ModuleName} and the current order name
is used for \comArg{OrderName}. In the last two cases the specification 
variable determined by the current module name and the current order name
will be updated (If current order name is \kw{noorder} the specification
variable associated with the current module name and the current termination
ordering will be updated too).
The specification may later be reused by thawing it from this file, cf. the 
\comRef{thaw}-command.
The state of CEC remains unchanged by this operation.
\end{command}

 
\begin{command}[\comName{greater}]
asks the user to enter a list of the following form: 
\begin{center}
\kw{[}\kw{[}\comArg{a}\kw{,}\comArg{b}\kw{,}\comArg{c}\kw{,} \ldots \kw{],}
\kw{[}\comArg{g}\kw{,}\comArg{h}\kw{,}\comArg{i}\kw{,} \ldots \kw{],} \ldots \kw{]}
\end{center}
and adds ordered pairs of operators to the precedence.\\
Meaning: \comArg{a} $>$ \comArg{b} $>$ \comArg{c} $>$ \ldots and 
\comArg{g} $>$ \comArg{h} $>$ \comArg{i} $>$ \ldots
\end{command}


\begin{command}[\comName{halt}]
ends a CEC session.
\end{command}


\begin{command}[\com{in}{\comArg{ModuleName}\ad\comArg{OrderName}}]
reads in a specification from the file \comArg{ModuleName}\suffix{.eqn}
and the associated order specification from the file
\comArg{ModuleName}\suffix{.}\comArg{OrderName}\suffix{.ord}.
As log-files are ``enriched'' order specifications, any log-file can
be used as an order file.
If not stated otherwise these files are assumed to be in the current 
directory. Before the specification is read in, CEC will be re-initialized, 
e.g. the current specification will be deleted. Specifications saved in 
variables will not be affected. \comArg{ModuleName} and \comArg{OrderName}
must be Prolog atoms. \comArg{OrderName} becomes the
current order name for the specification.\\
But \com{in}{\comArg{ModuleName}\ad\cec{noorder}}
has the effect that no order specification is consulted.
The termination ordering for the new
specification will be initialized to a default value 
(\kw{neqkns} or \kw{poly1}, depending on the presence of AC-operators).
Using \comRef{in} only with the parameter \comArg{ModuleName} yields the same
effect. \comArg{ModuleName} = \kw{user} expects input from terminal.
(For Quintus-Prolog2.x under EMACS: \comRef{in} without
parameter reads from \cec{Scratch.pl}).
\end{command}


\begin{command}[\comName{interpretation}]
displays all operator interpretations, provided \kw{poly}\nt{N} is the chosen termination
ordering and asks the user if he wants to change any. If so all rules will be turned 
back into equations.
\end{command}


\begin{command}[\com{load}{\comArg{ModuleName}\ad\comArg{OrderName}}]
loads the system which is currently the value of the variable
\comArg{ModuleName}\kw{.}\comArg{OrderName}, cf. the \comRef{store}-command. 
If \comRef{load} is used only with argument \comArg{ModuleName}, this actual parameter 
completely specifies the name of the variable.
Specification variables remain unchanged.
\comArg{StateName} = \cec{'$initial'} re-initializes the system.
\end{command}


\begin{command}[\com{loadLog}{\comArg{ModuleName}\ad\comArg{OrderName}}]
reads in the file \comArg{ModuleName}\suffix{.}\comArg{OrderName}\suffix{.@.ord}. 
%This file contains all the answer given during
%the completion process and the final termination ordering saved using
%the \comRef{saveLog}-command. 
If the completion process is started again, % all
questions whose answers are already contained in 
\comArg{ModuleName}\suffix{.}\comArg{OrderName}\suffix{.@.ord} 
will be suppressed. 
If \comRef{loadLog} is used without the argument \comArg{OrderName} the 
information will be taken from the file \comArg{ModuleName}\suffix{.@.ord}.
If \comRef{loadLog} is used without any argument
the name of the current specification together with the current order name
will be used.
%The current order name is the
%order name of the order specification for the current specification
%or the current termination ordering if no order specification was used.
\end{command}

\begin{command}[\com{loadOrder}{\comArg{ModuleName}\ad\comArg{OrderName}}]
reads in the file \comArg{ModuleName}\suffix{.}\comArg{OrderName}\suffix{.ord}. 
This file should contain informations concerning the termination ordering.
If \comRef{loadLog} is used without the argument \comArg{OrderName} the 
information will be taken from the file \comArg{ModuleName}\suffix{.ord}.
If \comRef{loadLog} is used without any argument
the name of the current specification together with the current order name
will be used.
\end{command}

\begin{command}[\com{loadRules}{\comArg{File}}]
compiles the set of rewrite rules which has been stored previously in the file
\comArg{File}\suffix{.rules}, cf. \comRef{compileRules}.
\end{command}

\begin{command}[\comName{moduleName}]
displays the name of the current specification.
\end{command}


\begin{command}[\com{nopEq}{\comArg{EquationIndex}}]
declares equation with index \comArg{EquationIndex} as nonoperational. 
The predicate fails if equation \comArg{EquationIndex} does not exist 
or if the equation is trivial.
\end{command}

\begin{command}[\com{norm}{\comArg{Expression}}]
normalizes the input expression \comArg{Expression}.
See also \comRef{eval}
\end{command}


\begin{command}[\comName{operators}]
displays all precedences and stati in \kw{kns} or \kw{neqkns} or
all polynomial interpretations in \kw{poly}\nt{N} respectively.
\end{command}

\begin{command}[\comName{order}]
indicates the current termination ordering and asks the user whether he wants to 
change it.
If a new ordering is selected and if the previous termination
ordering is incompatible with the new ordering, all rules are turned back into
equations and the completion must be repeated from the beginning.
\end{command}

\begin{command}[\com{orderEq}{\comArg{EquationIndex}}]
orients equation with index \comArg{EquationIndex}. The predicate fails if equation 
\comArg{EquationIndex} does
not exist or if the equation cannot be oriented or turned into a nonoperational
equation or if the equation is eliminated during reduction.
\end{command}

\begin{command}[\comName{orderName}]
displays the name of the order specification associated with
the current specification.
\end{command}

\begin{command}[\com{polGreater}{\comArg{Interpretation1}\ad\comArg{Interpretation2}}]
Only useful with ordering \kw{poly}\nt{N}.
If it succeeds, \comArg{Interpretation1} $>$ \comArg{Interpretation2} holds true
(if $>$ is the ordering on tuples of polynomials).
Interpretations (i.e. tupels of polynomials) of terms can be generated
via \comRef{polynomial}.
\end{command}

\begin{command}[\com{polynomial}{\comArg{Term}\ad\comArg{Interpretation}}]
yields the polynomial interpretation of \comArg{Term}. It
fails, if the ordering is not \kw{poly}\nt{N}. 
If there are operators in \comArg{Term},
for which no polynomial interpretation is known, the user is asked for
such an interpretation (and the given interpretation is stored).
\end{command}


\begin{command}[\comName{preregular}]
succeeds if the current (order-sorted) signature is preregular,
and fails otherwise.
The preregularity condition is the regularity of 
Smolka/Nutt/Goguen/Meseguer 87 (\refArrow \cite{SNGM87}).
\end{command}


\begin{command}[\com{prove}{\comArg{ConditionalEquation}}]
proves or disproves the conditional equation \comArg{ConditionalEquation} by 
rewriting the conclusion to normalforms, using the equations in the 
condition as additional rewrite rules. The method is incomplete for
nonempty conditions and/or noncanonical systems.
\end{command}

\begin{command}[\comName{pwd}]
prints out the current path.
\end{command}

\begin{command}[\com{redEq}{\comArg{EquationIndex}}]
reduces equation with index \comArg{EquationIndex}. 
The predicate fails if equation \comArg{EquationIndex} does
not exist or if the equation cannot be reduced or if the equation is eliminated
during reduction.
\end{command}

\begin{command}[\com{redNopEq}{\comArg{NopEqIndex}}]
reduces nonoperational equation with index \comArg{NopEqIndex}. 
The predicate fails if equation
\comArg{NopEqIndex} does not exist or if the equation cannot be reduced or if the 
equation is eliminated during reduction.
\end{command}
 
\begin{command}[\com{redRule}{\comArg{RuleIndex}}]
reduces rule with index \comArg{RuleIndex}. The predicate fails if rule 
\comArg{RuleIndex} does not
exist or if the rule cannot be reduced or if the rule is eliminated during 
reduction.
\end{command}

\begin{command}[\comName{regular}]
succeeds if the current (order-sorted) signature is regular,
and fails otherwise.
The regularity condition is the one of Goguen/Meseguer 87
(\refArrow \cite{GM87}).
\end{command}

\begin{command}[\com{renameSpec}{\kw{[}\comArg{OldSort1} \kw{<-} \comArg{NewSort1}, \ldots , 
\comArg{OldSortN} \kw{<-} \comArg{NewSortN},\\ \hspace*{6.2em}
\comArg{OldOperator1} \kw{<-} \comArg{NewOperator1}, \ldots , 
\comArg{OldOperatorM} \kw{<-} \comArg{NewOperatorM}\/\kw{]}}]
renames the current specification according to the given lists of sort associations
and operator associations. Only injective renamings of operators are allowed.
Sorts may be renamed arbitrarily. Sorts and operators which are not mentioned remain 
unchanged.
\end{command}

\begin{command}[\com{repeat}{\comArg{Predicate}}]
causes repeated backtracking of \comArg{Predicate} until \comArg{Predicate} fails.
\end{command}
 
\begin{command}[\com{resetOrient}{\comArg{Index}}]
turns the rule Index back into an equation. If \comRef{resetOrient}
is used without an argument, all rules are turned back into equations.
\end{command}


\begin{command}[\com{restoreCEC}{\comArg{FileName}}]
restores the CEC state (prolog state) in \comArg{FileName}. 
\end{command}

\begin{command}[\com{saveCEC}{\comArg{FileName}}]
saves the whole CEC state (prolog state) in \comArg{FileName}. The CEC state
can be used by simply invoking \comArg{FileName} instead of CEC or 
using the \comRef{restoreCEC}-command.
If \comRef{saveCEC} is used without argument, the current CEC-name,
i.e. the actual parameter for the last use of the \comRef{saveCEC}-command,
is used for \comArg{FileName}. If no current CEC-name is known, the name
\kw{cec} will be taken (so be careful).
\end{command}

\begin{command}[\comName{setInterpretation}]
asks the user to enter a list of the following form:
\begin{center}
\kw{[}\comArg{Operator(Arguments)} : \comArg{Interpretation}\kw{,} \ldots\kw{]}
\end{center}
Provided \kw{poly}\nt{N} is the current termination ordering a new interpretation
\comArg{Interpretation} for an operator \comArg{Operator} is added to the current state.
The interpretation
may be a polynomial over the variables in \comArg{Arguments} (if
N = 1) or a list with N polynomials.
The new interpretation must be compatible
with all C- or AC-declarations in the current state.
\end{command}

 
\begin{command}[\comName{show}]
shows the sets of equations, rules and nonoperational equations of the 
current specification in order-sorted notation.
(Usually there exists more than one many-sorted representation of
an order-sorted axiom.)

Rules \condRule{C}{\rewRule{l}{r}} which are marked by an asterix 
\kw{*} have an associated auxiliary rule of form
\condRule{C}{\rewRule{l+X}{r+X}}
where $+$ is the AC-operator on top of $l$ and $X$ is a new
variable of appropriate sort.
Auxiliary rules are automatically generated when needed during
completion modulo AC.
\end{command}
 
\begin{command}[\com{show}{\comArg{SpecificationVariable}}]
shows the sets of equations, rules and nonoperational equations 
of the specification which is stored in the variable 
\comArg{SpecificationVariable}.
\end{command}

 
\begin{command}[\comName{showCStatus}]
displays the current completion status.
\end{command}

\begin{command}[\comName{showms}]
shows the set of equations, rules and nonoperational equations of the 
current specification in many-sorted notation.
\end{command}
 
\begin{command}[\comName{sig}]
displays the signature of the current specification.
\end{command}

\begin{command}[\com{solve}{\comArg{Goal}\ad\comArg{Solution}}]
tries to solve \comArg{Goal} and to return an answer-substitution if
successful. The set of all answer-substitutions can be obtained by
backtracking (enter `\user{;}'\nt{return} at the user level). Enter
\nt{return}, if no more solutions are wanted.
\end{command}

\begin{command}[\com{sp}{\comArg{Index}\ad\comArg{NopEqIndex}}]
same as \com{superpose}{\comArg{Index}\ad\comArg{NopEqIndex}\ad\kw{left}\ad\kw{condition(1)}}.
\end{command}

\begin{command}[\comName{specifications}]
lists the module and order names of all specifications that are currently
saved in variables.
\end{command}

\begin{command}[\comName{status}]
asks the user to enter a list of the following form:
\begin{center}
\kw{[}\comArg{Operator} : \comArg{Status}\kw{,} \ldots \kw{]}
\end{center}
and declares that the operators should have the desired stati provided \kw{kns}
or \kw{neqkns}
is the type of the current termination ordering. \comArg{Status} can be
\kw{lr} for {\em left-to-right},
\kw{rl} for {\em right-to-left} or
\kw{ms} for {\em multiset}.
\end{command}

\begin{command}[\com{store}{\comArg{ModuleName}\ad\comArg{OrderName}}]
saves the current specification in a specification variable named
\comArg{ModuleName}\kw{.}\comArg{OrderName}. If \comRef{store} is used only with
argument \comArg{ModuleName} the specification is saved in a variable with this name,
if \comRef{store} is used without any argument, the name is created as 
\nt{moduleName}\kw{.}\nt{orderName}, with names as they are currently associated 
with the specification. For later restoring 
use the command \comArg{load}. The system remains unchanged except for this variable 
containing afterwards the current specification.
\end{command}

\begin{command}[\com{storeLog}{\comArg{ModuleName}\ad\comArg{OrderName}}]
creates the log-file.
The name of the log-file is 
\comArg{ModuleName}\suffix{.}\comArg{OrderName}\suffix{.@.ord}.
It has the format of an order specification file which can be used
with the \comRef{in}-command or the \comRef{loadLog}-command. 
If \comRef{storeLog} is used only with argument \comArg{ModuleName}
the log-file is named \comArg{ModuleName}\suffix{.@.ord},
if \comRef{storeLog} is used without any argument, the file is
created as \nt{moduleName}\suffix{.}\nt{orderName}\suffix{.@.ord}.
\end{command}

\begin{command}[\com{storeOrder}{\comArg{ModuleName}\ad\comArg{OrderName}}]
creates a file named
\comArg{ModuleName}\suffix{.}\comArg{OrderName}\suffix{.ord}.
It has the format of an order specification file which can be used
with the \comRef{in}-command or the \comRef{loadOrder}-command. 
If \comRef{storeOrder} is used only with argument \comArg{ModuleName},
the name \comArg{ModuleName}\suffix{.ord} is used.
If the command \comRef{storeOrder} is used without any argument, the file created
is named \nt{moduleName}\suffix{.}\nt{orderName}\suffix{.ord}.
\end{command}

\begin{command}[\com{superpose}{\comArg{RuleIndex}\ad\comArg{NopEqIndex}\ad\comArg{Literal}\ad\comArg{LiteralSide}}]
superposes the left-side of the rule \comArg{RuleIndex} on the 
\comArg{LiteralSide}-side of the
literal \comArg{Literal} of the nonoperational equation with index 
\comArg{NopEqIndex}.
It fails if no nontrivial superpositions can be found, if
any of the two axioms can be reduced, or if
superpositions of the specified type need not be computed to achieve 
fairness.\\
To denote the \comArg{LiteralSide} \kw{left} and \kw{right} are used. 
\comArg{NopEqIndex} must be the index of a nonoperational equation.
Considering superposition with
\condEq{L_1\condAnd\ldots\condAnd L_n}{L}, 
we use \kw{conclusion} to denote $L$, and \com{condition}{\comArg{$i$}}
to denote $L_i$ in \comArg{Literal}.
\end{command}

\begin{command}[\com{superpose}{\kw{reflexivity}\ad\comArg{NopEqIndex}\ad
\comArg{Literal}\ad\_}]
superposes \rewRule{x = x}{true} on the literal \comArg{Literal} of the 
nonoperational equation with index \comArg{NopEqIndex}.
It fails if no nontrivial superpositions can be found, if
any of the two axioms can be reduced, or if
superpositions of the specified type need not be computed to achieve 
fairness.
\end{command}

\begin{command}[\com{'superpose!'}{\comArg{RuleIndex}\ad\comArg{NopEqIndex}\ad\comArg{Literal}\ad\comArg{LiteralSide}}]
same effect as \comRef{superpose}, except that the specified superposition
will be performed in any case, even if not necessary for fairness of
completion.
\end{command}

\begin{command}[\com{thaw}{\comArg{ModuleName}\ad\comArg{OrderName}\ad\comArg{SpecificationVariable}}]
This command is the inverse operation of \comRef{freeze} and restores the specification
previously frozen in \comArg{ModuleName}\suffix{.}\comArg{OrderName}\suffix{.q2.0}
under the name \comArg{SpecificationVariable}. The current specification and other variables 
will be not affected by this operation.
If \comRef{thaw} is used without argument \comArg{StateName} 
the current specification is overwritten by the thawed specification. 
Specifications saved in variables will still not be affected by this operation.
If \comRef{thaw} is used only with argument \comArg{ModuleName} the frozen 
specification will be taken from the file \comArg{ModuleName}\suffix{.q2.0}.
\end{command}

\begin{command}[\comName{undo}]
can be entered at the system's top level to set the system to the state
before the last command that has caused a state change, if there was
any. \comRef{undo} can be used repeatedly to undo several steps of state changes.
It also undoes \comRef{undoUndo}-calls. At the moment, there is no way to backtrack
from single decisions that have been taken while running the completion
process.
\end{command}

\begin{command}[\comName{undoUndo}]
allows to undo the last \comRef{undo}-command at the system's top level. Repeated use of
this command undoes sequences of \comRef{undo}-commands. Chains of undo-calls begin at 
the last user interaction different from an \comRef{undo} or \comRef{undoUndo}.
\end{command}

\noindent 
The CEC-commands \comRef{equal}, \comRef{greater}, \comRef{status}, 
\comRef{setInterpretation} and \comRef{constructor} can be used as 
order pragmas, cf. chapter \ref{OrderPragmas}.
 % 15.3.89


\newpage

\bibliography{lit}
\bibliographystyle{mhshort}
\newpage

\appendix
\section{PREDEFINED OPERATORS OF CEC}
\label{PredefinedOperators}
\begin{center}
\begin{tabular}{|lrrcc|} \hline
CEC operators & priority & fix & reserved & changable \\ \hline
\cec{=>} & 950 & xfx & + & - \\
\cec{=} & 700 & xfx & + & - \\
\cec{:} & 600 & xfy & + & - \\
\cec{and} & 850 & fy & + & - \\
\cec{@} & 50 & fx & + & - \\ \hline
\cec{->} & 1050 & xfy & - & - \\ 
\cec{cons}, \cec{op} & 950 & fx & - & - \\
\cec{in} & 935 & xfy & - & - \\
\cec{let} & 910 & fy & - & - \\
\cec{var} & 1100 & fx & - & - \\
\cec{<} & 700 & xfx & - & - \\
\cec{using} & 600 & xfx & - & - \\
\cec{module}, \cec{order} & 500 & fx & - & - \\
\cec{+} & 500 & yfx & - & - \\
\cec{for} & 400 & xfx & - & - \\
\cec{*} & 400 & yfx & - & - \\ \hline
\cec{:-}, \cec{-->} & 1200 & xfx & - & + \\
\cec{:-}, \cec{?-}  & 1200 & fx  & - & + \\
\cec{public}, \cec{multifile}, \cec{mode}, \cec{meta_predicate}, \cec{dynamic} & 1150 & fx & - & + \\
\cec{;}             & 1100 & xfy & - & + \\
\cec{,}             & 1000 & xfy & - & + \\
\cec{spy}, \cec{nospy}, \cec{\+} & 900 & fy & - & + \\
\cec{not} & 900 & fx & - & + \\
\cec{is}, \cec{=..}, \cec{==}, \cec{\==}, 
\cec{@<}, \cec{@>}, \cec{@=<}, \cec{@>=}, \cec{=:=}, \cec{=\=},
\cec{>}, \cec{=<}, \cec{>=} & 700 & xfx & - & + \\
\cec{-}, \cec{/\}, \cec{\/} & 500 & yfx & - & + \\
\cec{|} & 500 & xfy & - & + \\
\cec{-} & 500 & fx & - & + \\
\cec{/}, \cec{//}, \cec{<<}, \cec{>>} & 400 & yfx & - & + \\
\cec{mod} & 300 & xfx & - & + \\
\cec{^} & 200 & xfy & - & + \\ 
\cec{?} & 200 & fx & - & + \\
\hline
\end{tabular}
\end{center}

If an operator is {\em reserved}, it cannot me used in any signature. If 
an operator is not {\em changable}, it can be used in a signature, but 
its priority and fix may not be changed. 

\section{INSTALLATION OF CEC}
The CEC distribution tape contains one tar file, which
includes the CEC-System. The size of the CEC sources is about 763 kbytes, the whole system needs
about  7657 kbytes.
To install CEC execute the following steps:\bigskip

\noindent
Create a directory ``\cec{cec}'' where you want to file in CEC, 
and use the tar-command to read the tar file.

\noindent
Then change to the directory ``\cec{cec/obj/cec}'' and correct
the value of environment variables\bigskip

\cec{PROSPECTRA}

\cec{QPROLOG}\bigskip

\noindent
in the file ``\cec{Makefile}'' before calling\bigskip

{\bf make cec}\bigskip

\noindent
After successful creation of a executable \cec{cec} call\bigskip

{\bf make clean}

{\bf make install}\bigskip

\noindent
to complete the installation of CEC.\bigskip

\noindent
The executable file is located in the directory ``\cec{cec/bin}'' and
you can work with CEC as specified in the user manual, e.g. \bigskip

\verb+| ?-+ {\bf cd 'demo/cec/math'.}

\verb+| ?-+ {\bf in(abelianGroup,poly1).}

\verb+| ?-+ {\bf c.}

\end{document}
