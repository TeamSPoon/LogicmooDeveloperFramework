\section{An Example Session}
\label{exampleSession}

To provide a first impression of  the capabilities of the CEC-System, we
describe the development of a quicksort algorithm on lists of
natural numbers.

\subsection{The Specification of Quicksort}

Specifications can be written and completed in a {\em modular}
way. CEC can combine completed specifications without 
repeating previous work. The hierarchy of modules for the
specification of quicksort on lists of natural numbers is given
in the following diagram:

\begin{picture}(450,100)
\put(50,0){\makebox(400,100){}}
\put(205,85){\makebox(0,0){{\tt qsortnat}}}
\put(200,80){\vector(-2,-1){45}}
\put(200,80){\vector(2,-1){45}}

\put(150,50){\makebox(0,0){{\tt nat}}}

\put(257,50){\makebox(0,0){{\tt qsort}}}
\put(255,45){\vector(-3,-2){30}}
\put(255,45){\vector(3,-2){30}}

\put(215,20){\makebox(0,0){{\tt totalOrder}}}

\put(300,20){\makebox(0,0){{\tt lists}}}
\end{picture}

\noindent
The module \cec{qsort} --- saved in a file named \cec{qsort.eqn} ---
describes the quicksort algorithm:
\begin{spec}
module qsort using lists + totalOrder.

op sort  : (list -> list).
op split : (elem * list -> pair).
cons (',') : (list * list -> pair).

sort([]) = [].
split(x, l) = (l1 , l2) => sort([x|l]) = append(sort(l1), [x|sort(l2)]).

split(x, []) = ([] , []).
(y =< x) = true and split(x, l) = (l1,l2) => split(x, [y|l]) = ([y|l1],l2).
(y =< x) = false and split(x, l) = (l1,l2) => split(x, [y|l]) = (l1,[y|l2]).
\end{spec}

Complex specifications can be constructed from modules by the 
elementary operations {\em combination}, {\em enrichment} and {\em renaming}
(\refArrow chapter \ref{OperationsOnSpecifications}).
The combine operator (\cec{+}) forms the union of two specifications and
the rename operator (\cec{<-}) allows to renaming sorts and operators.
Any specification in CEC is the enrichment of a possibly empty base
specification (``{\tt using} $<$base$>$'') by new vocabulary and axioms.
The module \cec{qsort} is based on \cec{lists} 
and \cec{totalOrder}. Here \cec{lists} is the imported module
\begin{spec}
module lists.

cons []   : list .
cons '.'  : (elem * list -> list).
op append : (list * list -> list).

append([], l) = l.
append([e | l1], l2) = [e | append(l1, l2)].
\end{spec}

We use the constructor \cec{[]} to denote the empty list and 
the constructor `\cec{.}' for adding an element to a list
(This is exactly the way how lists are constructed in Prolog and we can
use the usual Prolog notation for lists in which \cec{[e|l]} is
a synonym for \cec{(e '.' l)}). 
Additionally we have defined an 
operator \cec{append} for the concatenation of two lists and described 
its behaviour through two equations.

\cec{totalOrder} plays the r$\hat{o}$le of a formal parameter
of \cec{qsort}:
\begin{spec}
module totalOrder.

op (=<) : (elem * elem -> bool) .

(x =< x) = true .
(x =< y) = false => (y =< x) = true .
(x =< y) = true and (y =< z) = true => (x =< z) = true .
(x =< y) = true and (y =< x) = true => x = y .
\end{spec}

It describes a usable approximation of the usual first-order axioms
for total orders in a Horn clause setting.
An actual parameter candidate for \cec{totalOrder} is
the following specification of natural numbers:
\begin{spec}
module nat using totalOrder(elem <- nat).

cons 0 : nat.
cons(s, 100, fy) :  (nat -> nat).

(0 =< n ) = true.
(s n =< 0) = false .
(s n =< s m) = (n =< m).
\end{spec}

Computing with the above specification one has to write the numbers
$n$ as \cec{s}$^n$\cec{0} which is a tedious work. In CEC it is possible to provide different external
representations for the terms of a specification by specifying
the translations from the external to the internal
and from the internal to the external representation 
(\refArrow chapter \ref{ParseAndPretty}).

The instantiation of the formal parameter of \cec{qsort} with
nat yields \cec{qsortnat}:
\begin{spec}
module qsortnat using qsort(elem <- nat) + nat.
\end{spec}

The modules \cec{qsort} and \cec{nat} are combined after
renaming the sort \cec{elem} of \cec{qsort} into \cec{nat}.
The completion of \cec{qsortnat} will check the consistency of the axioms 
for \cec{=<} in the actual parameter \cec{nat} and in the formal
parameter \cec{totalOrder}. The latter proves the correctness
of the actual parameter in cases where the actual parameter is
constructor-complete. \cec{nat} is constructor-complete.

\subsection{Order Specifications}

In CEC the following termination orderings are available:
\begin{itemize}
\item
Two  {\em precedence orderings} \kw{kns} and \kw{neqkns}, according to 
Kapur et. al \cite{KNS85} (\refArrow chapter \ref{kns}).
%They are based on the recursive comparison of paths occurring
%in the terms to be compared. This order is induced by a partial
%order on operators called the {\it precedence}. The precedence ordering
%\kw{neqkns} forbids that two different operators have the same precedence.
%All constructors are given a precedence less than any nonconstructor operator.
\item
The method of {\it polynomial interpretations}
where \[t_1 > t_2 : \iff I(t_1) > I(t_2). \]
if $I(t)$ is the polynomial or tuple of polynomials associated with
it.
The concrete version of this technique as it is used in CEC is due to 
\cite{CL86} (\refArrow chapter \ref{poly}).
\kw{poly}\nt{N} stands for polynomial interpretations with 
tuplelength $N$.
\end{itemize}
The precedence declarations or polynomial interpretations
for the operators of a specification are given interactively
during the completion process or in a {\em order specification}
associated with the specification
(\refArrow chapter \ref{OrderSpecification}).
%The order specification determines the termination ordering to be used for 
%\nt{specificationName}, the order names for its direct imports
%and gives precedence declarations for the operators of the
%specification (if the termination ordering is \kw{kns} or \kw{neqkns}) or
%polynomial interpretations (if the termination ordering is \kw{poly}\nt{N}).

For example, the order specification for the module \cec{lists}
may have the following form:

\begin{spec}
order poly3 for lists.

setInterpretation(['[]'        : [2, 2, 2],
                   '.'(x,y)    : [3 * x + y + 1, x + y, x + y],
                   append(x,y) : [x + y, x + y, 2 * x + y]]).
\end{spec}

In this case, polynomial interpretations with tuple length 3
have been chosen. If this information is stored in the file \cec{lists.pol.ord},
\cec{pol} is called the {\em order name} of this order specification.

\subsection{Reading and Displaying}

We want to demonstrate how CEC handles this specification. 
First, we read in the specification of \cec{lists} using
the \comRef{in}-command (\refArrow chapter \ref{InCommand})
with two arguments, the name of the module and the order name
(\refArrow chapter \ref{orderBase}):

\begin{screen}
| ?- in(lists,pol).
[collecting garbage...]
[evaluating base of lists.eqn with lists.pol.ord...]
 [thawing standard.poly3.q2.0 into user...]
 [storing to standard.poly3]
[reading body of lists.eqn and lists.pol.ord...]
[analyzing axioms...]
[collecting garbage...]

Time used: 3.634 sec.
\end{screen}

\noindent
We can use the \comRef{sig}-command to display the signature:

\begin{screen}
| ?- sig.

Signature :

cons true       : bool.
cons false      : bool.
cons []	: list.
cons .  : (elem * list -> list).
op append       : (list * list -> list).
\end{screen}

The boolean constants \cec{true} and \cec{false} are imported
from the \cec{standard}-module, which is automatically imported into 
every specification. The fact that \cec{true} and \cec{false}
are constructors implies that any equation, explicitly given in
a specification or derived through completion, equating these
constants is forbidden.
As expected, we also see our two constructors \cec{[]} and
`\cec{.}' and the operator \cec{append}.

The \comRef{show}-command can be used to display the 
set of equations and rules (\refArrow chapter \ref{Displaying}):

\begin{screen}
| ?- show.

Current equations

  1    append([],l) = l
  2    append([e|l1],l2) = [e|append(l1,l2)]

Current rules

Current nonoperational equations

All axioms reduced.
All superpositions computed.
The set of equations is not empty.
\end{screen}

\noindent
The information about the polynomial interpretations can be made
visible using the \comRef{interpretation}-command. If \cec{kns}
or \cec{neqkns} is used, this can be achieved by the
\comRef{operators}-command (\refArrow chapter \ref{InspecTermination}).

Every consulted specification is saved in a special {\em specification
variable}. The user can restore an older state of his system
using the \comRef{load}-command (\refArrow chapter \ref{LoadCommand})
or can update the content of the variable using the \comRef{store}-command
(\refArrow chapter \ref{StoreCommand}). 
Whenever a specification is to be imported, the CEC system uses the content
of the corresponding specification variable, if this specification variable 
exists. So it increases one's speed, if the completion process of a hierarchical
specification follows the tree structure from its leaves to its root, 
storing every participating specification after successful completion.
We assume that attention is paid to this advice for our example.
Information about the current specification variables
can be retrieved using the \comRef{specifications}-command.

It is also possible to save a system externally using the 
\comRef{freeze}-command (\refArrow chapter \ref{FreezeCommand}).
To restore a externally saved system use the \comRef{thaw}-command
(\refArrow chapter \ref{ThawCommand}).

\subsection{Completion}
We now apply the completion procedure (\refArrow chapter
\ref{Completion}) to our example.

\subsubsection{Completing {\tt lists}}

The completion process --- started with the \comRef{c}-command ---
is able to orient the two equations of \cec{lists}
without requesting any additional information:

\begin{screen}
| ?- c.

new rule   1    append([],l) = l .

new rule   2    append([e|l1],l2) = [e|append(l1,l2)] .

[4 superpositions yet to be considered.]

0 superpositions have been computed.
Time used: 0.883011 sec.
\end{screen}

\noindent
The system is complete, and has the following rules, displayed 
by the \comRef{show}-command:

\begin{screen}
| ?- show.

Current equations

Current rules

  1    append([],l) = l
  2    append([e|l1],l2) = [e|append(l1,l2)]

Current nonoperational equations

All axioms reduced.
All superpositions computed.
No more equations, the system is complete.
\end{screen}


\subsubsection{Completing {\tt totalOrder}}

An important feature is that CEC does not require to transform all
equations into rewrite rules. This can be demonstrated during the
completion of \cec{totalOrder}.

After orienting the first equation into a rewrite rule, 
the system discovers that it is unable to orient the second 
equation:

\begin{screen}
Checking reductivity constraints for rule
        x=<y = false => y=<x = true:
The current ordering fails to prove
[y=<x]  >  [x=<y,false].
At this point you may take any of the the following actions:
a. for assume to be proved
c. for checking quasi-reductivity of the equation
p. for postpone
n. for considering the equation as nonoperational
   Please answer with a. or c. or p. or n. (Type A. to abort) > \user{n.}
\end{screen}

\noindent
We want to consider this equation as nonoperational
(\refArrow chapter \ref{Completion}), so we answer with \user{n.}.

Nonoperational equations become useless for the equational theory and, hence,
in fact nonoperational, if they are superposed on at least one of their
conditions by all rewrite rules. This yields new conditional equations. 
What concerns the equational theory, the new equations together have 
the same power as the original equation. 
However, they may have better operational properties than the
equation they have been generated from.
Equations that cannot be oriented into a reductive rule must
either be eliminated eventually or considered nonoperational.

Checking the convergence of conditional equations requires to compare different
applications of equations and rewrite rules. The comparison of equation application is performed
by comparing the literals of the equation instance. The {\em status} of an equation 
determines the order in which the literals of the equation should be
inspected. The status $ms$ means that the literals are compared as a multiset.
Instead of $ms$ the user can choose an arbitrary sequence of the literals
by entering a permutation of [0 .. n] where n is the number of conditions
(\refArrow \cite{Gan88a}). 

We want to use $ms$ here:

\begin{screen}
In which order should the literals of the equation be
inspected when comparing proofs that use this equation?
Please enter ms (for multiset ordering) or a permutation of [0 .. 1]
(0 stands for the consequent, i>0 for the ith condition). > \user{ms.}
\end{screen}

\noindent
The other two nonreductive equations are handled in the same way and we get the 
following complete system:

\begin{screen}
| ?- \user{show.}

Current equations

Current rules

  1    x=<x = true

Current nonoperational equations

  1    x=<y = false => y=<x = true
  2    x=<y = true and y=<x = true => x = y
  3    x=<y = true and y=<z = true => x=<z = true

All axioms reduced.
All superpositions computed.
No more equations, the system is complete.
\end{screen}

In this case, superposing the only rule \cec{x =< x = true} on the first
condition of the nonoperational equations does not generate any nontrivial
(nonconvergent) consequences.

\subsubsection{Completing {\tt qsort}}

The next step is the completion of the \cec{qsort} specification.
The completion procedure is able to orient the first and third 
equation of our specification, but fails to orient the second one.

\begin{screen}
| ?- c.

new rule   9    sort([]) = [] .

new rule  10    split(x,[]) = [],[] .
The equation
        split(x,l) = l1,l2 => sort([x|l]) = append(sort(l1),[x|sort(l2)])
is not reductive.
\end{screen}

\noindent
This is due to the presence of the {\em extra variables} \cec{l1} and \cec{l2}
in the condition of the equation.

Equations with extra variables in the condition or in the right side of
the consequence are usually not admitted as rewrite rules. 
Fortunately some of these equations belong to the class of what we call
{\em quasi-reductive rules} (\refArrow chapter \ref{Completion}). 
CEC is able to prove the remaining equations 2, 4
and 5 of the module \cec{qsort} to be quasi-reductive.
Quasi-reductive rules are a generalization
of reductive conditional rewrite rules and the associated rewrite
process is similarly efficient.
It specifies e.g. for the second equation the replacement
of \cec{sort([x|l])} by the term \cec{append(sort(l1), [x|sort(l2)])} 
if the normalform of \cec{split(x, l)} matches \cec{(l1,l2)}.
After this match the extra variables \cec{l1} and \cec{l2} are instantiated to
terms which include only variables of the left hand side.
In quasi-reductive rules, conditions are oriented, too. To solve an instance
of a condition equation means to rewrite its left side into an instance of the
right side.

\begin{screen}
Do you want a check for quasi-reductivity?
c. for check
n. for considering the equation as nonoperational
p. for postpone
   Please answer with c. or n. or p. (Type A. to abort) >\user{c.}
\end{screen}

\noindent
To orient an equation into a quasi-reductive rule, we must first indicate the 
desired orientation of the equations in the condition and the conclusion. 
In this example, the equation in the condition
and the conclusion should be oriented from left to right (literal annotation
``\user{l}'').

\begin{screen}
Enter annotations of literals in
	split(x,l) = l1,l2 => sort([x|l]) = append(sort(l1),[x|sort(l2)])
	: \user{[l,l].}
\end{screen}

\noindent
Now CEC attemps to prove the quasi-reductivity of the equation according to the
definition in chapter \ref{Completion}. This involves giving  polynomial
interpretations to some auxiliary operators.
Here we must give an appropriate interpretation for \cec{$h5_0}:

\begin{screen}
Checking reductivity constraint:
Consider the terms
        $h5_0((l1,l2),x)
and
        append(sort(l1),[x|sort(l2)])

There is no interpretation of operator '$h5_0' with arity 2
The default interpretation is :
   [ 2 * x * y ,
     2 * x * y ,
     2 * x * y ]
Do you want to change it ? (if so type 'y') \user{y}
Do you want to change it component for component ? (if so type 'y') \user{n}
Type in the new interpretation tuple
[ $h5_0 ] (x,y) = \user{[2*x+3*y+1,2*x+2*y,2*x+2*y].}
Resulting interpretation for Operator '$h5_0' with arity 2 :
   [ 2 * x + 3 * y + 1 ,
     2 * x + 2 * y ,
     2 * x + 2 * y ]
Do you accept it ? (if not, type 'n') \user{y}
\end{screen}

\noindent
Now the proof of the quasi-reductivity of the equation is completed:

\begin{screen}
new rule  11    split(x,l) = l1,l2 => sort([x|l]) = 
                                      append(sort(l1),[x|sort(l2)]) .
\end{screen}

\noindent
In the same way the equations

\cec{y=<x = true and split(x,l) = (l1,l2) => split(x,[y|l]) = ([y|l1],l2)}

\noindent
and 

\cec{y=<x = false and split(x,l) = (l1,l2) => split(x,[y|l]) = (l1,[y|l2])}

\noindent
can be oriented into quasi-reductive rules.

For the quicksort specification three nontrivial superposition instances are
computed. For any nontrivial equation with at least one condition that is
generated during completion, CEC will ask the user what to do with it. In the
example we decide to declare these consequences as ``nonoperational''.

\begin{screen}
instance   6    l1,l2 = l4,l3 and split(x1,l5) = l1,l2 => 
                                     append(sort(l1),[x1|sort(l2)]) = 
                                     append(sort(l4),[x1|sort(l3)])
of        6    split(x,l) = l1,l2 => sort([x|l]) = 
                                     append(sort(l1),[x|sort(l2)])
by superposing
          6    split(x,l) = l1,l2 => sort([x|l]) = 
                                     append(sort(l1),[x|sort(l2)]) 
on the left side.

Consider the equation
        l1,l2 = l4,l3 and split(x1,l5) = l1,l2 => 
                                     append(sort(l1),[x1|sort(l2)]) = 
                                     append(sort(l4),[x1|sort(l3)]).
The following actions may be taken:
o. for attempting to orient into a (quasi-)reductive rule
p. for postpone
n. for considering equation as nonoperational
   Please answer with o. or p. or n. (Type A. to abort) >\user{n.}
\end{screen}

As mentioned before it is sufficient to superpose all rewrite rules
on just one condition of a nonoperational conditional equation. So
CEC asks the user on which condition superposition should be
applied. 
We will choose the first equation of the condition, since we know it
will generate no nontrivial superpositions:

\begin{screen}
Which of the condition equations in
        l1,l2 = l4,l3 and split(x1,l5) = l1,l2 
should be selected for superposition?
Please enter index from 1 to 2. > \user{1.}

The equation l1,l2 = l4,l3 and split(x1,l5) = l1,l2 => 
                                          append(sort(l1),[x1|sort(l2)]) = 
                                          append(sort(l4),[x1|sort(l3)]) 
will be considered as nonoperational.
\end{screen}

\subsubsection{Completing {\tt nat}}

The completion of the \cec{nat} specification is straightforward using
the following order specification:

\begin{spec}
order poly1 for nat. 

setInterpretation([0 : 2,
                   s(x) : 8 * x]).
\end{spec}

\subsubsection{Combining {\tt qsort} and {\tt nat}}

We now want to combine the \cec{nat} specification with the 
\cec{qsort} specification.

\begin{screen}
| ?- store.
yes
| ?- load(qsort,poly3).
\end{screen}

\noindent
The \comRef{store}-command saves the completed {\tt nat} specification
into its {\em specification variable}.
The \comRef{load}-command loads the completed {\tt qsort} specification
from its specification variable.

\begin{screen}
| ?- renameSpec(elem <- nat).
yes
| ?- sig.

Signature :

cons [] : list.
cons .  : (nat * list -> list).
op append       : (list * list -> list).
cons true       : bool.
cons false      : bool.
op =<   : (nat * nat -> bool).
op sort	: (list -> list).
op split        : (nat * list -> pair).
cons ,    : (list * list -> pair).
yes
| ?- store(qsortnatSpec).
yes
\end{screen}

\noindent
Now we combine the two specifications and make the result 
to the current specification:

\begin{screen}
| ?- combineSpecs(qsortnatSpec,'nat.poly1_qsort',user).
yes
\end{screen}

\noindent
Because \cec{totalOrder} is a formal parameter for \cec{qsort},
completion now checks the consistency of the axioms for
\cec{=<} in the actual parameter \cec{nat} and in the formal
parameter \cec{totalOrder}.
The completion process will not do unnecessary work again. 
For the above example the completion process
will only compute overlaps between axioms
of the renamed module \cec{qsort} and axioms of the module \cec{nat}
but it will not recompute overlaps between axioms of one of these
modules.

If two constructor terms are shown to be equal then the specification
is inconsistent. In our case the system is completed without any user 
interaction, no inconsistency showing up.

\subsection{Computing in Completed Specifications}

Computation in specifications is realized by term reduction with the
rules of the completed specification. The result of such computations are unique
normal forms (\refArrow chapter \ref{NormCommand}):

\begin{screen}
| ?- norm(sort([5,3,6,1])).
The normalform of sort([5,3,6,1]) is [1,3,5,6] .
\end{screen}

\noindent
If confluence can be achieved and if all rules are reductive equational theorems 
become decidable, e.g. it is decidable if two terms are equivalent with respect to the
equations in the specification (\refArrow chapter \ref{ProveCommand}):

\begin{screen}
| ?- prove(sort([5,3]) = [3,5]).
Normal forms are: [3,5] and [3,5]
yes
\end{screen}

\noindent 
Conditional narrowing (\refArrow chapter \ref{NarrowCommand}) can then be used to solve 
equations.

\begin{screen}
| ?- solve(sort([1,x]) = [x,1],U).

Time used: 7.43298 sec.

U = {x-nat/0} ;

Time used: 7.91595 sec.

U = {x-nat/1} ;

no
\end{screen}

\noindent
Here we proved that the only substitutions for \cec{x} such that
\cec{sort([1,x])} is equal to \cec{[x,1]} are \cec{0} and \cec{1}.

\subsection{Saving the CEC System}

The whole state of the CEC system can be saved using the
\comRef{saveCEC}-command. 

\begin{screen}
| ?- saveCEC('qsortCEC').
[ Prolog state saved into /home/helga/cec/cec/qsortCEC ]
\end{screen}

Using \cec{qsortCEC} instead of CEC offers the possibility to have
the complete hierarchy of our \cec{qsort} example present, without
wasting time to reconsult the necessary frozen states of all the
specification used for this example.
