\section{Displaying and Changing the State of a Specification}

\subsection{Displaying the State of a Specification}
\label{Displaying}

\begin{command}[\comName{moduleName}]
displays the name of the current specification.
\end{command}

\begin{command}[\comName{orderName}]
displays the name of the order specification associated with
the current specification.
\end{command}

The state of a specification consists of the current signature, the set of equations
and rules and of the current definitions for the termination ordering. There are
several commands to print out this information.\bigskip

\begin{command}[\comName{sig}]
prints out the signature of the current specification:
\end{command}

If you are writing an order-sorted specifications there exist
additional, automatically generated operators which result from the 
translation into a many-sorted specification:
For every subsort relation s $<$ s' there will be an injective operator
with domain s und codomain s', called \verb|$inj-|s'\verb|-|s. If the domain can be
determined from the context, \verb|$inj-|s'\verb|-|s is abbreviated by s'.
These additional operators are also shown by the \comRef{sig}-command.\bigskip

\begin{command}[\comName{show}]
shows the sets of equations, rules and nonoperational equations of the 
current specification in order-sorted form.
(Usually there exists more than one many-sorted representation of
an order-sorted axiom.)

Rules \condRule{C}{\rewRule{l}{r}} which are marked by an asterix 
\kw{*} have an associated auxiliary rule of form
\condRule{C}{\rewRule{l+X}{r+X}}
where $+$ is the AC-operator on top of $l$ and $X$ is a new
variable of appropriate sort.
Auxiliary rules are automatically generated when needed during
completion modulo AC.
\end{command}

\begin{command}[\comName{showms}]
shows the sets of equations, rules and nonoperational equations of the 
current specification in many-sorted form.
\end{command}

\subsection{Inspecting the Termination Ordering}
\label{InspecTermination}

\begin{command}[\comName{order}]
indicates the current termination ordering and asks the user whether he wants to 
change it.
\end{command}

\begin{command}[\comName{operators}]
displays all precedences and stati in \kw{kns} or \kw{neqkns} or
all polynomial interpretations in \kw{poly}\nt{N} respectively.
\end{command}

\begin{command}[\comName{interpretation}]
displays all operator interpretations, provided \kw{poly}\nt{N} is the chosen termination
ordering and asks the user if he wants to change any. {\em If so all rules will be turned 
back into equations.}
\end{command}


\subsection{Checking Preregularity and Regularity}

To ensure that the set of order-sorted equations of the specification and
the set of rewrite rules produced by CEC describe the same equational
theory, the signature must be {\em preregular} (\refArrow \cite{SNGM87}).
If in addition the signature is {\em regular} the term algebra is an
initial algebra in the semantics of \cite{GM87} (and \cite{SNGM87}).

\noindent
There are commands which check preregularity and regularity:\medskip

\begin{command}[\comName{preregular}]
succeeds if the current (order-sorted) signature is preregular,
and fails otherwise.\\
The preregularity condition is the regularity of 
\cite{SNGM87}:
A signature is preregular, iff for every function symbol
$f$ and every string $w$ of sorts the set
$$   \{ t \mid \mbox{\ there is a declaration\ } f : w' \rightarrow t 
\mbox{\ such that\ } w \leq w' \} 
$$
is either empty or has a minimal element.
\end{command}

\begin{command}[\comName{regular}]
succeeds if the current (order-sorted) signature is regular,
and fails otherwise.
The regularity condition is the one of \cite{GM87}:
A signature is regular, iff for every function symbol
$f$ and every string $w$ of sorts the set
$$
   \{ (w',t) \mid \mbox{\ there is a declaration\ } f : w' \rightarrow t 
      \mbox{\ such that\ } w \leq w' \}
$$
is either empty or has a minimal element.
\end{command}

\subsection{The UNDO Mechanism}

\begin{command}[\comName{undo}]
can be entered at the system's top level to reset the system to the state
before the last command that has caused a state change, if there was
any. \comRef{undo} can be used repeatedly to undo several steps of state changes.
It also undoes \comRef{undoUndo}-calls. At the moment, there is no way to backtrack
from single decisions that have been taken while running the completion
process.
\end{command}

\begin{command}[\comName{undoUndo}]
allows to undo the last \comRef{undo}-command at the system's top level. Repeated use of
this command undoes sequences of \comRef{undo}-commands. Chains of undo-calls begin at 
the last user interaction different from an \comRef{undo} or \comRef{undoUndo}.
\end{command}


