                                                         
\section{Input Format of Specifications}

Specifications consist of a {\em specification name}, an optional 
{\em base declaration} and an {\em enrichment}. Additionally an
{\em order and action specification} can be supplied to support the completion
process. Of course it is possible to have more than one order specification
associated with a specification.

\begin{syntax}
\nt{specification} \IS  \kw{module} \nt{specificationName}
                   \AND [ \nt{base} ] \nt{full stop} 
                   \AND \nt{enrichment} 
\end{syntax}

\noindent
The name declaration has the format:

\begin{syntax}
\nt{specificationName}  \IS \nt{Prolog atom} \END
\end{syntax}

\noindent
The specification name is used to denote the specification.
If the specification is stored in a file,
the file should be named \nt{specificationName}\suffix{.eqn}.

The base declaration allows to import other specifications as
modules. The imported modules
again form a module called {\em base module}.  An enrichment defines (parts
of) a {\em signature}, {\em axioms} and {\em pragmas}. The base module
together with the enrichment forms the current specification.  
% Pragmas may be added in order to speed up the completion process, e.g. the 
% choice of a termination ordering and declarations for that ordering. This 
% is to avoid unnecessary computations of suggestions and unwanted user interaction.
Pragmas may redefine ``parsing'' and ``pretty-printing'' of specifications.
Any element of a specification must be terminated by a full stop 
which is a ``.''
followed by a white space.  This syntactic requirement follows from the
conventions of the underlying Prolog-System. {\em Comments} also
follow Prolog syntax, e.g.\ anything in a line after a \kw{%} or
all text between \kw{/*} and \kw{*/} is a comment.
For those who are unfamiliar with Prolog the syntax of a Prolog atom is provided:

\begin{syntax}
\nt{Prolog atom} 	 \IS \nt{textual atom} \ts \nt{symbol atom} \END
\nt{textual atom}        \IS \nt{lower case letter} \{ \nt{letter} \ts \kw{_} \ts \nt{digit} \} 
                         \OR \kw{'} \{\nt{letter} \ts \nt{digit} \ts \nt{symbol} \ts \nt{white space} \ts \kw{''} \} \kw{'} 
			 \OR \_ \END
%                         \GETON \hspace*{0.2ex} \ts \kw{'} \{\nt{letter} \ts \nt{digit} \ts \nt{symbol} \} \kw{'}\\
\nt{symbol atom}         \IS \nt{symbol} \{ \nt{symbol} \} \ts \kw{[}\,\kw{]} \END
\nt{full stop} 	         \IS \kw{.} \nt{white space} \END
\nt{white space} 	 \IS \kw{\t} \ts \kw{\n} \ts \nt{blank} \END
\\
\nt{Prolog number} 	 \IS \nt{integer} \ts \nt{floating-point number} \END
\end{syntax}

\noindent
(The syntax is given in EBNF with \{\} for repetitions and [ ] for options.
The terminal symbols are in bold face font style.)

We distinguish between {\em many-sorted} and {\em order-sorted} specifications:
A specification is called an {\em order-sorted specification}, if the
base module or the enrichment includes a {\em subsort declaration}. 
Otherwise it is called {\em many-sorted}.

\subsection{Base}
\label{base}

The base has the form

\begin{syntax}
\nt{base} \IS \kw{using} \nt{moduleExpression}
\end{syntax}

The \nt{moduleExpression} describes which modules should be imported and
how certain sorts and operators should be renamed to fit in the 
importing specification.

\begin{syntax}
\nt{moduleExpression} \IS \nt{specificationName} 
 			\OR \nt{moduleExpression} \kw{+} \nt{moduleExpression}
			\OR \nt{specificationName} \kw{(} \nt{AssociationList} \kw{)} \END
\\
\nt{AssociationList} \IS \nt{Association} \{ \kw{,} \nt{Association} \} \END
\nt{Association}         \IS \nt{sortAssociation}
			 \OR \nt{operatorAssociation} \END
\\
\nt{sortAssociation}      \IS \nt{sortName} \kw{<-} \nt{sortName} \END
\nt{sortName} 		   \IS \nt{Prolog atom} \END
\\
\nt{operatorAssociation}   \IS \nt{simpleOpName} \kw{<-} \nt{simpleOpName} \END
\nt{simpleOpName}          \IS \nt{Prolog atom} \ts \nt{Prolog number} \END
\end{syntax}

The semantics of the `\kw{+}'-operation  is union of specifications
(signature, axioms, termination ordering).
The union fails, if the termination orderings cannot be combined
with the heuristics in CEC
(e.g.\ different interpretations are given for the same operator).

In the third case the given specification morphism is applied to a 
module expression.
For the operator renaming injectivity is required. Noninjective
operator renamings must be simulated using auxiliary operators
and equations.

All  specifications which are refered to by their \nt{specificationName}  
in the base of a 
specification are called {\em direct imports} of this specification.


If no base is specified, the module `\file{standard}' is
taken from its specification variable, if present. Otherwise it is taken from
the current directory (thaw file `\file{standard.q2.0}').
If no module standard is found in the current directory,
the module `\file{standard}' from the CEC-distribution is used.
The latter simply declares the sort \cec{bool} and the two constructor
constants \cec{true} and \cec{false} of sort \cec{bool}.

The evaluation of the base depends on the presence and the contents of
an {\em order specification} (cf. chapter \ref{OrderSpecification}).

%The base of a specification is evaluated whenever this specification
%is consulted using the \comRef{in}-command. If the \comRef{in}-command also specifies
%an \nt{orderName} of an orderSpecification, this order specification is
%also consulted. The order specification determines which files must
%be consulted to find the specifications of the direct antecedents.
%If some <orderName> beside `noorder' is associated with the 
%<specificationName> of a direct import, the system tries to thaw 
%a frozen state of a specification (\refArrow \comRef{freeze}) from a file name
%   <specificationName>.<orderName>.q2.0    (Quintus 2.x versions)
%in the current directory (\refArrow \comRef{cd}, \comRef{pwd}).
%If no such files are present the system looks for a file
%	<specificationName>.eqn
%with the text of the specification module according to the syntax of
%<specification> to be read in and for a file
%	<specificationName>.<orderName>.ord
%which contains pragmas concerning the termination ordering.
%If the the <orderName> associated with <specificationName> is 
%'noorder' or there is no <orderName> associated with <specificationName> 
%the system looks only for a file 
%	<specificationName>.eqn
%in the current directory (\refArrow \comRef{cd}, \comRef{pwd}).

\subsection{Enrichment}
\label{enrichment}

An enrichment consists of {\em subsort declarations},
{\em operator declarations}, {\em variable declarations}, 
{\em axioms} and {\em clauses for parsing and pretty printing}:

\begin{syntax}
\nt{enrichment} \IS \{ \nt{subsortDeclaration} \}
 \AND \{\nt{operatorDeclaration} \} 
 \AND [ \nt{variableDeclarations} ] 
 \AND \{ \nt{axiom} \} 
% \AND \{ \nt{pragma} \} 
 \AND [ \nt{clauses for parse and pretty} ]
\end{syntax}

The semantics of a \nt{specification} is the specification
which is obtained by enriching the specified base module
by the constituents of the enrichment. There is no implicit
assumption about consistency or sufficient-completeness of this
enrichment.

\subsubsection{Subsort Declarations}
\label{subsortdecl}

In order-sorted specifications subsort declarations are given 
with regard to the following syntax: 

\begin{syntax}
\nt{subsortDeclaration} \IS \nt{sortName} \kw{<} \nt{sortName} \nt{full stop} \END
\nt{sortName}           \IS \nt{Prolog atom}
\end{syntax}

\subsubsection{Operator Declarations}
\label{operators}

Operator declarations define the {\em signature} of a specification. 
An operator is characterized by its name, which is a Prolog atom, and
its arity.
In addition it is 
possible to declare the {\em constructor} property of an operator. The system
makes use of this information when doing {\em termination proofs} and
{\em consistency checks}. It is not allowed to orient an equation into a rule
such that the top of the left side is a constructor.
Operator declarations may be given according to the following syntax: 

\begin{syntax}
\nt{operatorDeclaration} \IS \nt{opKind} \nt{opName} \kw{:} \nt{arity} \nt{full stop}\END
\nt{opKind} \IS \kw{op} \ts \kw{cons} \END
\nt{opName} \IS \kw{(} \nt{Prolog atom}\kw{,} \nt{priority}\kw{,} \nt{fix} \kw{)}
 \OR \nt{Prolog atom}
 \OR \nt{Prolog number}
\end{syntax}

\noindent
{\em Note that for the first alternative of \nt{opName} there must not be
 any \nt{white space}
between a preceding \nt{opKind} and the open parenthesis ``\kw{(}''. }

\begin{syntax}
\nt{priority} \IS \nt{{\em number from 1 to 1200}} \END
\nt{fix} \IS \kw{fx} \ts \kw{fy} \ts \kw{xfx} \ts \kw{xfy} 
              \ts \kw{yfx} \ts \kw{xf} \ts \kw{yf} \END
\nt{arity} \IS \nt{resultSort} 
 	  \OR \kw{(} \nt{ argumentSorts} \kw{->} \nt{resultSort} \kw{)} \END
\nt{argumentSorts} \IS \nt{sortName} \{ \kw{*} \nt{sortName} \} \END
\nt{resultSort}    \IS \nt{sortName}\END
\nt{sortName}      \IS \nt{Prolog atom} \END
\end{syntax}

As CEC is embedded in a Prolog environment it inherits all facilities of term
notation offered there, e.g.\ the definition of {\em infix, prefix} 
or {\em  postfix} operators. 

Each infix, prefix or postfix operator has a {\em precedence}, which is a
number from 1 to 1200. The precedence is used to disambiguate expressions in
which the structure of the term is not made explicit through the use
of parentheses.  The general rule is that the operator with the {\em highest}
precedence is the principal functor. 

%Thus if \cec{+} has a higher precedence than ``\cec{/}'', then \bigskip
%
%\cec{a+b/c   a+(b/c)} \bigskip
%
%\noindent
%are equivalent and denote the term \cec{+(a,/(b,c))}.
%Note that the infix form of the term \cec{/(+(a,b),c)}
%must be written with explicit brackets\bigskip
%
%\cec{(a+b)/c}  .\bigskip
%
%
If there are two operators in the expression having the same highest 
precedence,
the ambiguity must be resolved from the {\em fixes} of the operators. 
The possible fixes for an infix operator are \bigskip

\cec{xfx   xfy   yfx}  . \bigskip

Operators of fix ``\cec{xfx}'' are not associative: it is required 
that both arguments of the operator must be subexpressions of 
{\em lower} precedence than the operator itself, i.e. their principal 
functors must be of lower precedence, unless the subexpression is
written in parentheses (which gives it zero precedence).
Operators of fix ``\cec{xfy}'' are right-associative:
only the first subexpression must be of lower precedence; the second 
subexpression can be of the {\em same} precedence as the main
operator; and vice versa for an operator of fix ``\cec{yfx}''.

%For example, if the operators ``\cec{+}'' and ``\cec{-}'' both have fix 
%``\cec{yfx}'' and are
%of the same precedence, then the expression\bigskip
%
%\cec{a-b+c}\bigskip
%
%\noindent
%is valid, and means\bigskip
%
%\cec{(a-b)+c} \hspace{2em}     i.e.  \cec{+(-(a,b),c)}.\bigskip
%
%\noindent
%Note that the expression would be invalid if the 
%operators have fix ``\cec{xfx}'', and would mean\bigskip
%
%\cec{a - (b+c)} \hspace{2em}    i.e. \cec{-(a , +(b , c))}\bigskip
%
%\noindent
%if both operators have fix ``\cec{xfy}''.
%
\noindent
The possible fixes for a prefix operator are\bigskip

\cec{fx     fy} \bigskip

\noindent
and for a postfix operator they are\bigskip

\cec{xf     yf} .\bigskip

The meaning of the fixes should be clear by analogy with those for infix operators.
%As an example, if ``\cec{not}'' were declared as a prefix operator of fix 
%``\cec{fy}'', then \bigskip
%
%\cec{not not P} \bigskip
%
%\noindent
%would be a permissible way to write \cec{not(not(P))}. 
%If the fix were ``\cec{fx}'',
%the preceding expression would not be legal, although\bigskip
%
%\cec{not P}\bigskip
%
%\noindent
%would still be a permissible form for \cec{not(P)}.

It is possible to have more than one operator of the same name, as long as they are
of different kinds, i.e. infix, prefix or postfix. An operator of any kind may be
redefined by a new declaration of the same kind, except of some operators
listed in appendix \ref{PredefinedOperators}.

\noindent Examples of operator declarations are: \bigskip

\cec{cons 0 : nat.}\bigskip

\cec{cons(suc,100,fy) : (nat -> nat).}\hfill \bigskip

\cec{op + : (nat * nat -> nat).}\bigskip 

\cec{op =< : (nat * nat -> bool).}\bigskip

\noindent 
Note that in the last declaration the infix notation
\cec{op(=<,700,xfx)} will be inherited from Prolog. 

Overloading of operators is allowed. They will however be renamed internally.

\noindent
The following syntax restrictions serve to remove potential ambiguities associated with
prefix operators.
\begin{itemize}
\item
In a term written in standard syntax, the principal functor and its following 
``\kw{(}'' must {\em not} be separated by any blankspace. Thus
\begin{quotation}
\cec{point (X, Y, Z)} \medskip
\end{quotation}
is invalid syntax.
\item
If the argument of a prefix operator starts with a ``\cec{(}'', this ``\cec{(}'' 
must be separated from the operator by at least one space or other
non-printable character. Thus 
\begin{quotation}
\cec{:-(p ; q), r.}\medskip
\end{quotation}
(where ``\cec{:-}'' is the prefix operator) is invalid syntax, and must be written as
\begin{quotation}
\cec{:- (p ; q), r.}\medskip
\end{quotation}
\item
If a prefix operator is written without an argument, as an ordinary atom, the atom is
treated as an expression of the same precedence as the prefix operator, and must
therefore be written in parentheses where necessary. 
Thus the parentheses are necessary for \cec{-} in 
\begin{quotation}
\cec{op (-) : (nat * nat -> nat).}
\end{quotation}
\end{itemize}
For further details you may have to
look at the Quintus-Prolog user manual. 

\subsubsection{Variable Declarations}
\label{vardecl}

Each Prolog atom in the specification which is not declared as an 
operator will be taken as a variable. To declare the sort of
a variable you only have to add the sortname to the variablename:

\begin{syntax}
\nt{variable} \IS \nt{Prolog atom} [ \kw{:} \nt{sortName} ]
\end{syntax}

In many-sorted specifications without overloading it is not required to
attach a sort to a variable explicitly: A simple type inference mechanism is
used to find the correct sort for the variable. In many-sorted specifications
with overloading this type inference need not produce the desired typing. 
Then it is helpful to define the sorts of variables explicitly.
In order-sorted specifications it is usually not possible to deduce
the sort of a variable. In this case all variables
must be explicitly typed. This can be done by adding
the sort name as described above or by declaring the variable in the
variable declaration part: 

\begin{syntax}
\nt{variableDeclarations} \IS \kw{var} 
 \nt{variableDecl} \{ \kw{,} \nt{variableDecl} \} \nt{full stop} \END
\nt{variableDecl} \IS \nt{Prolog atom} \kw{:} \nt{sortName}
\end{syntax}

The scope of a variable declaration is not bound to the module it is
contained in, e.g.\ the variable declarations in a base module are
inherited by the current specification, if not redeclared.

\subsubsection{Axioms}
\label{axioms}

CEC includes a completion procedure for conditional (and unconditional)
equations. Equations are of the form:

\begin{syntax}
\nt{axiom} \IS \nt{unconditionalEquation} \nt{full stop}
           \OR  \nt{conditionalEquation} \nt{full stop} \END
\nt{unconditionalEquation} \IS \nt{equation} \END
\nt{conditionalEquation}   \IS \nt{condition} \nt{equation} \END
\nt{condition} \IS \nt{equation} \{ \kw{and} \nt{equation} \} \kw{=>}\END
\\
\nt{equation} \IS \nt{signatureTerm} \kw{=} \nt{signatureTerm} \END
\\
\nt{signatureTerm} \IS $<$well typed term with variables
		   \GETON \hspace{1ex} over the user-specified signature$>$
\end{syntax}

\noindent Thus \bigskip

\cec{isChar(c) = true => isLine([c | l]) = isLine(l).} \bigskip

\noindent and\bigskip

\cec{isSpace(c) = true => isLine([c | l]) = isLine(l).}\bigskip

\noindent
are valid conditional equations in a many-sorted specification as well as\bigskip

\cec{isChar(c) = true and isLine(l) = true => isLine([c | l]) = true.}\bigskip

\noindent
where \cec{c} and \cec{l} are variables.\\
In an order-sorted specification the variables must be declared 
in the variable declaration part:\bigskip

\cec{var c : char, l : string.} \bigskip

\noindent
or else the sort must be attached  to each occurrence of a variable:\bigskip

\cec{isChar(c:char) = true => isLine([c:char | l:string]) = isLine(l:string).} \bigskip

\noindent
Unconditional equations are treated as conditional equations with an empty condition.


\subsubsection{Clauses for Parse and Pretty}
\label{ParseAndPretty}

\noindent
The definition of a predicate {\em parse} allows the user to define his
own term notation. If parse fails, standard Prolog term notation is assumed.\bigskip

\begin{command}[\com{parse}{\comArg{Term}\ad\comArg{ParsedTerm}}]
should map the \comArg{Term} to its standard representation \comArg{ParsedTerm} as 
assumed by the system. This representation is the usual Prolog
representation formed from the operator symbols in the signature. The auxiliary
operator '@' is used to denote variables in the term (The term must not
contain any Prolog variables!).\bigskip


Example:\smallskip
\begin{tabbing}
12345\=12345\=eee\kill
\> Specification of integer with operators \cec{s} (successor) and 
\cec{p} (predecessor):\\
\\
\> \> \cec{parse(0,0).}\\
\> \> \cec{parse(I,s(T)) :- integer(I), I > 0, !, IM1 is I-1, parse(IM1,T).}\\
\> \> \cec{parse(I,p(T)) :- integer(I), I < 0, !, I1 is I+1, parse(I1,T).}\\
\\
\> \cec{parse(5, ParsedTerm)} yields \cec{ParsedTerm = s(s(s(s(s(0)))))}.
\end{tabbing}
\end{command}

\begin{command}[\com{pretty}{\comArg{Term}}]
Upon pretty printing of terms the system tries to call 
\com{pretty}{\comArg{Term}} (for each subterm). If this succeeds, 
the system assumes that the (sub-)term has been
printed. Otherwise, \com{print}{\comArg{Term}} is called.\bigskip

Example:\smallskip
\begin{tabbing}
12345\=12345\=eeee\kill
\> Specification of integer with operators \cec{s} (successor) and 
\cec{p} (predecessor):\\
\\
\> \> \cec{pretty(s(X)) :- predicate(fromUnary(s(X),N)), write(N).}\\
\> \> \cec{pretty(p(X)) :- predicate(fromUnary(p(X),N)), write(N).} \\
\\
\> \cec{pretty(s(s(s(s(s(0))))))} prints out {\tt 5}.
\end{tabbing}
\end{command}

\noindent
One may define arbitrary auxiliary predicates (i.e \comRef{fromUnary})
for the definition of \comRef{parse} or \comRef{pretty}. 
To prevent name clashes with internal CEC predicates
you have to enclose their definitions and calls in $predicate(\ldots)$. \bigskip

Example:\smallskip
\begin{tabbing}
12345\=12345\=eee\kill
\> Specification of integer with operators \cec{s} (successor) and 
\cec{p} (predecessor):\\
\\
\> \> \cec{predicate(fromUnary(X,_)) :- var(X), !, fail.}\\
\> \> \cec{predicate(fromUnary(0,0)).}\\
\> \> \cec{predicate(fromUnary(s(X),N1)) :- predicate(fromUnary(X,N)), N1 is N+1.}\\
\> \> \cec{predicate(fromUnary(p(X),N1)) :- predicate(fromUnary(X,N)), N1 is N-1.}
\end{tabbing}

If the current specification is a combination of several specifications, the clauses for \comRef{parse},
\comRef{pretty}, and \comRef{predicate} will be put together in
{\em arbitrary order}. Note that upon renaming of specifications parse and
pretty might no longer work for the renamed signature.

\subsection{Specification of Termination Orderings}
\label{OrderSpecification}

\noindent
While running the completion process you are interactively asked for additional
information, mainly
concerning the construction of the termination ordering for the given
specification. The system supports this process by making suggestions for precedence
extensions (path orderings). In many cases you will know a 
priori many of these properties. 
Then you can supply a corresponding {\em order specification} for your
initial specification to avoid unnecessary computations of suggestions 
and to reduce the number of CEC-questions.\bigskip

\begin{syntax}
\nt{orderSpecification} \IS \kw{order} \nt{orderType} \kw{for} \nt{specificationName} 
			 \AND [ \nt{orderBase} ] 
			 \AND [ \nt{orderPragmas} ] \END
\\
\nt{orderType} \IS  \kw{kns}
		\OR \kw{neqkns} 
		\OR \kw{poly}\nt{natural number}
		\OR \kw{manual}
\end{syntax}

\noindent
The order specification determines the termination ordering to be used for 
\nt{specificationName}, the order names for its direct imports (\refArrow
\ref{orderBase}) and gives precedence declarations for the operators of the
specification (if the termination ordering is \kw{kns} or \kw{neqkns}) or
polynomial interpretations (if the termination ordering is \kw{poly}\nt{N})
(\refArrow \ref{OrderPragmas}).

Available types of orderings are
\begin{itemize}
\item \kw{kns} \\
This ordering is a recursive path ordering based on the recursive comparison of paths
according to the definitions in \cite{KNS85}.
\item \kw{neqkns} \\
This ordering is a restriction of the kns ordering:
Different operators cannot have the same precedence.
\item \kw{poly}\nt{N}\\
Here, terms are given polynomial interpretations over +, * where
coefficients are natural numbers.
The order of terms is reduced to comparing polynomials in the order on natural
numbers (\cite{CL86}).
\item \kw{manual}\\
This is not a termination ordering because the user is asked to decide the
orientation of the rules manually. Therefore the termination and confluence of the
final rewrite system cannot be guaranteed
\footnote[1]{In CEC it is even more dangerous to switch to manual mode, as
the termination order is lifted internally to an ordering of proofs in
equational logic. The latter is used for elimination of conditional 
equations. This effect is completely outside of the user's control.}.
\end{itemize}

\noindent You should write order specifications which belong to the
specification \nt{specificationName} in a file named 

\hbox to \hsize{\hfill
\nt{specificationName}\kw{.}\nt{orderName}\suffix{.ord}, \hfill} 
\noindent where \nt{orderName} can be any Prolog atom.
\nt{orderName} is then the {\em current order name} of the specification
\nt{specificationName}. If no order specification is supplied, the current
order name is the actual termination ordering, which is \kw{neqkns} or
\kw{poly}\nt{N} depending on the presence of AC-operators. 
The termination ordering can be changed using the \comRef{order}-command.  Of
course it is also possible to have more than one order specification for a
single specification, simply distinguished by different order names.



\subsubsection{Order Base}
\label{orderBase}

\noindent
An order specification is an enrichment of an {\em order base definition}:

\begin{syntax}
\nt{orderBase} \IS \kw{using} \kw{(} \nt{orderExpression} \kw{)} \nt{full stop} \END
\\
\nt{orderExpression} \IS  \nt{orderName} \kw{for} \nt{specificationName} 
		      \OR  \nt{orderExpression} \kw{+} \nt{orderExpression} \END
\\
\nt{orderName} \IS \kw{noorder}
		\OR \nt{Prolog atom except \kw{noorder}}
\end{syntax}

For an explanation of the interpretation of the order base refer to the description of the
\comRef{in}-command (\refArrow chapter \ref{InCommand}).
\subsubsection{Order Pragmas}
\label{OrderPragmas}

\begin{command}[\com{constructor}{\comArg{OpName}}]
declares operator \comArg{OpName} to be a constructor operator. 
For order-sorted specifications also 
the many-sorted disambiguated operators are allowed for \comArg{OpName}.
\end{command}

\noindent
{\bf CEC Pragmas concering the termination orderings KNS and NEQKNS:}\bigskip

\begin{command}[\com{equal}{\kw{[}\kw{[}\comArg{a}\kw{,}\comArg{b}\kw{,}\comArg{c}\kw{,}\ldots\kw{]}, 
\kw{[}\comArg{g}\kw{,}\comArg{h}\kw{,}\comArg{i}\kw{,} \ldots \kw{],} \ldots \kw{]}}\kw{.}]
declares operators to have equivalent precedences (only allowed for \kw{kns}).\\
Meaning: \comArg{a} = \comArg{b} = \comArg{c} = \ldots and 
\comArg{g} = \comArg{h} = \comArg{i} = \ldots
\end{command}

\begin{command}[\com{greater}{\kw{[}\kw{[}\comArg{a}\kw{,}\comArg{b}\kw{,}\comArg{c}\kw{,} \ldots \kw{],}
\kw{[}\comArg{g}\kw{,}\comArg{h}\kw{,}\comArg{i}\kw{,} \ldots \kw{],} \ldots \kw{]}}\kw{.}]
adds ordered pairs of operators to the precedence.\\
Meaning: \comArg{a} $>$ \comArg{b} $>$ \comArg{c} $>$ \ldots 
and \comArg{g} $>$ \comArg{h} $>$ \comArg{i} $>$ \ldots
\end{command}

\begin{command}[\com{status}{\kw{[}\comArg{Operator} : \comArg{Status}\kw{,} \ldots \kw{]}}\kw{.}]
declares that \comArg{Operator} has status \comArg{Status} provided 
\kw{kns} or \kw{neqkns} is the chosen termination ordering.\\
\comArg{Status} can be
\kw{lr} for {\em left-to-right}, \kw{rl} for {\em right-to-left} or
\kw{ms} for {\em multiset}.
\end{command}

\noindent
{\bf CEC Pragmas concerning the termination ordering POLY$<$N$>$:}\bigskip

\begin{command}[\com{setInterpretation}{\kw{[}\comArg{Operator(Arguments)} : \comArg{Interpretation}\kw{,} \ldots\kw{]}}\kw{.}]
Provided that \kw{poly}\nt{N} is the current termination ordering a new interpretation
\comArg{Interpretation} for an operator \comArg{Operator} is added to the current state. The interpretation
\comArg{Interpretation} must be a polynomial over the variables in 
\comArg{Arguments} (if N = 1) or a list with 
 N such polynomials. Commutative and associative-commutative operators
require interpretations of a particular kind.
An example: \bigskip

\cec{setInterpretation(['=='(x, y) : x + y]).}\bigskip

\noindent
For more detailed explanations concerning termination proofs based on polynomial 
interpretations see below. Initially operators have no interpretations.
\end{command}

These CEC-pragmas should not be used interactively while CEC is running. Use the
corresponding CEC-commands  \comRef{constructor}, \comRef{equal}, \comRef{greater}, \comRef{status}
or \comRef{setInterpretation} without parameters which 
request their input interactively.\bigskip


\subsection{Syntax and Type Errors}

When reading in a specification two kinds of errors may occur:
\begin{itemize}
\item
The Prolog parser detects a syntax error at the level of Prolog terms. 
\item
The CEC-parser finds a syntax error.
\item
The type checker detects a type inconsistency or is unable to infer a type for some
term because of missing operator definitions.
\end{itemize}

\noindent
In this case the specifications is rejected and CEC sets back to its
state before reading.

