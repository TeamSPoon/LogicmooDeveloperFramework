
\section{Input and Output of Specifications}
\label{InputOutput}

CEC has a variety of different I/O-commands. There are two commands for reading in
original specifications from a file (\comRef{in} and \comRef{enrich})
two commands for saving and restoring partially completed specifications 
to/from files (\comRef{freeze} and \comRef{thaw})
two commands for saving and restoring specifications to/from 
specification variables (\comRef{store} and \comRef{load}) and
two commands for storing and loading {\em log-files}
(\comRef{storeLog} and \comRef{loadLog}).
Also, the commands for saving and 
restoring the whole CEC state, \comRef{saveCEC} and \comRef{restoreCEC}, can be used. But
notice that saving CEC states (prolog states) usually generates very
large files. 
%Conventions concerning the allowed file and variable names are 
%those of the underlying prolog system. 
Reading specifications from standard input
(keyboard, terminal) is possible by using the predefined file name \cec{user}.


\subsection{Reading Specifications from Files}
\label{InCommand}

\begin{command}[\com{in}{\comArg{ModuleName}\ad\comArg{OrderName}}]
reads in a specification from the file \comArg{ModuleName}\suffix{.eqn}
and the associated order specification from the file
\comArg{ModuleName}\suffix{.}\comArg{OrderName}\suffix{.ord}.
As log-files are ``enriched'' order specifications, any log-file can
be used as an order file.
The  order  base of the order specification determines  the order names for the direct
imports of a specification. The system will first look for the
specification variable 

\hbox to \hsize{\hfill
\nt{specificationName}\kw{.}\nt{orderName}\hfill.}
If such a variable exists its content will be used. If such a variable does not
exist and some \nt{orderName} $\neq$ \kw{noorder} is associated with  
\nt{specificationName} the system will look for the file

\hbox to \hsize{\hfill
\nt{specificationName}\kw{.}\nt{orderName}\suffix{.q2.0}\hfill}
\noindent in the current directory containing a frozen state of the
referenced specification.
\noindent
If this file does'nt exist the system looks for a file

\hbox to \hsize{\hfill
\nt{specificationName}\suffix{.eqn}\hfill}
\noindent
which contains the specification according to the syntax of 
\nt{specification} and for a file 

\hbox to \hsize{\hfill
\nt{specificationName}\kw{.}\nt{orderName}\suffix{.ord}\hfill}
\noindent
which contains the order specification for the specification.
If the \nt{orderName} associated with \nt{specificationName} is 
\kw{noorder} the system looks for the specification in file

\hbox to \hsize{\hfill
\nt{specificationName}\suffix{.eqn}\hfill}
\noindent
in the current directory.

If not stated otherwise these files are assumed to be in the current 
directory. Before the specification is read in, CEC will be re-initialized, 
e.g. the current specification will be deleted. Specifications saved in 
variables will not be affected. \comArg{ModuleName} and \comArg{OrderName}
must be Prolog atoms. \comArg{OrderName} becomes the
current order name for the specification.\\
\com{in}{\comArg{ModuleName}\ad\cec{noorder}}
has the effect that no order specification is consulted.
The termination ordering for the new
specification will be initialized to a default value 
(\kw{neqkns} or \kw{poly1}, depending on the presence of AC-operators).
Using \comRef{in} only with the parameter \comArg{ModuleName} yields the same
effect. \comArg{ModuleName} = \kw{user} expects input from terminal.
(For Quintus-Prolog2.x under EMACS: {\it in} without
parameter reads from \cec{Scratch.pl}).
\end{command}

If you want to access files from other directories you will have to specify these
directories relative to the current directory at the time of invoking Prolog, e.g.
\cec{in('examples/math/int')}. A more convenient way will be to specify the
necessary prefix to all files once and for all by\bigskip

\begin{command}[\com{cd}{\comArg{Path}}]
Changes, as the cd command in UNIX, the directory for all following
file-related CEC-commands.
The path is given in form of a Prolog atom, hence don't forget the
quotes, if the path contains `\kw{/}', `\kw{.}', or `\kw{..}' or other special
characters.

\comRef{cd} is predeclared as prefix operator. So after execution 
of \bigskip

\cec{cd 'examples/math'.} \bigskip

\noindent
the command \cec{in(int)} will read in \cec{examples/math/int.eqn}.

\comRef{cd} without an argument resets the current directory to the one in
which the CEC-system was initially invoked.
\end{command}

\begin{command}[\comName{pwd}]
prints out the current path for I/O-related commands.
\end{command}

\begin{command}[\com{enrich}{\comArg{ModuleName}\ad\comArg{OrderName}}]
reads in additional parts of a specification from the files 
\comArg{ModuleName}\suffix{.eqn} and \comArg{ModuleName}\suffix{.}\comArg{OrderName}\suffix{.eqn}
after saving the current state for later checks for consistency of the enrichment. 
These additional parts must form an enrichment (cf. chapter~\ref{enrichment}).
\comArg{ModuleName} and \comArg{OrderName} can be arbitrary Prolog atoms.
The \comArg{OrderName} or even both arguments 
can be omitted, with similar effects as for \comRef{in}.
%Specify \comArg{ModuleName} = \kw{user} if input from terminal is wanted.
\end{command}

\subsection{Freezing and Thawing Partially Completed Specifications to/from Files}
\label{FreezeCommand}
\label{ThawCommand}

For saving and restoring states of specifications to/from files there exist the 
commands \comRef{freeze} and \comRef{thaw}.\bigskip

\begin{command}[\com{freeze}{\comArg{ModuleName}\ad\comArg{OrderName}}]
writes the state of the current specification to the file
\comArg{ModuleName}.\comArg{OrderName}\suffix{.q2.0}.
If freeze is called without \comArg{OrderName} the state is written
to \comArg{ModuleName}\suffix{.q2.0}.
%the current order name is taken for \comArg{OrderName}. 
If freeze is called without any argument at all the module name of the current 
specification is used for \comArg{ModuleName} and the current order name
is used for \comArg{OrderName}.
In this case, also a log-file is produced.
The specification may later be reused by thawing it from this file, cf. the 
\comRef{thaw}-command.
The state of CEC remains unchanged by this operation.
\end{command}

\begin{command}[\com{thaw}{\comArg{ModuleName}\ad\comArg{OrderName}\ad\comArg{SpecificationVariable}}]
This command is the inverse operation of \comRef{freeze} and restores the specification
previously frozen in \comArg{ModuleName}\suffix{.}\comArg{OrderName}\suffix{.q2.0}
to the specification variable \comArg{SpecificationVariable}. The current specification and other variables 
will not be affected by this operation.
If \comRef{thaw} is called without \comArg{SpecificationVariable} 
the current specification is overwritten by the thawed specification. 
Specifications stored in variables will not be affected by this operation.
If \comRef{thaw} is used only with argument \comArg{ModuleName} the frozen 
specification will be taken from the file \comArg{ModuleName}\suffix{.q2.0}.
\end{command}

\subsection{Storing and Loading of Log-files}
\label{StorelogCommand}
\label{LoadlogCommand}

Log-files are used to save the information given from the user during the last
completion process and to save the definition of the current termination
ordering. The log-file can be used to replay a completion fully (or partially)
automatically on ``closely related'' specifications. \bigskip

\begin{command}[\com{storeLog}{\comArg{ModuleName}\ad\comArg{OrderName}}]
creates the log-file.
The name of the log-file is 
\comArg{ModuleName}\suffix{.}\comArg{OrderName}\suffix{.@.ord}.
It has the format of an order specification file which can be used
with the \comRef{in}-command or the \comRef{loadLog}-command. 
If \comRef{storeLog} is used only with argument \comArg{ModuleName}
the log-file is named \comArg{ModuleName}\suffix{.@.ord},
if \comRef{storeLog} is used without any argument, the file is
created as \nt{moduleName}\suffix{.}\nt{orderName}\suffix{.@.ord},
with names as they are currently associated with the specification.
\end{command}

\begin{command}[\com{loadLog}{\comArg{ModuleName}\ad\comArg{OrderName}}]
reads in the file \comArg{ModuleName}\suffix{.}\comArg{OrderName}\suffix{.@.ord}. 
%This file contains all the answer given during
%the completion process and the final termination ordering saved using
%the \comRef{saveLog}-command. 
If the completion process is started again, % all
questions whose answers are already contained in 
\comArg{ModuleName}\suffix{.}\comArg{OrderName}\suffix{.@.ord} 
will be suppressed. 
If \comRef{loadLog} is used without the argument \comArg{OrderName} the 
information will be taken from the file \comArg{ModuleName}\suffix{.@.ord}.
If \comRef{loadLog} is used without any argument
the name of the current specification together with the current order name
will be used.
%The current order name is the
%order name of the order specification for the current specification
%or the current termination ordering if no order specification was used.
\end{command}

\subsection{Assigning and Retrieving Specifications to/from Specification Variables }
\label{StoreCommand}
\label{LoadCommand}

Some operations on specifications like \comRef{combine}, cf. the
\comRef{combineSpecs}-command, need a way to reference different specifications. In CEC
this is done by storing specifications to named variables and referencing them
afterwards by these variable names. Variable names are arbitrary Prolog atoms.\bigskip

\begin{command}[\com{store}{\comArg{ModuleName}\ad\comArg{OrderName}}]
saves the current specification in a specification variable named
\comArg{ModuleName}\kw{.}\comArg{OrderName}. If \comRef{store} is used only with
argument \comArg{ModuleName} the specification is saved in a variable with this name,
if \comRef{store} is used without any argument, the name is created as 
\nt{moduleName}\kw{.}\nt{orderName}, with names as they are currently associated 
with the specification. For later restoring 
use the command \comArg{load}. The system remains unchanged except for this variable 
containing afterwards the current specification.
\end{command}

\begin{command}[\com{load}{\comArg{ModuleName}\ad\comArg{OrderName}}]
loads the system which is currently the value of the variable
\comArg{ModuleName}\kw{.}\comArg{OrderName}, cf. the \comRef{store}-command. 
If \comRef{load} is used only with argument \comArg{ModuleName}, this actual parameter 
completely specifies the name of the variable.
Specification variables remain unchanged.
\comArg{StateName} = \cec{'$initial'} re-initializes the system.
\end{command}

