%----------------------------------------------------------------------
% Macros

% Theorems with roman font
% To declare a theorem-environment using roman font style just say
% \newtheorem{NAME}{TEXT}[COUNTER]
% \def\NAME{\Thm{NAME}{TEXT}}
% or
% \newtheorem{NAME}[OLDNAME]{TEXT}
% \def\NAME{\Thm{OLDNAME}{TEXT}}
\catcode`\@=11
\def\Thm#1#2#3{\refstepcounter
{#2}\@ifnextchar[{\@Ythm{#1}{#2}{#3}}{\@Xthm{#1}{#2}{#3}}}
\def\@Xthm#1#2#3{\@Begintheorem{#1}{#3}{\csname the#2\endcsname}\ignorespaces}
\def\@Ythm#1#2#3[#4]{\@Opargbegintheorem{#1}{#3}{\csname
       the#2\endcsname}{#4}\ignorespaces}
\def\@Begintheorem#1#2#3{\trivlist \item[\hskip \labelsep{\bf #2\ #3}]#1}
\def\@Opargbegintheorem#1#2#3#4{\trivlist
      \item[\hskip \labelsep{\bf #2\ #3\ (#4)}]#1}
\catcode`\@=12

%---------- layout ----------

\newcommand{\nl}{\hfill\\}

%\def\definitionname{Definition}
%\def\theoremname{Theorem}
%\def\propositionname{Proposition}
%\def\corollaryname{Corollary}
%\def\lemmaname{Lemma}
%\def\conjecturename{Conjecture}
%\def\remarkname{Remark}
%\def\examplename{Example}
%\def\proofname{Proof}

\def\definitionname{Definition}%
\def\theoremname{Satz}%
\def\propositionname{Proposition}%
\def\corollaryname{Korollar}%
\def\lemmaname{Lemma}%
\def\conjecturename{Vermutung}%
\def\remarkname{Bemerkung}%
\def\examplename{Beispiel}%
\def\proofname{Beweis}%

\newtheorem{definition}{}
  \def\definition{\Thm{\rm}{definition}{\definitionname}}
\newtheorem{theorem}[definition]{\theoremname}
  \def\theorem{\Thm{\it}{definition}{\theoremname}}
\newtheorem{proposition}[definition]{\propositionname}
  \def\proposition{\Thm{\it}{definition}{\propositionname}}
\newtheorem{corollary}[definition]{\corollaryname}
  \def\corollary{\Thm{\it}{definition}{\corollaryname}}
\newtheorem{lemma}[definition]{\lemmaname}
  \def\lemma{\Thm{\it}{definition}{\lemmaname}}
\newtheorem{conjecture}[definition]{\conjecturename}
  \def\conjecture{\Thm{\it}{definition}{\conjecturename}}
\newtheorem{remark}[definition]{\remarkname}
  \def\remark{\Thm{\rm}{definition}{\remarkname}}
\newtheorem{example}[definition]{\examplename}
  \def\example{\Thm{\rm}{definition}{\examplename}}
\newenvironment{proof}{\smallskip{\em \proofname:}}{\hfill\(\Box\)\\\smallskip}

%---------- internal ----------
\newcommand{\inMath}[1]{\relax\ifmmode{#1}\else${#1}$\fi}

%---------- math ----------

% decl : name * type -> declaration
\newcommand{\decl}[2]{\inMath{{#1}:{#2}}}

% fun : domain * codomain -> function type
\newcommand{\fun}[2]{\inMath{{#1}\rightarrow{#2}}}

% fundecl : function name * domain * codomain -> function declaration
\newcommand{\fundecl}[3]{\decl{#1}{\fun{#2}{#3}}}

% argpos : underscore to indicate argument positions for infix etc.
\newcommand{\argpos}{\underline{\ }}


\newcommand{\implies}{\;\Rightarrow\;}
\newcommand{\dfeq}{\;:=\;}
\newcommand{\dfiff}{\;\:\Longleftrightarrow\;}

% eset : elements -> enumerated set
\newcommand{\eset}[1]{\inMath{\{{#1}\}}}

% pset : variable * predicate -> set
\newcommand{\pset}[2]{\inMath{\{{#1}|{#2}\}}}

% family : element of family * variable in set -> family
\newcommand{\family}[2]{\inMath{({#1})_{#2}}}

% union of sets
\newcommand{\union}[2]{\inMath{{#1}\,\cup\,{#2}}}

% intersection of sets
\newcommand{\intersection}[2]{\inMath{{#1}\,\cap\,{#2}}}

% difference of sets
\newcommand{\difference}[2]{\inMath{{#1}\backslash{#2}}}

% disjointness of sets
\newcommand{\disjoint}[2]{\inMath{\intersection{#1}{#2}=\emptyset}}


% card : set -> its cardinality
\newcommand{\card}[1]{\inMath{|{#1}|}}

\newcommand{\tuple}[1]{\inMath{\langle{#1}\rangle}}
% pair : sth * sth -> pair
\newcommand{\pair}[2]{\tuple{{#1},{#2}}}
% triple : sth * sth * sth -> triple
\newcommand{\triple}[3]{\tuple{{#1},{#2},{#3}}}

% index * from * to -> from =< index =< to
\newcommand{\inrange}[3]{\inMath{{#1}={#2}\ldots{#3}}}

%---------- categories ----------

% Obj : collection of objects in a category
% Mor : collection of morphisms in a category
\newcommand{\Obj}{{\rm Obj}}
\newcommand{\Mor}{{\rm Mor}}

% ObjOf : category -> collection of objects in category
% MorOf : category -> collection of morphisms in category
\newcommand{\ObjOf}[1]{{\Obj\;{#1}}}
\newcommand{\MorOf}[1]{{\Mor\;{#1}}}

% dom : morphism -> its domain
% cod : morphism -> its codomain
% id : object -> identity morphism
% comp : morphism * morphism -> composition of morphisms
\newcommand{\id}[1]{\inMath{\dot{#1}}}
\newcommand{\dom}[1]{\inMath{{#1}^{-}}}
\newcommand{\cod}[1]{\inMath{{#1}^{+}}}
\newcommand{\comp}[2]{\inMath{{#1};{#2}}}

\newcommand{\Cat}{\mbox{\rm CAT}}
\newcommand{\Set}{\mbox{\rm SET}}
\newcommand{\dual}[1]{\inMath{{#1}^{\rm op}}}

%---------- signatures ----------

\newcommand{\emptysig}{\inMath{\emptyset}}

\newcommand{\opdecl}[3]{\fundecl{#1}{#2}{#3}}
\newcommand{\preddecl}[2]{\decl{#1}{#2}}
\newcommand{\vardecl}[2]{\decl{#1}{#2}}
\newcommand{\sigtermdecl}[3]{\inMath{{#2}:_{#1}{#3}}}
\newcommand{\termdecl}[2]{\sigtermdecl{}{#1}{#2}}

\newcommand{\Sign}{\inMath{\rm SIGN}}
\newcommand{\EqSign}{\inMath{\rm EQSIGN}}

\newcommand{\Spec}{\inMath{\rm SPEC}}

\newcommand{\sort}[1]{{\it {#1}}}
\newcommand{\pred}[1]{{\it {#1}}}

\newcommand{\varsOf}[1]{{\it var({#1})}}

%---------- structures ----------

\newcommand{\AlgOf}[1]{\inMath{\rm {#1}\mbox{-}ALG}}
\newcommand{\StructOf}[1]{\inMath{\rm {#1}\mbox{-}STRUCT}}

\newcommand{\reduct}[1]{\inMath{\bar{#1}}}

\newcommand{\T}[1]{\inMath{T_{#1}}}
\newcommand{\TSig}{\T{\Sigma}}
\newcommand{\TSigX}{\T{\Sigma(X)}}
\newcommand{\TSigY}{\T{\Sigma(Y)}}

\newcommand{\Atoms}[1]{\inMath{At_{#1}}}

\newcommand{\HerbStruct}[2]{\inMath{HS({#1},{#2})}}

\newcommand{\evalAss}[1]{\inMath{{#1}^{\#}}}
\newcommand{\evalGnd}[1]{\inMath{\#_{#1}}}

%---------- specifications ----------

\newcommand{\eq}{\approx}
\newcommand{\eqS}{\dot{\approx}}
\newcommand{\clause}[2]{\inMath{{#1}\rightarrow{#2}}}
\newcommand{\vclause}[3]{\inMath{\forall{#1}.{#2}\rightarrow{#3}}}
\newcommand{\vuclause}[2]{\inMath{\forall{#1}.{#2}}}

%---------- models ----------

\newcommand{\InitOf}[1]{\inMath{{\cal I}({#1})}}

%---------- theories ----------

\newcommand{\ThOf}[1]{\inMath{{\rm Th}({#1})}}
\newcommand{\IThOf}[1]{\inMath{{\rm ITh}({#1})}}

\newcommand{\entails}{\inMath{\;\models\;}}
\newcommand{\indEntails}{\inMath{\;\models_I\;}}

%---------- terms ----------

\newcommand{\subterm}[2]{\inMath{{#1}/{#2}}}

%---------- rewriting ----------

\newcommand{\rew}{\rightarrow}
\newcommand{\rewR}{\stackrel{\epsilon}{\rightarrow}}
\newcommand{\rewT}{\stackrel{+}{\rightarrow}}
\newcommand{\rewRT}{\stackrel{*}{\rightarrow}}
\newcommand{\rewI}{\leftarrow}
\newcommand{\rewIR}{\stackrel{\epsilon}{\leftarrow}}
\newcommand{\rewIT}{\stackrel{+}{\leftarrow}}
\newcommand{\rewIRT}{\stackrel{*}{\leftarrow}}
\newcommand{\rewS}{\leftrightarrow}
\newcommand{\rewSR}{\stackrel{\epsilon}{\leftrightarrow}}
\newcommand{\rewST}{\stackrel{+}{\leftrightarrow}}
\newcommand{\rewSRT}{\stackrel{*}{\leftrightarrow}}

\newcommand{\rewG}{\diamondsuit}
\newcommand{\rewGRT}{\stackrel{*}{\diamondsuit}}

\newcommand{\rewD}{\downarrow}

%---------- proof ----------

\newcommand{\ProofsOf}[1]{{\rm Proofs}({#1})}
\newcommand{\trans}{\inMath{\vdash}}

\newcommand{\Proofs}{\inMath{{\cal P}}}
\newcommand{\iProofs}{\inMath{\Proofs_i}}
\newcommand{\diProofs}{\inMath{\Proofs_d}}

\newcommand{\proofGT}{\succ^P}
\newcommand{\proofGE}{\succeq^P}
\newcommand{\proofLT}{\prec^P}
\newcommand{\proofLE}{\preceq^P}

\newcommand{\cplGT}{\succ^c}
\newcommand{\cplGE}{\succeq^c}
\newcommand{\cplLT}{\prec^c}
\newcommand{\cplLE}{\preceq^c}

%---------- type predicates ----------

\newcommand{\tp}[1]{\inMath{\mbox{\it is-}{#1}}}
\newcommand{\tc}[1]{\inMath{{\it tc}({#1})}}
\newcommand{\TC}[1]{\inMath{{\it TC}({#1})}}
\newcommand{\TS}{\inMath{\it TS}}
\newcommand{\TF}[1]{\inMath{{\TS_{#1}}}}

% end of macros
%----------------------------------------------------------------------
