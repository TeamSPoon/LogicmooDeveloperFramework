\section{Operations on Specifications}
\label{OperationsOnSpecifications}

The specification building operators \kw{+} and \kw{rename} in 
module expressions can also be invoked interactively.
%As CEC is planned to be an experimental system for specification (program) development
%it offers several operations for combining specifications. The operations include the
%renaming of operator and sort names as well as the union of two specifications.

With these primitive operations more complex ones, such as passing actual parameters to
specification modules can be derived, cf. chapter \ref{exampleSession}.
%The parameter passing is
%demonstrated by the example of sorted lists over an arbitrary ordered set, that is
%replaced by the set of natural numbers \\
%(\refArrow file \file{cec/KBMANUAL/combining\_example} in this distribution).

\subsection{Renaming of Operator and Sort Names}

\begin{command}[\com{renameSpec}{\kw{[}\comArg{OldSort1} \kw{<-} \comArg{NewSort1}, \ldots , 
\comArg{OldSortN} \kw{<-} \comArg{NewSortN},\\ \hspace*{6.2em}
\comArg{OldOperator1} \kw{<-} \comArg{NewOperator1}, \ldots , 
\comArg{OldOperatorM} \kw{<-} \comArg{NewOperatorM}\/\kw{]}}]
renames the current specification according to the given lists of sort associations
and operator associations. Only injective renamings of operators are allowed.
Sorts may be renamed arbitrarily. Sorts and operators which are not mentioned remain 
unchanged.
\end{command}

%The system tries to carry over the  previous termination proof to the renamed
%specification. This is not always possible, as the collapsing of previously distinct
%operator symbols might not be compatible with precedences or abstract interpretations.
%In such a case, all rules will be turned back into equations.


\subsection{Combining of Specifications}

The combine operator forms the union of two specifications if they can be combined, 
that is if their termination orderings can proved to be compatible. If so, the two orderings
are combined. \cec{kns} and \cec{poly}\nt{N} are assumed to be incompatible.\bigskip

\begin{command}[\com{combineSpecs}{\comArg{StateName1}\ad\comArg{StateName2}\ad\comArg{CombinedSpec}}]
The specifications  stored (\refArrow \comRef{store}) 
in specification variables \comArg{StateName1} and \comArg{StateName2}
will  be combined, if possible, by forming
the union of the signature, axioms and pragmas.
If \comArg{CombinedSpec} $\neq$ \cec{user}, the result will be stored in
\comArg{CombinedSpec}, and the current specification will not be affected.
Otherwise, the combined specification becomes the new current
specification.
\comRef{combineSpecs} requires the compatibility of the termination
orderings of the involved specifications:
\begin{itemize}
\item
Termination orderings of the same type may be combinable, also \kw{kns} together with
\kw{neqkns}, yields \kw{kns} for the combined specification.
In the case of \kw{kns} and \kw{neqkns} 
the combination fails if the operator precedences are contradictory,
e.g.\ $f > g$ in the first specification but $g \geq f$ in the second 
specifcation. Also, different stati for operators are not compatible.
\item
\kw{poly}\nt{N} and \kw{poly}\nt{M} can be combined 
into \kw{poly}\nt{N} if $N \geq M$. In addition, operators 
which occur in both operand specifications must have
the same interpretation polynomials in the first M components.
\end{itemize}
\end{command}
